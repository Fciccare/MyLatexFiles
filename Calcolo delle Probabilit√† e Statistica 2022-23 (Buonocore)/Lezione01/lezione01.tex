\section{Lezione 01 - 06/03/2023}

\subsection{Il Gioco della Zara con 2 Dadi}
Prevede l'utilizzo di due dadi (nel gioco originale tre), a turno ogni giocatore chiama un numero e lancia i dadi.\\
Se la somma dei dadi è pari al numero scelto si vince.\\
2 dadi onesti danno luogo a 2 punteggi da 1 a 6: $P_1, P_2$.\\
Possiamo rappresentiamo graficamente le coppie di tutti i possibili casi:

\begin{center}
\begin{tabular}{ cc }
%Normale & Trasformata\\

\begin{tabular}{ |c|c|c|c|c|c| } 
 \hline
 (1,1) & (1,2) & (1,3) & (1,4) & (1,5) & (1,6) \\ 
 (2,1) & (2,2) & (2,3) & (2,4) & (2,5) & (2,6) \\ 
 (3,1) & (3,2) & (3,3) & (3,4) & (3,5) & (3,6) \\ 
 (4,1) & (4,2) & (4,3) & (4,4) & (4,5) & (4,6) \\ 
 (5,1) & (5,2) & (5,3) & (5,4) & (5,5) & (5,6) \\ 
 (6,1) & (6,2) & (6,3) & (6,4) & (6,5) & (6,6) \\ 
 \hline
\end{tabular} &

$\xrightarrow[]{Z_2}$

\begin{tabular}{ |c|c|c|c|c|c| } 
 \hline
 2 & 3 & 4 & 5 & 6 & 7 \\ 
 3 & 4 & 5 & 6 & 7 & 8 \\ 
 4 & 5 & 6 & 7 & 8 & 9 \\ 
 5 & 6 & 7 & 8 & 9 & 10 \\ 
 6 & 7 & 8 & 9 & 10 & 11 \\ 
 7 & 8 & 9 & 10 & 11 & 12 \\ 
 \hline
\end{tabular}\\

\end{tabular}
\end{center}
Possiamo notare che coppie possibili sono 36, poiché ogni dado ha 6 faccie, quindi $6^2=6*6=36$ possibili risultati.\\
Espriamo il "Lanciare i dadi" come $\xi$ (e tondo) cioè \textbf{ESPERIMENTO ALEATORIO}.\\
L'insieme dei possibili risultati di $\xi$ si può esprimere così: 
$$ \Omega=\{(i,j):i,j=1,2,..,6\}=\{(1,1), (1,2), ..., (6,6)\} $$
Questo insieme $\Omega$ (omega) prende il nome di \textbf{SPAZIO CAMPIONE}.\\
La coppia $ (i,j) \in \Omega $ è chiamato \textbf{PUNTO CAMPIONE}.\\  
Per ogni esper. ale. $\xi$ bisogna prendere una \textbf{FAMIGLIA DI EVENTI:} 
$$\text{(f tondo)}\:\:\mathcal{F} = \mathcal{P}(\Omega)$$
In questo caso tutti i possibili sottoinsiemi cioè l'insieme delle parti dello spazio campione. \\
$Z_2\footnotemark$ (Zara due) è una funzione che preso un punto campione restituisce la somma delle ordinate, è definita nel suguente modo:
$$ Z_2: \Omega \rightarrow \mathcal{R} $$
\footnotetext{Il pedice 2 sta ad indicare che stiamo considerando due dadi, è utile per distunguirlo da un eventuale $Z_3$, ma può essere anche omesso.}
\centerline{(tutte le funzioni finiscono sempre in $\mathcal{R}$)}
Come si può facilmente notare i risultati possibili sono compresi tra 2 e 12 (inclusi). \\
Possiamo formalizzarlo nel seguente modo:
$$ {S_Z}_2 = \{2,3,4,5,6,7,8,9,10,11,12\} $$
Questo insieme ${S_Z}_2$ prende il nome di \textbf{SPETTRO}.\\
La possibilità di trovare un numero non appartente a questo insieme è nulla.
\newpage
Per calcolare la probabiltà ci basta mettere a rapporto i seguenti dati:
\begin{equation*}
\frac{\text{\#\footnotemark OCCORRENZE DI N}}{\text{\# SPAZIO CAMPIONE}} =
\frac{\#{Z^{-1}_2}(\{N\})}{\#\Omega}
\end{equation*}
\footnotetext{\# indica la cardanalità, è usato come sostituto di $\parallel$}
Poniamo che voglia sapere la probabilità che la somma dei 2 dadi faccia 4, allora diremo che la \textbf{LA PROBABILITÀ DELL'EVENTO:}
\begin{equation*}
\mathcal{P}(Z=4) =
\frac{\#{Z^{-1}_2}(\{4\})}{\#\Omega} =
\frac{\# \{ (1,3),(2,2),(3,1) \} }{\#\Omega} =
\frac{3}{36}
\end{equation*}
\begin{center}
(l'antimmagine finisce sempre in $\mathcal{P}(\Omega)$ e mai in $\Omega$)
\end{center}
%Il 3 è stato ricavato contando quante volte appare 4 nella tabella.\\
Possiano notare che il numero con la più alta probabilità è il 7, poiché figura sei volte, quindi $ \frac{6}{36} $.\\
Possiamo rappresentare la probabilità di ogni numero dello spettro:
$$ \mathcal{P}(Z=2) = \frac{1}{36} = \mathcal{P}(Z=12) $$
$$ \mathcal{P}(Z=3) = \frac{2}{36} = \mathcal{P}(Z=11) $$
$$ \mathcal{P}(Z=4) = \frac{3}{36} = \mathcal{P}(Z=10) $$
$$ \mathcal{P}(Z=5) = \frac{4}{36} = \mathcal{P}(Z=9)  $$
$$ \mathcal{P}(Z=6) = \frac{5}{36} = \mathcal{P}(Z=8)  $$
$$ \mathcal{P}(Z=7) = \frac{6}{36} $$
%%Possiamo riassumere questo esprimento con la seguente terna:
%%$$ (\Omega, P(\Omega), P) \xrightarrow[]{Z} (R, , P_z) $$
Inoltre possiamo notare che a parte la diagonale secondaria, la matrice è speculare, cioé ogni numero opposto ha la stessa probabilità di uscire.\\ %possiamo dire che 12 è issoprimo di 2.
Possiamo verificare che la probabilità che esca un numero pari è uguale ai dispari:
$$ Pari = 2*(\frac{1}{36}) + 2*(\frac{3}{36}) + 2*(\frac{5}{36}) = \frac{18}{36} = \frac{1}{2}$$
$$ Dispari = 2*(\frac{2}{36}) + 2*(\frac{4}{36}) + 1*(\frac{6}{36}) = \frac{18}{36} = \frac{1}{2}$$\\
Possiamo affermare che, ogni probabilità è compresa tra 0 e 1 e che la
probabilità dello spazio campione è \textbf{sempre} uguale 1 (condizione di normalizzazzione), cioè la somma delle probabilità di tutti i valori dello spettro dello spazio campione ($\Omega$) deve essere uguale a 1.



 




