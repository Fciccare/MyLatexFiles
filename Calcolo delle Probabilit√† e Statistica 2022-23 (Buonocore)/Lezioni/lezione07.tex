\section{Lezione 07 - 22-03-2023}

\subsection{Soggettività}
La probabilità (soggetiva) di un evento è il prezzo p (compreso tra 0 e 1), che un individuo coerent ritiene "equo" pagare per ricevere 1 se l'evento A si verifica.\\
(Non sarà possibile con nessun insieme assicurarsi perdita o vincita certe).\\
Si fa uso di due principi:
\begin{itemize}
\item[equità)] Sia $A$ un evento sulla cui realizzazione scommettono due giocatori, di cui uno è il banco e l'altro e lo scommettitore. Supponiamo che spetti ad uno di essi, fissare la probabilità di $A$. Allora la scomessa è \textbf{equa} se al secondo giocatore è lasciata la possibilità di stabile se fare il banco oppure lo scommettitore.
(questo permette ad entrambi di non avere una vittoria/perdita certa)·
\item[coerenza)] Non è possibile individuare un sistema di
eventi che possa assicurare vantaggio/svantaggio all’individuo che pone
le probabilità.
\end{itemize}

\subsection{Recap Probabilità}
Quindi riassumento possiamo esprimere la probabilità tramite tre diverse definizioni:
\begin{itemize}
\item Classica/Laplace: Se c'è simmetria.
\item Frequentistica/Statistica: Se l'esperimento si può ripetere infinite volte.
\item Soggettiva: Quando l'esperimento si può eseguire una sola volta.
\end{itemize}

\subsection{Impostazione Assiomatica}
Gli eventi sono sottoinsiemi di uno spazio $\Omega$ e formano una $\sigma$-algebra $F$:
\begin{itemize}
\item[a)] $ F \neq \emptyset$
\item[b)] $ A \in F \Rightarrow A^C \in F $
\item[c)] $ \forall n \in F, \forall n \in N \Rightarrow \bigcup_{n=1}^{\infty} A_n \in F $
\end{itemize}
Una misura di probabilità sullo spazio $\Omega$ è una funzione $P: F -> R$ tale che:
\begin{itemize}
\item[d)] $\forall A \in F, P(A) >= 0 \:\:\: (\text{ma} < \infty$
\item[e)] $ P(\Omega) = 1 $
\item[f)] $ \text{se} \{A_n: n \in N \} \subseteq F: (i \neq j ) \: A_i \cap A_j \neq \emptyset \Rightarrow P(\bigcup_{n=1}^{\infty}A_n) = \sum_{n=1}^{\infty}P(A_n)$
\end{itemize}
La tripla $(\Omega, F, P)$ prende il nome di \textbf{spazio di probabilità}.

\subsection{Conseguenze immediate degli assiomi}
\subsubsection{Teorema 01}
Teorema 01: $ P(\emptyset) = 0 $\\
Dim:\\
Il vuoto è un evento in quanto complementare del certo. Inoltre il vuoto può essere visto come unione numerabile di insiemi vuoti (per una delle propietà di indentità):
$$ \emptyset = \emptyset \uplus \emptyset \emptyset \uplus ... = \uplus_{n=1}^{\infty} \emptyset $$
Dall'assioma $f)$ si ottiene allora:
$$ P(\emptyset) = P(\bigcup_{n=1}^{\infty} \emptyset = \sum_{n=1}^{\infty} P(\emptyset) = P(\emptyset) + P(\emptyset) + ...$$
e per l'assioma $d)$ l'unico numero che soddisfa la precedente relazione è $P(\emptyset)=0$

\subsubsection{Teorema 02}
Se $A_1 \in F, A_2 \in F, ..., A_n \in F$ e $A_i \cap A_j = \emptyset$ per $i \neq j$ allora:
$$ P(\uplus_{i=1}^n A_i) = \sum_{i=1}^n P(A_i)$$
Dim:\\
Poniamo $B_1 = A_1, B_2 = A_2,...,B_n = A_n$ e $B_{n+1} = B_{n+2} = ... = \emptyset$
Ovviamente riesce $B_i \cap B_j = \emptyset$ per $i \neq j$ per cui dall'assioma $f)$ si ha:
$$ P(\uplus_{i=i}^{\infty} B_i) = \sum_{i=i}^{\infty} P(B_i)$$
Dall'altra parte:
$$ \uplus_1^{\infty} B_i = B_1 \uplus B_2 \uplus ... \uplus B_n \uplus B_{n+1} \uplus B_{n+2} \uplus ... $$
$$ = (A_1 \uplus A_2 \uplus ... \uplus A_n) \cup (\emptyset \uplus \emptyset \uplus ...) $$
$$ = (A_1 \uplus A_2 \uplus ... \uplus A_n)\uplus \emptyset $$
$$ = \uplus{n=1}^n A_i $$
e:
$$ \sum_{i=1}^{\infty} P(B_i) = P(B_1) + P(B_2) + ... + P(B_n) + P(B_{n+1}) + ... $$ 
$$ = P(A_1) + P(A_2) + ... + P(A_n) + P(\emptyset) + P(\emptyset) + ... $$
$$ = \sum_{i=1}^n P(A_i) + (0+0+...)$$
$$ = \sum_{i=1}^n P(A_i) $$
La tesi segue da $(i), (ii), (iii)$.\\
La misura di probabilità $P$ e quindi anche finitamente additiva.

\subsubsection{Teorema 03}
$\forall A \in F, P(A^C) = 1 - P(A)$
Dim: \\
Per ogni $A \in F$ si ha $\Omega = A \cup A^C $ per cui essendo $P$ finitamente additivita si ha:
$$ P(\Omega) = P(A \cup A^C) = P(A)+P(A^C)$$
Ricordando l'assioma: $e) P(\Omega)=1$ otteniamo infine:
$$ 1 = P(A)+P(A^C) \Rightarrow P(A^C) = 1 - P(A)$$

\subsubsection{Teorema 04}
$\forall A \in F, P(A) <=1$
Dim:\\
Dal precedente risultato e dall'assioma $d)$ discende immediatamente:
$$ P(A)= 1-P(A^C) <= 1$$

\subsubsection{Teorema 05}
Se $A \in F$ e $B \in F$ sono tali che $A \subseteq B$ allora $P(A) <= P(B)$.
Dim: \\
Risulta allora:
$$ B = B \cap \Omega = B \cap (A \cup A^C) = (B \cap A) \uplus (B \cap A^C) = A \uplus (B \cap A^C)$$
in quanto dall'ipotesi $A \subseteq B$ si ha $B \cap A = A$.\\
Dalla finita additività di $P$ e dall'assioma $d)$ si ha che:
$$ P(B) = P(A) + P(B \cap A^C) >= P(A)$$


