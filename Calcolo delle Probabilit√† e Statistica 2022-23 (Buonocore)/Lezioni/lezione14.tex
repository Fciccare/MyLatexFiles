\section{Lezione 14 - 14/04/2023}

\subsection{Propietà della Controimmagine}
Consideriamo:
$$ f: S_1 \rightarrow S_2$$
$$s_1\rightarrow f(s_2)$$

elenchiamo le sue propietà:
\begin{itemize}
\item[1)] $f^{-1}(T) = \{f \in T\} = S$
\item[2)] $f^{-1}(\emptyset) = \{f \in \emptyset\} = \emptyset$
\item[3)] $B_1 \subseteq B_2 \subseteq B \Rightarrow f^{-1}(B_1) \subset  f^{-1}(B_2)$\\
\textbf{Dim:}\\
$s \in f^{-1}(B_1) \Rightarrow f(s) \in B_1 \Rightarrow f(s) \in B_2 \Rightarrow s \in f^{-1}(B_2)$
\item[4)] $f^{-1}(\bigcap_{i \in \mathbb{I}} B_i) = \bigcap_{i \in \mathbb{I}} f^{-1}(B_i)$\\
\textbf{Dim:}
$$ SE \; S \in f^{-1}(\bigcap_{i \in \mathbb{I}} B_i) \Leftrightarrow f(s) \in \bigcap_{i \in \mathbb{I}} B_i \Leftrightarrow$$
$$ i \in \mathbb{I}, f(s) \in B_i \Leftrightarrow i \in \mathbb{I}, s \in f^{-1}(B_i) \Leftrightarrow$$
$$ SE \; S \in f^{-1}(\bigcap_{i \in \mathbb{I}} B_i)$$
\item[5)] $f^{-1}(\bigcup_{i \in \mathbb{I}} B_i) = \bigcup_{i \in \mathbb{I}} f^{-1}(B_i)$\\
\item[6)] $B_1 \cap B_2 = \neq, f^{-1}(B_1) \cap f^{-1}(B_2) = \neq $ 
\end{itemize}

\subsection{Boreliani}