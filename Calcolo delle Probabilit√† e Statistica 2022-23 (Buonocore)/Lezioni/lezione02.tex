\section{Lezione 02 - 08/03/2023}

\subsection{Regola Moltiplicativa}
Se una procedura di scelta si può suddividere in $r$ sottoprocedure allora il numero $n$ delle possibili scelte è dato da:
$$ n = n_1*n_2*...*n_r$$
Dove $i=1,2,...,r$ rappresenta il numero delle possibili scelte nella sottoprecedura i-sima.\\
\subsubsection{Esempio Cartellini Camicie}
Vogliamo sapere quanti cartellini delle camicie dobbiamo fabbricare avendo i seguenti dati:
4 Taglie, 2 Foggie, 7 Colori.\\
Usando la regola moltiplicativa poniamo $r=3$ avendo tre possibili varianti, $n_1=4$ per le taglie, $n_2=2$ per le foggie, $n_3=7$ per i colori, ora calcoliamo il totale:
$$ n = n_1*n_2*n_3 = 4*2*7 = 56 \:\:\: \textbf{CARTELLINI} $$

\subsection{Fattoriale}
%Sia $n$ un intero positivo. Il prodotto dei primi $n$ interi positivi è chiamato fattoriale di n e si pone come
Il fattoriale di $n>=0$ si esprime come $n!$ ed è definita come il prodotto di tutti i numeri precendenti, definiamo tramite ricorsione:
\begin{equation*}
n! = 
\begin{cases}
1 \: \: \: \text{SE} \: \: \: n=0\\
n*(n-1)! \: \: \: \text{SE} \: \: \: n>0
\end{cases}
\end{equation*}
Esempio: 
$$6! = 1*2*3*4*5*6 = 720$$
$$ \frac{13!}{11!} = \frac{13*12*\cancel{11!}}{\cancel{11!}} = 13*12 = 156 $$
$$ \frac{n!}{(n-1)!} = \frac{n(n-1)!}{(n-1)!} = n $$
\newpage

\subsection{Coifficiente Binomiale}
Presi $n e k$ con $k<=n$, possiamo definire il cofficiente binomiale in questo modo:
$$ \binom{n}{k} = \frac{n!}{k!(n-k)!} $$
%Esempio:
$$ \binom{6}{4} = \frac{6!}{4!(6-4)!} = \frac{6!}{4!*2!} = \frac{6*5*\cancel{4!}}{\cancel{4!}*2!} = \frac{\cancel{6}^3*5}{\cancel{2}} = 3*5 = 15 $$

\subsubsection{Propietà del C.B. con esempi}
Andiamo ad elencare alcune propietà del coifficiente binomiale con i rispettivi esempi:
\begin{description}
  \item [Propietà 01] 
  	$$ \binom{n}{n} = 1 $$
	$$\binom{5}{5} = \frac{\cancel{5!}}{\cancel{5!}*\equalto{(5-5)}{0!=1}!} = 1 $$ 
	
  \item [Propietà 02] 
  	$$ \binom{n}{n-1} = n $$
	$$\binom{5}{4} = \frac{5*\cancel{4!}}{\cancel{4!}*\equalto{(5-4)}{1}!} = 5 $$ 
  \item [Propietà 03]
  $$ \binom{n}{k} = \binom{n}{n-k} $$
  $$ \binom{12}{4} = \frac{12!}{4!*\equalto{(12-4)!}{8!}} = \frac{\cancel{12}^{\cancel{3}}*11*\cancel{10}^5*9*\cancel{8!}}{\cancel{2}*\cancel{3}*\cancel{4}*\cancel{8!}} = 5*9*11 = 495 = \frac{12!}{8!*\equalto{(12-8)!}{4!}} = \binom{12}{8} $$
  \item [Propietà 04 Se $k<n$ ]
  $$ \binom{n}{k} = \binom{n-1}{k-1} + \binom{n-1}{k} $$
  $$ \binom{}{}$$
  \item [Propietà 05 ($n=6, k=3$)]
  $$ \binom{n+1}{k} = \binom{n}{k} + \binom{n}{k-1}  $$
\begin{equation*}
\resizebox{\textwidth}{!}
{%
$\binom{7}{3} = \frac{7*\cancel{6}*5*\cancel{4!}}{\cancel{3!}*\cancel{4!}} = 7*5 = 35 = 20+15 =\frac{\cancel{2}*\cancel{3}*4*5*\cancel{6}}{\cancel{6} * \cancel{6}} + \frac{\cancel{6}^3*5*\cancel{4!}}{\cancel{2}*\cancel{4!}} = \frac{6!}{3!*3!} + \frac{6!}{2!*4!} = \binom{6}{3} + \binom{6}{2}$%
}
\end{equation*}
\end{description}

Un possibile uso del coifficiente binomiale è quello di poter sapere il numero dei sottoinsiemi di ordine $k$ con $n$ valori.\\
Esempio poniamo di avere un insieme $S=\{1,2,3,4\}$ con cardilinità $\#S = 4$, vogliamo sapere quanti sono tutti i possibili sottoinsiemi di ordine due:

$$ \binom{4}{2} = \frac{4!}{2!*(4-2)!} = \frac{\cancel{4}^2*3*\cancel{2!}}{\cancel{2}*\cancel{2!}} = 2*3 = 6$$

$$ T={ \{1,2\}, \{1,3\}, \{1,4\}, \{2,3\}, \{2,4\}, \{3,4\}} \: \: \#T=6 $$


\subsection{Coifficiente Multinomiale}
Sia $n$ un intero posi+tivo e $n_1,n_2...n_r$ interi tali che $n_1+n_2+...+n_r = n$, possiamo scrivere il coifficiente multinomilae in questo modo:
$$ \binom{n}{n_1,n_2,...,n_r} = \frac{n!}{n_1!*n_2!*...*n_r!} $$
%Esempio:
$$ \binom{7}{2,3,2} = \frac{7!}{2!*3!*2!} = \frac{7*6*5*\cancel{4}*\cancel{3!}}{\cancel{4}*\cancel{3!}} = 210 \:\:\:(2+3+2 = 7) $$
\newpage
\subsection{Problema del Contare}
Sia $S$ un insieme costituito da un numero $n$ finito di elementi distinti. In problemi coinvolgenti la selezione occorre distungere il caso in cui questa è effettuata con o senza ripetizioni. Si può inoltre porre o meno l'attenzione sull'ordine con cui gli elementi di S si presentano nella selezioni.

\subsection{Disposizioni e Combinazioni}
Per ovviare al problema del contare andiamo a definire le seguenti classificazioni:\\\\
\textbf{Disposizione:} è una selezione dove l'ordinamento è \textbf{IMPORTANTE}.\\
Possiamo suddividerla in:\\
Disposizione: è ammessa la \textbf{ripetizione} di qualunque elemento\\
Diposizione Semplice: \textbf{non è amessa} la ripezioni\\\\
\textbf{Combinazioni: } è una selezione dove l'ordinamente \textbf{non è IMPORTANTE}.\\
Possiamo suddividerla in:\\
Combinazioni: è ammessa la \textbf{ripetizione} di qualunque elemento\\
Combinazioni Semplice: \textbf{non è amessa} la ripezioni\\\\

\subsection{Disposizioni semplici/ripetizioni}
\label{sec:disposizioni}
Per calcolare tutte le k-disposizioni con ripetizione di S usiamo questa formula:
$$ D^{(r)}_{n,k} = n^k$$ 
%$$ D^{(r)}_{n,k} = n^k \: \: \: \: \: (k>=n)$$ 

Per calcolare tutte le k-disposizioni semplici di S usiamo questa formula:

$$ D_{n,k} = \frac{n!}{(n-k)!} \: \: \: \: \: (k<=n)$$ 

\begin{center}
($n$ cardinalità dell'insieme, $k$ la lunghezza della disposizione)
\end{center}

\subsubsection{Esempio di Disposizione}
Poniamo caso di voler sapere le possibili di dispozioni normali e semplici di un dato insieme di lettere.
Per semplicità consideriamo l'insieme $S=\{c,a\}$, poniamo caso che vogliamo sapere tutte le possibili parole di lunghezza $2$.\\
Quindi $n = \#S = 2$ e $k = 2$, allora:

$$ D^{(r)}_{n,k} = n^k = 2^2 = 4 = \{(c,c),(a,a),(c,a),(a,c)\} $$
$$ D_{n,k} = \frac{n!}{(n-k)!} = \frac{2!}{0!} = 2! = 2 = \{(c,a), (a,c)\}$$ 


















