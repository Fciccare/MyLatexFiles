\section{Lezione 09 - 27-03-2023}

\subsection{Problema delle Concordanze (formula inclusione-esclusione)}
Sia $n \in \mathbb{N}$ il numero dei cartoncini numerati da $1$ ad $n$ andiamo ad indicarli con $k$.\\
Si verifica una \textbf{concordanza} quando la $i$-esimo cartoncini porta $i$.\\
Esempio: se il quinto cartoncino riporta il numero $5$ allora 	questa è una concordanza.\\
Vogliamo verificare tre casi:
\begin{itemize}
\item[1)] Determinare la probabilità di avere 0 concordanze
\item[2)] Determinare la probabilità di avere almeno 1 concordanza
\item[3)] Determinare la probabilità di avere esattamente 1 concordanza
\end{itemize}
Definiamo un paio di boh:
$$ E_{k,n}: \; \text{"Esattamente k-concordanze"} $$
$$ E_{0,n}: \; \text{"Nessuna Concordanze"} $$
$$ A: \; \text{"Almeno una corcordanza"} $$
$$ A^c = E_{0,n}: \; \text{Il negato di "almeno una corcanza" è "nessuna ..."}$$
$$ C_i: \; \text{"Concordanza alla chiamata i-esima"} $$
\subsubsection{Probablità di avere almeno 1 corcordanza}
Possiamo definire insiemisticamente \textbf{almeno una corcondanza}, nel segunte modo: 
$$ A: \bigcup_{i=i}^n C_i$$
$$ \text{almeno uno in insiemestica (tipo or) è: } \cup $$
Per esprimere la probabilità di $A$ dobbiamo usare la formula di \textbf{unione-esclusione}:
$$ P(A): \sum_{i=1}^n \mathbb{P}(C_i) - \sum_{i<j}^n \mathbb{P}(C_i \cap C_j) + \sum_{i<j<k}^n \mathbb{P}(C_i \cap C_j \cap C_k) + ... + (-1)^{n+1} \mathbb{P} (C_1 \cap ... \cap C_n) $$
Dato che siamo troppo vagi, andiamo a esprimere la probabilità dei vari termini tramite Laplace:
$$ \mathbb{P}(C_i) = \frac{(n-1)!}{n!} = ... = \frac{1}{n} $$
Il denomitore è $n!$ poiché sono tutte le possibili mischiate.\\
Il numeratore invece è $(n-1)!$ poiché abbiamo una posizione fissate e le altre $n-1$ libere.\\
Se vogliamo fare un esempio concreto se consideriamo $\mathbb{P}(C_1)$ quindi si verifica una concordanza al primo posto cioè quando alla prima alzata corrisponde il numero $1$, abbiamo la prima posizione bloccata (dalla concordanza) e le altre $n-1$ posizioni con numeri liberi, possiamo generalizzare questo caso a tutti i numeri quindi ad $i$.\\
Proseguendo col secondo membro applichiamo lo stesso ragionamento:
$$ \mathbb{P}(C_i \cap C_j) = \frac{(n-2)!}{n!} = \frac{(n-2)!}{n*(n-1)*(n-2)!} = \frac{1}{n(n-1)}  $$
Abbiamo applicato la stesso ragionamento cioè, se consideriamo $ P(C_1 \cap C_2)$ (ricordiamo che $\cap$ vuol dire and), quindi abbiamo il numero $1$ alla prima alzata e il numero $2$ alla seconda alzata, quindi $2$ posti occupati e $n-2$ posti liberi.\\
Quindi possiamo applicare questo ragionamento con $1,2,3...n$ elementi.\\
Quindi andando a sostituire:
$$ \mathbb{P}(A) = \frac{1}{n} \sum_{i=i}^n 1 - \frac{(n-2)!}{n!} * \sum_{i=i}^n \sum_{j=i+1}^n 1 + \frac{(n-3)!}{n!} * \sum_{i=1}^n \sum_{j=i+1}^n \sum_{k=j+1}^n 1 + ... + (-1)^{n+1} * \frac{(n-n)!}{n!} $$
Possiamo fare un paio di considerazioni:
$$ \sum_{i=i}^n 1 = n $$
$$ \sum_{i=i}^n \sum_{j=i+1}^n 1 $$
Quest'ultimo possiamo considerarlo come le $2-$selezioni senza ripetizioni (semplici) in cui conta l'ordine quindi:
$$ \sum_{i=i}^n \sum_{j=i+1}^n = \binom{n}{2} $$
Questo ragionamento possiamo applicare a tutte le sommatorie, quindi:
$$ \mathbb{P}(A) = \frac{1}{n} * n - \frac{(n-2)!}{n!} * \binom{n}{2} + \frac{(n-3)!}{n!} * \binom{n}{3} + ... + (-1)^{n+1} \frac{(n-n)!}{n!} $$
Svogliamo i binomiale (aggiungo quadre per mantenere ordine):
$$ \mathbb{P}(A) = \frac{n}{n} - [ \frac{(n-2)!}{n!} * \frac{n!}{2!(n-2)!} ] +  [ \frac{(n-3)!}{n!} * \frac{n!}{3! (n-3)!} ] + ... + (-1)^{n+1} \frac{(n-n)!}{n!} $$
Facciamo le varie semplificazioni:
$$ \mathbb{P}(A) = 1 - \frac{1}{2!} + \frac{1}{3!} + ... + (-1)^{n+1} \frac{1}{n!} $$
Possiamo esprimere questa formula finale anche col simbolo di sommatoria:
$$ \mathbb{P}(A) = \sum_{i=1}^n ((-1)^{i+1} * \frac{1}{i!}) $$
Piccola considerazione:
$$ \textbf{cose sul numero di nepero ecc... da aggiungere} $$
\subsubsection{Probabilità 0 concordanze}
Piccolo recap:\\
Siamo riusciuti a ricavare la formula per verificare \textbf{la probabilità di avere almeno 1 concordanza}, da qui segue banalemente per come abbiamo detto prima cioè che la negazione di "almeno una" è "nessuna" la \textbf{la probabilità di avere 0 concordanze}:
$$ \mathbb{P}(A^c) = \mathbb{P}(E_{0,n}) = 1 - \mathbb{P}(A) $$

\subsubsection{Probabilità di avere K concordanze}
Da aggiungere quando carica i foglietti

\subsection{Proseguimento Teoremi Eventi non Incompatibili}
\subsubsection{Teorema 09}
Sia $A_n \in F$, $\forall n \in N$.\\
Per ogni intero $k$ si ha:
$$ P(\bigcup_{n=1}^k A_n) <= \sum_{n=1}^k P(A_n) $$
Dim per induzione:\\
Poniamo la base induttiva $k=2$, la relazione è vera, infatti da:
$$ A_1 \cup A_2 = A_1 \cup (A_2 \cap A_1^C) $$
per la finita additività di $P$ si ottiene:
$$ P(A_1 \cup A_2) = P(A_1) + P(A_2 \cap A_1^C) <= P(A_1) + P(A_2) $$
L'ultimo passaggio deriva dall'essere:
$$ A_2 \cap A_1^C \subseteq A_2 $$
Supponiamo ora la tesi vera per $k-1$ (passo induttivo):
$$ P(\bigcup_{n=1}^k A_n) = P[(\bigcup_{n=1}^k A_n) \cup A_k ] <= P(\bigcup_{n=1}^{k-1} A_n)+P(A_k) <= $$
$$ <= \sum_{n=1}^{k-1} P(A_n) + P(A_k) = \sum_{n=1}^k P(A_n)$$
La tesi è vera per il principio di induzione matematica.

\subsubsection{Teorema 10 (Disuguaglianza di Boole) [DA RIVEDERE UN PO']}
Sia $A_n \in F$, $\forall n \in N$. Si ha:
$$  P(\bigcup_{n=1}^{\infty} A_n) <= \sum_{n=1}^{\infty} P(A_n) $$
Dim:\\
Poniamo:
$$ B_1 = A_1 $$
$$ B_2 = A_2 \cap A_1^C $$
$$ B_3 = A_3 \cap (A_1^c \cap A_2^c) $$
$$ ... $$
$$ B_n = A_n \cap (A_1^c \cap A_2^c ... \cap A_{n-1}^c) = A_n \cap $$

\subsection{Eventi quasi Impossibili/Certi}










