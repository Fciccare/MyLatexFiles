\section{Lezione 04 - 15/03/2023}
\subsection{Riassunto Algebra}
\resizebox{\columnwidth}{!}{
\begin{tabular}{ |c|c|c| } 
 \hline
 Begin & Algebra degli Insiemi \\ 
 $ \emptyset $ & Insieme Vuoto \\
 $ \mathbb{N} $ & Interi positivi (senza zero)\\ 
 $ \mathbb{N}_0 $ & Numeri Naturali (con zero)\\ 
 $ \mathbb{Z}$ & Numeri Relativi \\
 $ \mathbb{Q} $ & Numeri Razionali \\ 
 $ \mathbb{R} $ & Numeri Reali \\ 
 $\Omega$ & Insieme universo \\ 
 $A$ & Insieme \\
 $A \cup B$ & Unione di A e B \\
 $A \setminus B $ & Differenza tra A e B \\
 $A^C$ & Complementare di A  \\
 $ A \cap B $ & Intersezione tra A e B \\
 $ A \subset B $ & A contenuto in B  \\
 $ ]a,b[ $ & Intervallo aperto \\
 $ [a,b] $ & Intervallo chiuso \\
 
 \hline
\end{tabular}
}
Le operazioni di uninione e intersezione hanno propietà di idempotenza, associtività, commutatività, distrubutività, identità, complementanzione, de morgan.

\newpage

\subsection{Cardinalità Insiemi}
\subsubsection{Insieme Finito}
Un insieme è \textbf{finito} se è possibile mettere ogni elemento dell'insieme in corrispondenza biuniva.
$$ |\Omega| = \#\Omega = n $$
\subsubsection{Insieme Numerabile}
$\Omega$ si dice \textbf{numerabile} se è possibile mettere ogni elemento dell'insieme in corrispondenza biuniva con $\mathbb{N}=\{1,2,...\}$
$$ |\Omega| = \#\Omega = \alpha_0 $$
\subsubsection{Insieme Continuo}
$\Omega$ si dice \textbf{continuo} se non ne finito ne numerabile
$$ |\Omega| = \#\Omega = c $$

\subsection{Classi (Famiglie)}
Quando gli elementi di un insieme $a$ sono a loro volta degli insiemi si usa per $a$ la parola \textbf{classe}.
$$ a = \{ \{2,3\},\{2\},\{5,6\} \}$$
In particolare se $\Omega$ è un insieme, la classe di tutti i sottinisiemi di $\Omega$ si dice l'insieme delle parti di $\Omega$ e si indica con $P(\Omega)$.\\
Se $\Omega$ è un insieme e $a$ è una classe di sottinsimi di $\Omega$ tale che l'unione di essi ha come risultato $\Omega$ allora $a$ è detta essere un \textbf{ricoprimento} di $\Omega$.\\
Un ricoprimento $a$ di $\Omega$ è detto essere una \textbf{partizione} di $\Omega$ se i suoi elementi a due a due disgiunti.\\
Esempio:
 $$ \Omega= \{1,2,3,4,5,6,7,8,9\} $$
 $$ a = \{ \{1,3,5\},\{2,6\},\{4,7\},\{7,8,9\} \} \:\:\: \text{è un ricomprimento ma non una partizione}$$
 $$ a = \{ \{1,3,5\},\{2,4,6,8\},\{7,9\} \} \:\:\: \text{è partizione poiché  tutti gli insiemi sono disgiunti}$$
