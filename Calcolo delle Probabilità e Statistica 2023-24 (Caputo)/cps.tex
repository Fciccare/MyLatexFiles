\documentclass[12pt,a4paper]{article}
\usepackage[a4paper, total={6in, 9.5in}]{geometry}
\usepackage[utf8]{inputenc}
\usepackage[italian]{babel}
\usepackage{eucal}
\usepackage{cancel}
\usepackage{svg}
\usepackage[utf8]{inputenc}
\usepackage{graphicx}
\graphicspath{ {./images/} }
\usepackage{import}
\usepackage[T1]{fontenc}
\usepackage{blindtext}
%\usepackage{amsmath}
\usepackage{mathrsfs}
\usepackage{amsfonts}
\usepackage{mathtools}
\usepackage{systeme}
\usepackage{eqparbox}
\newcommand{\verteq}{\rotatebox{90}{$\,=$}}
\newcommand{\equalto}[2]{\underset{\scriptstyle\overset{\mkern4mu\verteq}{#2}}{#1}}
\newcommand{\mystackrel}[3][T]{\stackrel{\eqmakebox[#1]{\scriptsize#2}}{#3}}
%\usepackage{lipsum}
\usepackage{hyperref}
\usepackage{soul}
\usepackage{color}
\DeclareRobustCommand{\hlcyan}[1]{{\sethlcolor{cyan}\hl{#1}}}
\hypersetup{linktoc=all}
\title{Calcolo delle probabilità e Statistica 2023-24\\ (G. Caputo)}
\author{}
\date{}

\newcommand{\cupdot}{\mathbin{\mathaccent\cdot\cup}}

\begin{document}
\maketitle
\tableofcontents


\section{Lezione 01 - XX/03/2023}

\subsection{Definizioni di base}

\subsubsection{Prodotto Cartesiano}

Presi $S,T \neq \emptyset$, possiamo definire il prodotto cartesiano:
\begin{equation*}
SxT = \{(s,t)/ s \in S, t \in T\}
\end{equation*}
\begin{equation*}
S^2 = SxS = \{(s,t)/ s \in S, t \in T\}
\end{equation*}
Da non confendere con la definizione di diagonale: $ S^2 = SxS = \{(s,s)/ s \in S\} $.

\subsubsection{Coppie}
La definizione di coppia è la seguente:
\begin{equation*}
(s,t) = \{\{s,t\}, \{s\}\}
\end{equation*}

Negli insiemi l'ordine non conta $ \{s,t\} = \{t,s\}$, invece nelle coppie è rilevante, infatti due coppie sono uguali se e solo  sono ordinatamente uguali:
$$ (s,t) = (s^\prime, t^\prime) \Leftrightarrow s=s^\prime, t=t^\prime $$

Andiamo a dimostrare questa affermazione: 

\begin{itemize}
\item DIM $\Leftarrow$: BANALE
\item DIM $\Rightarrow$
$ (s,t) = (s^\prime, t^\prime) \Leftrightarrow \{\{s,t\}, \{s\}\} = \{\{s^\prime,t^\prime\}, \{s^\prime\}\} $\\
Ragioniamo per casi:
	\begin{itemize}
		\item[a] SE $s=t$:\\
			$$ \text{Sx:} \{\{s,t\}, \{s\}\} \Rightarrow \{\{s,s\},\{s\}\} \Rightarrow \{s\} $$
			$$ \text{Dx:} \{\{s^\prime,t^\prime\}, \{s^\prime\}\} \Rightarrow \{\{s^\prime,s^\prime\},\{s^\prime\}\} \Rightarrow \{s^\prime\} $$
		\item[b] SE $s \neq t$:\\
			Usiamo le definizioni di uguaglianza tra insiemi:
			$$\{s\} = \{s^\prime\} \Rightarrow s=s^\prime$$
			$$\{s,t\} = \{s^\prime,t^\prime\} \wedge s=s^\prime \:\: \Rightarrow t = t^\prime$$
	\end{itemize}
\end{itemize}

% $ a \bullet b $



\subsubsection{Operaziona Interna}



\subsubsection{Operaziona Esterna}

\subsubsection{Prodotto Scalare Standard}

\subsubsection{Matrice in R}


\section{Lezione 02 - 08/03/2023}

\subsection{Regola Moltiplicativa}
Se una procedura di scelta si può suddividere in $r$ sottoprocedure allora il numero $n$ delle possibili scelte è dato da:
$$ n = n_1*n_2*...*n_r$$
Dove $i=1,2,...,r$ rappresenta il numero delle possibili scelte nella sottoprecedura i-sima.\\
\subsubsection{Esempio Cartellini Camicie}
Vogliamo sapere quanti cartellini delle camicie dobbiamo fabbricare avendo i seguenti dati:
4 Taglie, 2 Foggie, 7 Colori.\\
Usando la regola moltiplicativa poniamo $r=3$ avendo tre possibili varianti, $n_1=4$ per le taglie, $n_2=2$ per le foggie, $n_3=7$ per i colori, ora calcoliamo il totale:
$$ n = n_1*n_2*n_3 = 4*2*7 = 56 \:\:\: \textbf{CARTELLINI} $$

\subsection{Fattoriale}
%Sia $n$ un intero positivo. Il prodotto dei primi $n$ interi positivi è chiamato fattoriale di n e si pone come
Il fattoriale di $n>=0$ si esprime come $n!$ ed è definita come il prodotto di tutti i numeri precendenti, definiamo tramite ricorsione:
\begin{equation*}
n! = 
\begin{cases}
1 \: \: \: \text{SE} \: \: \: n=0\\
n*(n-1)! \: \: \: \text{SE} \: \: \: n>0
\end{cases}
\end{equation*}
Esempio: 
$$6! = 1*2*3*4*5*6 = 720$$
$$ \frac{13!}{11!} = \frac{13*12*\cancel{11!}}{\cancel{11!}} = 13*12 = 156 $$
$$ \frac{n!}{(n-1)!} = \frac{n(n-1)!}{(n-1)!} = n $$
\newpage

\subsection{Coifficiente Binomiale}
Presi $n e k$ con $k<=n$, possiamo definire il cofficiente binomiale in questo modo:
$$ \binom{n}{k} = \frac{n!}{k!(n-k)!} $$
%Esempio:
$$ \binom{6}{4} = \frac{6!}{4!(6-4)!} = \frac{6!}{4!*2!} = \frac{6*5*\cancel{4!}}{\cancel{4!}*2!} = \frac{\cancel{6}^3*5}{\cancel{2}} = 3*5 = 15 $$

\subsubsection{Propietà del C.B. con esempi}
Andiamo ad elencare alcune propietà del coifficiente binomiale con i rispettivi esempi:
\begin{description}
  \item [Propietà 01] 
  	$$ \binom{n}{n} = 1 $$
	$$\binom{5}{5} = \frac{\cancel{5!}}{\cancel{5!}*\equalto{(5-5)}{0!=1}!} = 1 $$ 
	
  \item [Propietà 02] 
  	$$ \binom{n}{n-1} = n $$
	$$\binom{5}{4} = \frac{5*\cancel{4!}}{\cancel{4!}*\equalto{(5-4)}{1}!} = 5 $$ 
  \item [Propietà 03]
  $$ \binom{n}{k} = \binom{n}{n-k} $$
  $$ \binom{12}{4} = \frac{12!}{4!*\equalto{(12-4)!}{8!}} = \frac{\cancel{12}^{\cancel{3}}*11*\cancel{10}^5*9*\cancel{8!}}{\cancel{2}*\cancel{3}*\cancel{4}*\cancel{8!}} = 5*9*11 = 495 = \frac{12!}{8!*\equalto{(12-8)!}{4!}} = \binom{12}{8} $$
  \item [Propietà 04 Se $k<n$ ]
  $$ \binom{n}{k} = \binom{n-1}{k-1} + \binom{n-1}{k} $$
  $$ \binom{}{}$$
  \item [Propietà 05 ($n=6, k=3$)]
  $$ \binom{n+1}{k} = \binom{n}{k} + \binom{n}{k-1}  $$
\begin{equation*}
\resizebox{\textwidth}{!}
{%
$\binom{7}{3} = \frac{7*\cancel{6}*5*\cancel{4!}}{\cancel{3!}*\cancel{4!}} = 7*5 = 35 = 20+15 =\frac{\cancel{2}*\cancel{3}*4*5*\cancel{6}}{\cancel{6} * \cancel{6}} + \frac{\cancel{6}^3*5*\cancel{4!}}{\cancel{2}*\cancel{4!}} = \frac{6!}{3!*3!} + \frac{6!}{2!*4!} = \binom{6}{3} + \binom{6}{2}$%
}
\end{equation*}
\end{description}

Un possibile uso del coifficiente binomiale è quello di poter sapere il numero dei sottoinsiemi di ordine $k$ con $n$ valori.\\
Esempio poniamo di avere un insieme $S=\{1,2,3,4\}$ con cardilinità $\#S = 4$, vogliamo sapere quanti sono tutti i possibili sottoinsiemi di ordine due:

$$ \binom{4}{2} = \frac{4!}{2!*(4-2)!} = \frac{\cancel{4}^2*3*\cancel{2!}}{\cancel{2}*\cancel{2!}} = 2*3 = 6$$

$$ T={ \{1,2\}, \{1,3\}, \{1,4\}, \{2,3\}, \{2,4\}, \{3,4\}} \: \: \#T=6 $$


\subsection{Coifficiente Multinomiale}
Sia $n$ un intero posi+tivo e $n_1,n_2...n_r$ interi tali che $n_1+n_2+...+n_r = n$, possiamo scrivere il coifficiente multinomilae in questo modo:
$$ \binom{n}{n_1,n_2,...,n_r} = \frac{n!}{n_1!*n_2!*...*n_r!} $$
%Esempio:
$$ \binom{7}{2,3,2} = \frac{7!}{2!*3!*2!} = \frac{7*6*5*\cancel{4}*\cancel{3!}}{\cancel{4}*\cancel{3!}} = 210 \:\:\:(2+3+2 = 7) $$
\newpage
\subsection{Problema del Contare}
Sia $S$ un insieme costituito da un numero $n$ finito di elementi distinti. In problemi coinvolgenti la selezione occorre distungere il caso in cui questa è effettuata con o senza ripetizioni. Si può inoltre porre o meno l'attenzione sull'ordine con cui gli elementi di S si presentano nella selezioni.

\subsection{Disposizioni e Combinazioni}
Per ovviare al problema del contare andiamo a definire le seguenti classificazioni:\\\\
\textbf{Disposizione:} è una selezione dove l'ordinamento è \textbf{IMPORTANTE}.\\
Possiamo suddividerla in:\\
Disposizione: è ammessa la \textbf{ripetizione} di qualunque elemento\\
Diposizione Semplice: \textbf{non è amessa} la ripezioni\\\\
\textbf{Combinazioni: } è una selezione dove l'ordinamente \textbf{non è IMPORTANTE}.\\
Possiamo suddividerla in:\\
Combinazioni: è ammessa la \textbf{ripetizione} di qualunque elemento\\
Combinazioni Semplice: \textbf{non è amessa} la ripezioni\\\\

Per calcolare tutte le k-disposizioni con ripetizione di S usiamo questa formula:
$$ D^{(r)}_{n,k} = n^k$$ 
%$$ D^{(r)}_{n,k} = n^k \: \: \: \: \: (k>=n)$$ 

Per calcolare tutte le k-disposizioni semplici di S usiamo questa formula:

$$ D_{n,k} = \frac{n!}{(n-k)!} \: \: \: \: \: (k<=n)$$ 

\begin{center}
($n$ cardinalità dell'insieme, $k$ la lunghezza della disposizione)
\end{center}

\subsubsection{Esempio di Disposizione}
Poniamo caso di voler sapere le possibili di dispozioni normali e semplici di un dato insieme di lettere.
Per semplicità consideriamo l'insieme $S=\{c,a\}$, poniamo caso che vogliamo sapere tutte le possibili parole di lunghezza $2$.\\
Quindi $n = \#S = 2$ e $k = 2$, allora:

$$ D^{(r)}_{n,k} = n^k = 2^2 = 4 = \{(c,c),(a,a),(c,a),(a,c)\} $$
$$ D_{n,k} = \frac{n!}{(n-k)!} = \frac{2!}{0!} = 2! = 2 = \{(c,a), (a,c)\}$$ 



















\section{Lezione 03 - 21-09-2023}

\subsection{Algoritmo v3}
L'algoritmo può essere anche migliorato, riusciendo ad arrivare ad una complessità \textbf{lineare}, nel seguente modo:

\begin{lstlisting}[language=Python]
int Max_seq_sum_3(int N, array a[])
	maxsum = 0
	sum = 0
	for j=1 to N
		if (sum + a[j] > 0) then
			sum = sum + a[j]
		else
			sum = 0
		maxsum = max(maxsum,sum)
	return maxsum
\end{lstlisting}
Il ragionamento è il seguente:
Se prendiamo un insieme di numeri da sommare, (da i ad a), possiamo controllare se esso è positivo o negativo.
Nel caso in cui $\sum_{e=i}^{a} A[e]$ risultasse positiva, andiamo a espandere il nostro range fintantochè il risultato della sommatoria riamanga positivo.
Nel caso in cui invece il risultato fosse negativo, non ci conviene tenere traccia dei numeri più piccoli di quel range, dato che se quella sommatoria è minore del numero successivo alla sommatoria, non ha senso tenerne conto. E quindi invece ha senso tenere traccia del numero successivo. Da quel numero poi sommare i numeri successivi continuando il processo sopracitato.


\subsection{Strutture Dati - Insieme Dinamico}
Vediamo come rappresentare un insieme di dati dinamico $S$ (con insieme dinamico si intende una collezione di elementi variabile nel tempo, quindi è possibile aggiungere o rimuovere elementi);%Rubata da Simone Cerrone
$$ S=\{a_1,a_2,...,a_n\} \;\;\; n \ge 0  $$
Andiamo a definire alcune operazioni:
\begin{itemize}
\item Insert$(S,a) \rightarrow S^{\prime}$ ($ S^{\prime} = S \cup \{x\}$)
\item Deletes$(S,a) \rightarrow S^{\prime}$ ($S^{\prime} = S \backslash \{x\}$)
\item Search$(S,a) \rightarrow \{True,False\}$
\item Massimo$(S) \rightarrow a$
\item Minimo$(S) \rightarrow a$
\item Successore$(S,a) \rightarrow a^{\prime}$
\item Predecessore$(S,a) \rightarrow a^{\prime}$
\end{itemize}
\section{Lezione 04 - 15/03/2023}
\blindtext
\section{Lezione 05 - 16-03-2023}

\subsection{Definizioni simboli Insiemestici ed Eventi}
\begin{center}

\begin{tabular}{ |c|c|c| } 
 \hline
 Begin & Algebra degli Insiemi & Logica degli Eventi \\ 
 $\Omega$ & Insieme universo & Spazio Campione \\ 
 $A$ & Insieme & Evento \\
 $A^C$ & Complementare di A & Negato di A \\
 $A \cup B$ & Unione di A e B & OR degli eventi, deve verificarli almeno uno tra A e B \\
 $ A \cap B $ & Intersezione tra A e B & AND degli eventi, devono verificarsi entrambi \\
 $ \bigcup_{k=1}^n A_k $ & Unione finita & n verifica almeno una tra $A_1, A_2,...,A_n$ \\
 $ \bigcup_{k=1}^{\infty} A_k $ & Unione numerabile & boh \\
 $ \bigcap_{k=1}^n A_k$ & Intersezione finita & Si verifica se tutti gli eventi $A_1,...,A_n$ si verificano \\
 $ \bigcap_{k=1}^{\infty} A_k$ & Unione numerabile & boh  \\
 $ \emptyset $ & Insieme Vuoto & Evento Impossibile \\
 $ A \cap B = \emptyset $ & A e B sono disgiunti & Eventi Incompatibili \\
 $ A \subset B $ & A contenuto in B & Il verificare di A implica il verificare di B \\
 $ \uplus_k A_k = \Omega $ & Ricomprimento disgiunto (partizione) & $A_1, A_2,...,A_n$ eventi neccessari\\
 \hline
\end{tabular}

\end{center}


\subsection{Esempio Lancio Moneta 1}
Poniamo caso che vogliamo descrivere l'evento che al terzo lancio di una moneta esca Testa, per prima cosa scegliamo un spazio campione:
$$ \Omega = {\{T,C\}}^{N} $$
Una moneta ha solo due casi, testa oppure croce, ora descriviamo che testa esca al terzo lancio:
$$ T_3 = \{ (w_1,w_2,...) \in \Omega: w_3 = "T" \} $$ 
abbiamo descritto questo eveno tramite propietà degli insiemi, nel caso volessimo esprimere lo stesso concetto ma per le croci ci basta fare il comlemento:
$$ T_3^C = C_3 = \{ (w_1,w_2,...) \in \Omega: w_3 = "C" \}  $$

\subsection{Esempio Lancio Moneta 2}
Poniamo invecere di voler complicare le cose, voglia esprimere l'evento che escano due Testa prima di due Croci, chiamiamo questo evento $A$, questo evento ha infinite possibilità, facciamo alcuni esempi:
$$ A_2 = T_1, T_2, \Omega $$
$$ A_3 = C_1, T_2, T_3, \Omega $$
$$ A_4 = T_1, C_2, T_3, T_4, \Omega $$
$$ A_5 = C_1, T_2, C_3, T_4, T_5 \Omega $$
Possiamo fare alcune osservazioni, $A_2, A_3$ sono incompatibili, non possono verificarci contemporaneamente, invece $A_5$ è incompatibile con $A_2,A_3,A_4$ .
Possiamo esprimere il verificarsi dell'evento $A$ in vari modi:\\
$$ A = A_2 \cup A_3 $$
$$ A = A_2 \cup A_3 \cup A_4 $$
$$ A = A_2 \uplus A_3  \uplus A_4 $$
$$ A = A_2 \uplus A_3 \uplus A_4 \uplus A_5 $$
Possiamo esprimere questo evento $A$ tramite \textbf{Unione Numberabile}:
\begin{equation*}
A_n =
\begin{cases}
C_1,T_2,...,C_{n-2},T_{n-1},T_n  \:\:\:\:\:\text{n dispari inizia con una croce}\\
T_1,C_2,...,C_{n-2},T_{n-1},T_n  \:\:\:\:\:\text{n pari inizia con una testa}
\end{cases}
\Rightarrow A = \bigcup_{n=2}^{\infty} \:\:\:\:\: \text{è un evento}
\end{equation*}

\subsection{Famiglia/Classi}
\blindtext

\subsection{Algebra e Sigma Algebra}
Preso un $\Omega$ spazio campione e un $a$ (a tondo), classe non vuota di sottinsiemi di $\Omega$ allora:
$$ a \: \text{è un algebra} \Leftrightarrow $$
$$ i) A \in a \Rightarrow A^C \in a \:\:\: \text{(a è chiusa rispetto il complemento)} $$
$$ ii) A_1, A_2 \in a \Rightarrow A_1 \cup A_2 \in a \:\:\: \text{(a è chiusa rispetto l'unione di due elementi)} $$
C'è un anche una sua variante chiamanta Sigma(numerabile) Algebra che si definisci così:
$$ \sigma-algebra \Leftrightarrow $$
$$ i)uguale $$
$$ ii) n \in N, A_n \in a \Rightarrow \bigcup_{n=1}^{\infty} A_n \in a$$
\subsubsection{Osservazioni}
Posto $a=\{\{2,3\}, \{6\}, \{4,5\}\}$, osserviamo i seguenti esempi:
\begin{center}
$ \{4,5\} \subseteq a $ SBAGLIATO\\
$ \{4,5\} \in a $ CORRETTO\\
$ \{\{4,5\}\} \subseteq a $ CORRETTO
\end{center}

\subsubsection{Casi Particolari}
Poniamo $A \subseteq \Omega$, si definisce \textbf{algebra(sigma) banale}, $a$ posto come:
$$ a = \{\emptyset, \Omega \}$$
È l'unica algebra a due elementi, ovviamente entrambe le propietà sono banalmente dimostrate poiché:
$$ \Omega^C = \emptyset $$
$$ \Omega \cup \emptyset = \Omega \in a $$
Gli elementi $\emptyset$ e $\Omega$ sono neccessari per essere un \textbf{algebra}.
Poniamo caso di un $a=\{A, A^c\}$ questa non è un algebra poiché $A \cup A^c = \Omega \not \in a$, se aggiunssimo solo $\Omega$ non sarebbe rispettata la prima condizione poiché $ \Omega^c = \emptyset \not \in a$.\\
Ricapitolando:
$$ a=\{A, A^C\} \:\:\: \textbf{non è algebra} \:\:\:\:\: a=\{A,A^C,\emptyset,\Omega\} \:\:\:\:\:\textbf{è algebra (sigma)}$$
Per contrapposizione la (sigma) algebra più grande è $P(\Omega)$, tutte le altre algebra(sigma) sono sottoinsiemi di $P(\Omega)$

\subsection{Propietà (conseguenze)}
\begin{enumerate}
\item $a$ è una algebra (sigma) $\Rightarrow \: \emptyset,\Omega \in \: a$ (come abbiamo osservato prima)\\Tutti gli elementi dell'algebra banale devono essere presenti in ogni algebra(sigma).
\item L'unione finita di elementi di un algebra (sigma) appartiene comunque ad $a$\\ Per $ii)$ abbiamo visto come l'unione si applica per due elementi, ma essendo $\cup$ associativa nel caso di $n-elementi$ basta operarli a due a due e quindi portare questa propietà fino a n elementi.
\item $ Sigma\:algebra \Rightarrow Algebra \:\:MA\:\: Sigma\: algebra \not \Leftarrow Algebra $\\
Questo poiché un unione finita da 0 a $+\infty$ non appartiene a tutte le algebra, cose che invece accade con le sigma algebra.
\end{enumerate}








\section{Lezione 06 - 24-03-2023}

\subsection{Propietà Sottospazio Vettoriale}

\begin{itemize}
\item[$W \underline{<} V$ è stabile rispetto alla somma di $n$ oggetti]
Siano $\underline{w}_1, ... , \underline{w}_n \in W$ si ha $w_1 + w_2 \in W \Rightarrow$ 
\item[Famiglia di sottospazi vettoriali]
Sia $ \mathbb{L}$ una famiglia di sottospazi di $V$, l'intersezione dei sottospazi della famiglia $\mathbb{L}$ è un sottospazio e si indica: 
$$ \bigcap_{L \in \mathbb{L}} L $$
L'intersezione di una qualunque famiglia di sottospazi è un sottospazio.\\
Dimostriamolo: 
	\subitem Neutro: Il neutro è un elemento comune, quindi è sempre contenuto.
	\subitem Stab $+$: Siano $ \underline{v},\underline{w} \in \bigcap_{L \in \mathbb{L}} L \Rightarrow \forall L \in \mathbb{L} \Rightarrow \underline{v},\underline{w} \in L \Rightarrow \underline{v}+\underline{w} \in \bigcap_{L \in \mathbb{L}} L $
	\subitem Stab $\cdot$: Siano $ \underline{v} \in \bigcap_{L \in \mathbb{L}} L, h \in \mathbb{R} \Rightarrow \forall L \in \mathbb{L} \Rightarrow \underline{v},\underline{w} \in L \Rightarrow \underline{v}+\underline{w} \in \bigcap_{L \in \mathbb{L}} L $
\end{itemize}

\subsection{Sottospazio Generato}
Sia $ S \subseteq V $, indicheremo con $<S>$ il \textbf{sotto spazio generato da S}.\\
$$ <S> = \bigcap_{L \in \mathbb{L}_s} L$$
In altri termini: è il più piccolo sottospazio rispetto all'intersezione.

\subsubsection{Esempi:}
\begin{itemize}
\item[•] $ <H> = H $ SEMPRE!
\item[•] $ <\{\underline{0}\} = \{\underline{0}\} $
\item[•] $ <V> = V $ 
\item[•] $ <\emptyset> = {0} $ Singleton dell'elemento neutro, poiché appartiene ad ogni elemento.
\end{itemize}

$ S= H \cup K $ con $H,K \underline{<} V$
$$ <H \cup K> = H+K = \{\underline{h}+\underline{k} / \underline{h} \in H, \underline{k} \in K \} $$
Dim:
Come sempre per dimostrare l'uguaglianza dobbiamo dimostare la doppia inclusione:
$$ <H \cup K> \subseteq \textbf{al contrario } H+K = \{\underline{h}+\underline{k} / \underline{h} \in H, \underline{k} \in K \} $$
non ho capito\\

Dimostriamo che sia spazio vettoriale:
\begin{itemize}
\item[Neutro] $$ \underline{0} = \underline{0}^{\text{preso da H}} + \underline{0}^{\text{preso da K}} $$
\item[Stabile $+$] $$ (\underline{h} + \underline{k}) + (\underline{h}^' + \underline{k}^') \in H+K $$
$$ (h+h^') + (k+k^') $$
\item[Stabile $\cdot$] $$ \alpha(\underline{h}+\underline{k}) = \alpha\underline{h} + \alpha \underline{k}  $$
\end{itemize}

\subsubsection{Esempio}
$$ H=\{(0,k) / x \in \mathbb{R}\} \;\;\; K=\{(y,0) / y \in \mathbb{R} \} $$
$$ <H \cup K> = H+K = (0+y, x+0) = \mathbb{R}^2 $$

\subsection{Propietà Sottospazio Generato}
Posto $H,K \underline{<} V$, allora valgono le seguenti propietà:
\begin{itemize}
\item[•] $ H \quad K = H \cap K = \{ \underline{0} \} \;(\text{neutro}) $ Si dicono in somma diretta.
\item[•] $ H + K = V$ allora $H,K$ si dicono supplementari
\item[•] $ H \quad K = V$ allora si dicono complementari (in altri termini devono essere in somma diretta e supplementari).\footnote{È un concetto un po' strano, perché vuol dire somma normale (quindi caso 2), ma ricandoci che l'intersezione da il neutro (quindi caso 1)}
\end{itemize}

\subsubsection{Esempio}
Posti $\{ \underline{0} \}$ e $V$:
\begin{itemize}
\item[Somma diretta]: $ \{ \underline{0} \} \quad V = \{ \underline{0} \} \cap V = \{ \underline{0} \}$
\item[Supplementari]: $  \{ \underline{0} \} + V = V$
\item[Complementare]: Dato che è sia somma diretta che supplementare
\end{itemize}

\subsection{Dipendenza/Indipendenza Lineare}
Sia $V$ uno spazio e vettore e siano $\underline{v}_1, \underline{v}_2, ..., \underline{v} \in V$, sono detti \textbf{linearmente dipendenti} (o legati) $\Leftrightarrow$
$$ \exists \alpha_1, \alpha_2, ..., \alpha_n \neq (0,0,...,0) = \alpha_1\underline{v_1} + ... + \alpha_n+\underline{v}_n = \underline{0}$$
La loro combinazione lineare deve essere il vettore nullo.
Se tali scalari non esistono allora si dice che sono \textbf{linearmente indipendenti} (o liberi), l'unica soluzione valida è quella formata da tutti zero: $0\underline{v}_1+...+0\underline{v}_n = \underline{0}$.
$$ \textbf{Se non sono dipendenti} \Rightarrow \textbf{Sono indipendenti}$$

\subsubsection{Esempio}

\subsection{Propietà dipendenza lineare}
\begin{itemize}
\item[1)] $\underline{0}$ dipende smpre da qualunque sistema
$$ \underline{0} = 0\underline{v}_1+...+0\underline{v}_n $$
\item[2)] $ \forall i \underline{v}_i $ dipende da $ \underline{w}_1,...,\underline{w}_n$ \footnote{Una specie di transitività della dipendenza}
$$  $$
\end{itemize}








%\section{Lezione 07 - 29/03/2023}

\subsection{Sottospazi Equivalenti}
Siano $S_1,S_2 \le V $ si dicono \textbf{equivalente} se e solo se generano lo stesso sottospazio vettoriale quindi:
$$ \Leftrightarrow <S_1> = <S_2> $$
($<S_1>,<S_2>$ si dicono sistema di generatori)

\subsubsection{Esempi}
Presi $\underline{v}, \underline{v}, \underline{0}, \underline{w}$ equivale a $\underline{v}, \underline{w}$?\\
Dobbiamo andare a verificare che ogni elemento di $S_1$ si possa scrivere come combinazione lineare di $S_2$, quindi dobbiamo andare a verificare la doppia inclusione.\\
In questo possiamo notare come vale l'equivalenza poiché, possiamo levare la doppia ripetizione di $\underline{v}$ dal $S_1$, e $\underline{0}$ essendo il neutro deve essere necessariamene presente per essere sottospazio, quindi vale la doppia inclusione.\\

Ponendoci in $\mathbb{R}^3$ consideriamo il seguente sottospazio:
$$ <(1,2,1),(2,4,2),(0,0,1),(1,2,50)> $$
Possiamo notare come $(2,4,2)$ e $(1,2,50)$ sono combinazioni lineari, questo ci permette di eleminarli, quindi:
$$ <(1,2,1),(0,0,1)> \; \; \textbf{Base}$$
Questi due elementi sono indipendenti poiché l'unica combinazione possibile è $h=k=0$.\\

Consideriamo $\mathbb{R}^n$ come sempre possiamo considerare la matrice come una lunga riga.
$$
\begin{pmatrix}
1 & 2 & 1 \\
2 & 4 & 2 \\
0 & 0 & 1 \\
1 & 2 & 50\\
\end{pmatrix}
\rightarrow
\begin{pmatrix}
1 & 2 & 1 \\
0 & 0 & 0 \\
0 & 0 & 1 \\
0 & 0 & 49\\
\end{pmatrix}
\rightarrow
\begin{pmatrix}
1 & 2 & 1 \\
0 & 0 & 1 \\
0 & 0 & 49 \\
0 & 0 & 0\\
\end{pmatrix}
\rightarrow
\begin{pmatrix}
1 & 2 & 1 \\
0 & 0 & 1 \\
0 & 0 & 0 \\
0 & 0 & 0 \\
\end{pmatrix}
$$
Le trasformazioni di riga ($E_1,E_2,E_3,E_4$) mantendono i sottospazi.\\
Le righe non nulla di una matrice sono sempre sistemi indipendenti.\\
Poniamoci in $\mathbb{R}_2[x]$ e prendeniamo:
$$ <x^2+2x+1, 2x^2+4x+2, 1, x^2+2x+50> $$
Possiamo consideraro anche solo i termini senza le incognite:
$$ <(1,2,1),(2,4,2),(0,0,1),(1,2,50)> $$
Possiamo portarla in forma matriciale:
$$ 
\begin{pmatrix}
1 & 2 & 1 \\
2 & 4 & 2 \\
0 & 0 & 1 \\
1 & 2 & 50 \\
\end{pmatrix}
$$
E possiamo portarla a gradini:
$$
\begin{pmatrix}
1 & 2 & 1 \\
0 & 0 & 1 \\
0 & 0 & 0 \\
0 & 0 & 0 \\
\end{pmatrix}
$$
Quello che ci viene alla fine è:
$$ <x^2+2x+1, 1> $$

\subsection{Osservazioni sulla in/dipendenza}
\subsubsection{Dipendenza}
Sia $v_1,...,v_n \in V$ \footnote{Consideriamo $n \ge 2$ perché se $n = 1$ i casi di riducono unicamente a: $ \underline{v} \neq \underline{0} $ indipendente e $ \underline{v} = \underline{0} $ dipendente} sono linearmente dipendenti $ \Leftrightarrow \exists i: \underline{v}_i $ dipende dai rimanenti.
Dim:\\
Per ipotesi sappiamo: 
$$ \exists h_1,...,h_n \neq (0,0,...0): h_1\underline{v}_1 + ... + h_n\underline{v}_n = \underline{0} $$
Supponiamo $h_1 \neq 0$ allora:
$$ \underline{v_1} = h_1^{-1}(-h_2\underline{v}_2 + ... + (-h_n)\underline{v}_n $$
Quindi $\underline{v}_1$ è combinazione lineare di $\underline{v}_2,...,\underline{v}_n$ quindi dipende da questi vettori.\\
(vale anche il viceversa).

\subsubsection{Indipendenza}
Per scrivere l'indipendenza ci basta unicamente fare il negato della dipendenza:\\
$ v_1,...,v_n \in V $ sono indipendenti $ \Leftrightarrow \forall \underline{v}_i: \underline{v}_i$ non dipende da: $\underline{v}_1,...,\underline{v}_n $\\

\subsubsection{Propietà}
Sia $V$ spazio vettoriale, definiamo le seguenti propietà:
\begin{itemize}
\item[•] Se $\underline{v}_1,...,\underline{v}_n$ dipendono aggiungere $\underline{w}_1,...,\underline{w}_n$ fa rimanere la dipendenza.

\item[•] Se $\underline{v}_1,...,\underline{v}_n, \underline{w}_1,...,\underline{w}_n$ sono indipendenti allora andando a restringere a $\underline{v}_1,...,\underline{v}_n$ rimane indipendente.

\item[•] Se $\underline{v}_1,...,\underline{v}_n$ sono indipendenti allora c'è \textbf{l'unicità di scrittura}.\\
Dim:
$$ \textbf{da aggiungere} $$

\item[•] Presi $W_1,W_2 \le V$ e per ipotesi in somma diretta $W_1 \cap W_2 = \{\underline{0}\}$ e presi $\underline{0} \neq \underline{v} \in W_1$ e $\underline{0} \neq \underline{w} \in W_2$ allora $\underline{v}$ e $\underline{w}$ \textbf{sono indipendenti}.\\
Dim:
$$ \textbf{da aggiungere} $$

\item[•] Generalizziamo il caso precendente

\item[•] La somma direta implica l'unicità di scrittura

\end{itemize}







\include{Lezioni/lezione08alex}
%\subsection{Lezione 9 - 05/10/2023}

\subsubsection{Successore Iterativo - BST}

Per questo tipo di algoritmo, dobbiamo ragionare in modo diverso. In questo caso non iterativo non possiamo permetterci di omettere determinati controlli a posteriori. In particolare il controllo nel caso in cui il valore di cui vogliamo il successore non ha figli destri ed e una foglia. In questo caso particolare non abbiamo la possibilita di risalire a ritroso ricorsivamente ma dobbiamo tenere traccia ogni volta che il nodo scende a sinistra, segnandoci il puntatore di quest'ultimo.

\begin{lstlisting}[language=Java]
	SuccI(T,k)
	Tmp = T
	ret = NIL
	while Tmp != NIL andd TMP->key != k then
	if Tmp->key < k then
	Tmp = Tmp->dx
	else 
	ret = Tmp
	Tmp = Tmp->sx
	if Tmp != NIL && Tmp->dx != NIL then
	ret = Min(Tmp->dx)
	return ret
\end{lstlisting}


\subsubsection{Predecessore Ricorsivo - BST}
L'algoritmo del predecessore e simile al successore strutturalmente parlando ma invertendo segni e qualche operazione

\begin{lstlisting}[language=Java]
	PredR(T,k)
		ret = NIL
			if ret != NIL then
				if ret->key = k then
					ret = Max(ret->sx)
				else if ret->key < k then
					ret = PredR(ret->dx,k)
				else 
					ret = PredR(T->sx,k)
					if ret = NIL then
						ret = T
		return ret
\end{lstlisting}

\subsubsection{Insert Ricorsiva - BST}

L'algoritmo dell'inserimento in un albero in un albero binario di ricerca puo vantare del fatto che e piu facile trovare il nodo nel quale si puo aggiungere il valore che abbiamo in input alla funzione. Ci bastera semplicemente scorrere a destra o a sinistra il nostro puntatore per poi arrivare nel primo punto NIL favorevole e "returnare" a cascata i puntatori dei padri.

\begin{lstlisting}[language=Java]
	InsertR(T,k)
	ret = T
		if T = NIL then
			ret = new_node(k)
		else if T->key < k then
			T->dx = InsertR(T->dx,k)
		else if T->key > k then
			T->sx = InsertR(T->sx,k)
	return ret
\end{lstlisting}

\subsubsection{New Node - Generico}
La funzione new node va a creare un nuovo nodo dinamico all'interno dell'albero.

\begin{lstlisting}[language=Java]
	new_node(k)
		ret = alloca_nodo() //andiamo a restituire il puntatore a nuovo nodo allocato in memoria a ret
		ret->key = k
		ret->sx = NIL
		ret->dx = NIL
		return ret
\end{lstlisting}

\subsubsection{Insert Iterativa - BST}
Versione iterativa della insert prevede dei controlli in piu per quanto riguarda la ricerca e inserimento. In questo caso specifico abbiamo bisogno di un puntatore in piu che ci segue nello scorrimento, chiamato \textbf{P} e sta a indicare il Padre del nodo a cui stiamo scorrendo.

\begin{lstlisting}[language=Java]
	InsertI(T,k)
	ret = T
	P = NIL
	Tmp = T
	while Tmp != NIL && Tmp->key != k do
		P = Tmp
		if Tmp->key < k then
			Tmp = Tmp->dx
		else
			Tmp = Tmp->sx
	
	If Tmp = NIL then
		x = new_node(k)
		if P->key < k then
			P->dx = x
		else
			P->sx = x
	else 
		P-sx = x
	return ret
\end{lstlisting}

\subsubsection{DeleteR - BST}
La delete prevede la delete del nodo e la restituzione dell'albero con quel nodo mancante. A primo acchitto non sembra un'operazione cosi difficile ma dobbiamo come sempre andare a ragionare per casi.

\begin{itemize}
	\item Caso albero vuoto. In questo caso dobbiamo semplicemente restituire T, il puntatore (vuoto) alla radice dell'alberoe
	\item Caso albero non vuoto. In questo caso la radice del sottoalbero ha un sottoalbero destro e sinistro.
		\begin{itemize}
			\item Se T->key < k then Delete(T->dx,k)
			\item Se T->key > k then Delete(T->sx,k)
			\item Se T->key = k, dobbiamo distinguere dei casi
			\begin{itemize}
				\item Nel caso in cui il nodo non ha figli allora, si puo procedere all'eliminazione del nodo
				\item Il caso in cui il nodo o ha figlio destro o figlio sinistro e basta.
				\item Il caso in cui il nodo ha entrambi i figli collegati. In questo caso specifico deleghiamo la distruzione del nodo a un'altra funzione chiamata \textbf{Stacca(T->dx,k)}.
			\end{itemize}
		\end{itemize}
\end{itemize}

In particolare questa funzione sara formata in questo modo:

\begin{lstlisting}[language=Java]
	StaccaMin(T,P)
		ret = T
		If T != NIL then
			ret = StaccaMin(T->sx,P)
			if ret = NIL then
				if P != NIL then
					if P->sx = T then
						P->sx = T->dx
					else
						P->dx = T->dx
		return ret
\end{lstlisting}



%\section{Lezione 10 - 12/04/2023}

\subsection{Componenti}
Sia $V$ spazio vettoriale prendiamo un riferimento (base ordinata): $\mathbb{R} = (\underline{e}_1,...,\underline{e}_n)$, e prendiamo un vettore $\underline{v} \in V$ allora $\underline{v}$ si potrà scrivere come combinazione lineare: 
$$\underline{v} = h_1 \underline{e}_1+...+h_n\underline{e}_n$$
Chiamiamo i cofficienti $h$ come \textbf{componenti}, più nello specifico:
$$ (h_1,...,h_n) \; \text{n-pla delle componenti} \; \underline{v} \; \text{nel riferimento} \; \mathbb{R} $$
\subsubsection{Esempi}
TODO: INSERIRE ESEMPIO

\subsection{Cambiamento di Riferimento (Formula di Passaggio)}
Presi due riferimenti vogliamo portare un vettore scritto come combinazione linere delle componenti del "vecchio" riferimento al "nuovo" riferimento.\\
Consideriamo: 
$$ \mathbb{R}_1 = (\underline{e}_1,...,\underline{e}_n) \; \text{VECCHIO RIFERIMENTO} $$
$$ \mathbb{R}_2 = (\underline{f}_1,...,\underline{f}_n) \;  \text{NUOVO RIFERIMENTO}$$
Consideriamo un vettore $\underline{v}$ del vecchio riferimento:
$$ \underline{v} = k_1\underline{e}_1+...+k_n\underline{e}_n $$
Come abbiamo visto prima le componenti di questo vettore sono: $(k_1,..,k_n)$\\
Cominciamo ad esperimere $\underline{e}_1,..,\underline{e}_n$ come combinazione lineare del nuovo riferimento:
$$\underline{e}_1 = h_{1,1}\underline{f}_1+...+h_{1,n}\underline{f}_n $$
$$ ... $$
$$\underline{e}_n = h_{n,1}\underline{f}_1+...+h_{n,m}\underline{f}_n $$
Il primo pedice indica il vettore, il secondo lo scorrimento.\\
Andiamo a sostituire la nuova comb. lin. di $\underline{e}_1$ in $\underline{v}$:
$$ \underline{v} = k_1(h_{1,1}\underline{f}_1,+...+h_{1,n}\underline{f}_n)+..+k_n(h_{n,1}\underline{f}_1,+...+h_{n,m}\underline{f}_n) $$
Ora mettiamo in evidenza $\underline{f}$:
$$ \underline{v} = \underline{f}_1\equalto{(k_1 h_{1,1}+...+k_n h_{1,n})}{k_1^{\prime}}+...+\underline{f}_n\equalto{(k_1 h_{n,1}+...+k_n h_{n,m})}{k_n^{\prime}} $$
Quindi $(k_1^{\prime},...,k_n^{\prime})$ le componenti di $\underline{v}$ in $\mathbb{R}_2$
In definitiva avremmo:
$$ k_1^{\prime} = (k_1 h_{1,1}+...+k_n h_{1,n})  $$
$$ ... $$
$$ k_n^{\prime} = (k_1 h_{n,1}+...+k_n h_{n,m})  $$

\subsubsection{Matrice di Passaggio}
Un altro metodo oltre la "formula di passaggio" è la "matrice di passaggio"\\
Andiamo ad esprimere 

\subsubsection{Esempio}
Posti in $\mathbb{R}^3$ consideriamo i seguenti riferimenti:
$$ \mathbb{R}_1 = ((1,2,3),(4,5,6),(0,0,2) \; \text{VECCHIO RIFERIMENTO} $$
$$ \mathbb{R}_2 = ((0,1,0),(0,1,1),(1,1,1)) \;  \text{NUOVO RIFERIMENTO}$$
Usiamo il metodo della "matrice di passaggio", cominciamo col esprimere i vecchie riferimenti in favori dei nuovi:
$$ (1,2,3) = -1(0,1,0) + 2(0,1,1)+1(1,1,1) $$
$$ (4,5,6) = -1(0,1,0) + 2(0,1,1)+4(1,1,1) $$
$$ (0,0,2) = -2(0,1,0) + 2(0,1,1)+0(1,1,1) $$
Ora costruiamo la matrice (per colonna):
$$\begin{pmatrix}
-1 & -1 & -2 \\
2 & 2 & 2 \\
1 & 4 & 0 \\
\end{pmatrix}$$
Ora basta fare il prodotto righe per colonne di un qualsiasi vettore che vogliamo "trasportare":
$$ 
\begin{pmatrix}
-1 & -1 & -2 \\
2 & 2 & 2 \\
1 & 4 & 0 \\
\end{pmatrix}
\begin{pmatrix}
1 \\
0 \\
0 \\
\end{pmatrix}
=
\begin{pmatrix}
-1 \\
2 \\
1 \\
\end{pmatrix}
$$

\subsection{Ultima Osservazione Spazi Vettoriali}
Consideriamo: 
$$ \mathbb{R}^3: < (1,1,1),(0,1,5),(2,3,4),(2,2,2),(4,5,6) >$$
Vogliamo trovare una base (procediamo con la riduazione a gradini):
$$ 
\begin{pmatrix}
1 & 1 & 1 \\
0 & 1 & 5 \\
2 & 3 & 4 \\
2 & 2 & 2 \\
4 & 5 & 6 \\
\end{pmatrix}
\rightarrow
\begin{pmatrix}
1 & 1 & 1 \\
0 & 1 & 5 \\
0 & 1 & 2 \\
0 & 0 & 0 \\
0 & 1 & 2 \\
\end{pmatrix}
\rightarrow
\begin{pmatrix}
1 & 1 & 1 \\
0 & 1 & 5 \\
0 & 1 & 2 \\
0 & 0 & 0 \\
0 & 0 & 0 \\
\end{pmatrix}
\rightarrow
\begin{pmatrix}
1 & 1 & 1 \\
0 & 1 & 2 \\
0 & 1 & 5 \\
0 & 0 & 0 \\
0 & 0 & 0 \\
\end{pmatrix}
\rightarrow
\begin{pmatrix}
1 & 1 & 1 \\
0 & 1 & 2 \\
0 & 0 & 3 \\
0 & 0 & 0 \\
0 & 0 & 0 \\
\end{pmatrix}
$$
Quindi la base sarà:
$$ <(1,1,1),(0,1,2),(0,0,3) $$
\begin{itemize}
\item[•] Avendo $3$ pivot $\Rightarrow dim = 3$
\item[•] Le righe non nulle di una matrice a gradini \textbf{sono indipendenti}
\end{itemize}

\subsection{Determinanti (matrice quadrata)}
Il determinante esiste solo e solamente per matrici quadrate; il determinante si definisce per ricorsione\footnote{Attenzione da non confondere "dimostrazione per induzione" con "definizione per induzione"}.\\

\subsubsection{Matrice Complementare}
La matrice complemenare di $A(i,j)$ consiste nell'eliminare la $i$ riga e $j$ colonna.\\
Esempio $A(2,2)$:
\begin{center}
\stackMath
\stackinset{c}{}{c}{0\baselineskip}{\rule{4.4\baselineskip}{.4pt}}{%
\stackinset{c}{0\baselineskip}{c}{}{\rule{.4pt}{4.0\baselineskip}}{%
\begin{pmatrix}
2 & 3 & 4 \\
1 & 0 & 0 \\
5 & 0 & 1 \\
\end{pmatrix}}}
$\Rightarrow
\begin{pmatrix}
2 & 4 \\
5 & 1 
\end{pmatrix}$
\end{center}

\subsubsection{Complemento Algebrico}
Il complemento algebrico di elemento $a_{i,j}$ della matrice è definito:
$$ A_{i,j} = (-1)~{i+j} det(A(i,j)) $$
Si indica con un "A grande" : $A_{i,j}$\\
Esempio, consideriamo una matrice:
$$ 
\begin{pmatrix}
a_{1,1} & a_{1,2} & \dots & a_{1,n} \\
a_{2,1} & a_{2,2} & \dots & a_{2,n} \\
\vdots & \vdots & \ddots & \vdots \\
a_{n,1} & a_{n_2} & \dots & a_{n,n} \\
\end{pmatrix}
$$
Allora possiamo ricavare il determinante:
$$ a_{1,1}A_{1,1}+a_{1,2}A_{1,2}+...+a_{1,n+1}A_{1,n+1} = det A $$
$$ a_{2,1}A_{2,1}+a_{2,2}A_{2,2}+...+a_{2,n+1}A_{2,n+1} = det A $$
$$ ... $$
$$ a_{n,1}A_{n,1}+a_{n,2}A_{n,2}+...+a_{n,n+1}A_{n,n+1} = det A $$
\begin{center}
\textbf{TUTTI QUESTI VALORI SONO UGUALI}
\end{center}
Questo procedimento è stato effettuato sulle righe, ma si può applicare uguale alle colonne.

\subsection{Determinante per Casi}

\subsubsection{Matrice 2x2}
Per le matrici $2x2$ il determinante si ottiene come diagonale primare - diagonale secondaria\\
Per le matrici $2x2$ il determinante si ottiene come diagonale primare - diagonale secondaria\\
\begin{equation}
\begin{vmatrix}
a & b \\
c & d
\end{vmatrix}
= a*d - b+c
\end{equation}

\subsubsection{Matrice 3x3 - Regola di Sorrus}
Avendo una matrice $3x3$ "duplichiamo" la prima e la seconda colonna  e li posiniamo infondo
$$
\begin{vmatrix}
a_{11} & a_{12} & a_{13}\\
a_{21} & a_{22} & a_{23}\\
a_{31} & a_{32} & a_{33}
\end{vmatrix}
\rightarrow
\begin{pmatrix}
a_{11} & a_{12} & a_{13} & a_{11} & a_{12}\\
a_{21} & a_{22} & a_{23} & a_{21} & a_{22}\\
a_{31} & a_{32} & a_{33} & a_{31} & a_{32}
\end{pmatrix}
$$
Da cui otteniamo:
$$ a_{11}a_{22}a_{33} + a_{12}a_{23}a_{31} + a_{13}a_{21}a_{32} - (a_{13}a_{22}a_{31} + a_{11}a_{23}a_{23} + a_{12}a_{21}a_{33})$$
Esempio:
$$ 
\begin{vmatrix}
1 & 1 & 1 \\
0 & 1 & 1 \\
2 & 2 & 2
\end{vmatrix}
= 2+2+0-(2+0+2) = 0
$$

\subsection{Propietà sul Determinante}
Dimostrazioni(dove neccessarie) ed esempi omessi (per ora)
\begin{itemize}
\item[•] Se una riga dipende dalle rimanenti allora il determinante è zero
\item[•] Se una matrice ha due righe uguale allora il determinante è zero
\item[•] Se il determinante è diverso da zero allora righe indipendenti
\item[•] Se scambiamo due righe il determinante cambia di segno
\item[•] Teorema di Cauchy-Binet: $det(AB) = det(A) * det(B)$
\item[•] $det A$ = $det A^t$
\end{itemize}


%\section{Lezione 11 - 14/04/2023}

\subsection{Operazioni di Riga riguarda il Determinante}
Le operazioni di riga preservano la non nullità del determinante ma non il valore.
\begin{itemize}
\item[$E_1$]: Scambiare due rige fa cambiare il segno
\item[$E_2$]: Bisogna moltiplicare per uno scalare anche il determinante (da rivdere)
\end{itemize}

\subsection{Determinante di una Matrice a Gradini}
Il modo più semplice per riusciure a calcolare il determinante di una matriace è portarla a gradini, poiché il determinante è \textbf{il prodotto della diagonale principale}.\\
$$ 
\begin{pmatrix}
1 & 0 & 0 \\
0 & 2 & 2 \\
0 & 0 & 3
\end{pmatrix} 
\Rightarrow det = 1*2*3 = 6 \Rightarrow Indipendente
$$

\subsection{Invertilità di una Matrice}
Presa una matrice $A \in \mathbb{R}_{n,n}$ esiste una matrice $B \in R_{n,n}$ tale che:
$$ AB = I_n = BA $$
Vale solo se il determinante è diverdo da zero.
$$ |A| \neq 0 \Leftrightarrow \exists inversa B=A^{-1} $$
$A^{-1}$ è unica.
Dim unicità:
$$ B_1=B_1I_N=B_1(AB_2)=(B_1A)B_2=I_nB_2=B_2 $$
Dim $\Rightarrow$:
$$ AB = I_n \Rightarrow det(AB) = det(I_n) = 1 $$
Dim $\Leftarrow$:
TODO:FINIRE

\subsection{Minore di una Matrice (sottomatrice quadrata)}
Indichiamo il minore come:
$$ A_{(i_1,...,i_h; j_1,...,j_n)} $$
\begin{itemize}
\item[]$i_1,...,i_h$ indica le righe
\item[]$j_1,...,j_n$ indica le colonne
\end{itemize}
Consideriamo la seguente matrice:
$$
\begin{pmatrix}
2 & 3 & 0 \\
0 & 1 & 2 \\
1 & 0 & 0
\end{pmatrix}
$$
$$ A_{(2,3;2,3)} = \begin{pmatrix}
1 & 2\\
0 & 0
\end{pmatrix} $$
I minori valgono anche sulle matrice rettangolari ma i minori rimangono sottomatrici quadrate.\\

\subsection{Grado Massimo}
Un minore si dice di \textbf{grado massimo} se il suo grado coincide con $min\{n,m\}$.\\
Una matrice non quadrata ha sempre più di un minore di ordine massimo, mentre una matrice quadrata ha un solo minore di ordine massimo ed è la matrice stessa.
\subsubsection{Esempio}
$$
\begin{pmatrix}
1 & 1 & 1 & 1 \\
2 & 2 & 2 & 2 \\
0 & 0 & 1 & 1 
\end{pmatrix}
$$
Consideriamo:
$$ A_{(2,3;2,4)} = \begin{pmatrix}
2 & 2 \\ 0 & 1
\end{pmatrix}$$
Questo non di grado massimo poichè il grado massimo è $min\{3,4\} = 3$

\subsection{Orlato}
Se abbiamo un minore che non è di grado massimo, possiamo orlarlo aggiungendo una riga e una colonna.\\
Un orlato rimane un minore, e si più orlare un orlato.\\
Riprendiamo l'esempio di sopra, eravamo rimasti che $A_{(2,3;2,4)}$ non fosse di grado massimo, andiamo ad orlarlo:
$$ A_{(1,2,3;2,3,4)} = \begin{pmatrix}
1 & 1 & 1 \\
2 & 2 & 2 \\
0 & 1 & 1 \\
\end{pmatrix} $$
Abbiamo raggiunto una matrice di grado massimo orlando cioè abbiamo aggiunto la $1$ riga e la $3$ colonna.

\subsection{Minore Fondamentale}
Si definisce \textbf{minore fondamentale} un minore che rispetta queste propietà:
\begin{itemize}
\item[1)] $det \neq 0$
\item[2)] Tutti i suoi orlati hanno $det = 0$  
\end{itemize}
\textbf{ESISTONO SEMPRE I MINORI FONDAMENTALI}\\
Non ci sono sempre minori fondamentali, ad esempio la matrice nulla non ha minori fondamentali, ma, se la matrice non è nulla allora esiste sempre almeno un minore fondamentale. Il minore fondamentale non è necessariamente unico.\\
Se un minore di ordine massimo ha determinante diverso da zero allora esso è un minore fondamentale

\subsubsection{Algoritmo per trovare Minore Fondamentale}
Un modo semplice per trovare un \textbf{minore fonmentale} è cominciare sempre da un minore molto piccolo e poi andare ad orlarlo fino a raggiugere il grado massimo.\\
Cominciamo con prendere una matrice:
$$ \begin{pmatrix}
1 & 1 & $\hlcyan{1}$ & 0 \\
2 & 3 & 4 & 0 \\
1 & 1 & 1 & 0 
\end{pmatrix} $$
Andiamo a prendere un minore "piccolo" con determinante diverso da zero:
$$ A_{(1;3)} = (1) \; \text{con det} \neq 0 $$
Abbiamo escludo gli "zeri" poiché il loro determinante è zero.\\
Procediamo con nostro "algoritmo" andando ad orlarlo:
$$ A_{(1;3)} \rightarrow A_{(1,2;3,4)} = \begin{pmatrix} 1 & 0 \\ 4 & 0 \end{pmatrix} det = 0 $$
Avendo trovato $det = 0 = (1*0)-(0*4)$ dobbiamo scegliere un altro orlato:
$$ A_{(1;3)} \rightarrow A_{(1,2;2,3)} = \begin{pmatrix} 1 & 1 \\ 3 & 4 \end{pmatrix} det = 1 $$
Abbiamo raggiunto un buon candidato ora dobbiamo vericare che tutti i suoi orlati abbiano $det = 0$, in questo caso il suo unico orlato è:
$$ A_{(1,2;3,4)} \rightarrow A_{(1,2,3;1,2,3)} = \begin{pmatrix}
 1 & 1 & 1 \\
 1 & 3 & 4 \\
 1 & 1 & 1
\end{pmatrix} \; \text{ unico orlato con }  det=0$$
Quindi in definitiva:
$$ A_{(1,2;2,3)} = \begin{pmatrix} 1 & 1 \\ 3 & 4 \end{pmatrix} $$ 
$$ \textbf{MINORE FONDAMENTALE}$$

\subsection{Teorema degli Orlati (NO DIM)}
Sia $A \in \mathbb{R}_{n,m} $ matrice rettangolare e $A(i_1,...,i_h; j_1,...,j_n)$ minore fondamentale allora $\Rightarrow$:
$$ \underline{a}_{1,1},...,\underline{a}_{i,n}$$
Sono una base dello spazio vettoriale generato dalle \textbf{righe}.\\
(Esiste equivalente per colonne)
Conseguanza di ciò:
$$ \textbf{dim Righe Generate = dim Colonne Generate} $$

\subsubsection{Rango}
$$ \textbf{Rango = dim(Minore Fondamentale)} $$
Da questo sappiamo che i pivot di una matrice a gradini corrisponde alla dimensione, quindi al \textbf{rango}.

\subsubsection{Corollari derivati}
\begin{itemize}
\item[•] Il rango di riga è sempre uguale al rango di colonna, inoltre il rango di $A$ è uguale al numero di pivot di una matrice a gradini equivalente per righe.
\item[•] Tutti i minori fondamentali hanno lo stesso grado
\item[•] Il determinante di una matrice quadrata è diversa da zero se e solo se le righe (o colonne) sono indipedenti
\item[•] $det A = 0 \Leftrightarrow \; \text{righe (o colonne) dipendenti}$
\end{itemize}

\subsection{Criteri di compatibilità sistemi di equazioni lineare}
Consideriamo un sistema lineare:
$$
\syssubstitute{A{a_{11}}B{a_{21}}C{a_{m1}}D{a_{1n}}E{a_{2n}}F{a_{mn}}X{x_{n}}}
\systeme{
  A x_1 +...+ D X  = c_1,
  B x_1 +...+ E X = c_2,
  C x_1 +...+ F X = c_n
}
$$
Possiamo scriverlo in forma matrice nel seguenti modo $AX=C$:
$$ A = \begin{pmatrix} a_{11} & \dots & a_{1n} \\ \vdots & \ddots & \vdots \\ a_{n1} & \dots & a_{n1} \end{pmatrix} \; X = \begin{pmatrix} x_1 \\ \vdots \\ x_n \end{pmatrix} \; C =  \begin{pmatrix} c_1 \\ \vdots \\ c_n \end{pmatrix} $$
Possiamo esprimere per $C$ e spezzarla:
$$ 
C = x_1 \begin{pmatrix} a_{11} \\ \vdots \\ a_{n1} \end{pmatrix} +...+ x_n \begin{pmatrix} a_{1m} \\ \vdots \\ a_{nm} \end{pmatrix}
$$
Quindi se c’è una soluzione la colonna dei termini noti quindi se c’è una soluzione la colonna dei termini noti.\\
\subsubsection{Primo Criterio di Compatibilità}
Un sistema S  è compatibile $\Leftrightarrow$ la colonna dei termini noti è combinazione lineare della matrice incompleta (è soluzione).







%\section{Lezione 12 - 19/04/2023}

\subsection{Teorema di Rouché-Capelli}
Il teorema di Rouché-Capelli anche noto come secondo sistema di compatibilità afferma che un sistema $S$ è compatibile $\Leftrightarrow$ il rango della matrice incompleta è uguale al rango della matrice completa.
$$
\syssubstitute{A{a_{11}}B{a_{21}}C{a_{m1}}D{a_{1n}}E{a_{2n}}F{a_{mn}}X{x_{n}}}
\systeme{
  A x_1 +...+ D X  = c_1,
  B x_1 +...+ E X = c_2,
  C x_1 +...+ F X = c_n
}
\; \text{è compatibile} \Leftrightarrow
v(A) = v(A^{\prime})
$$

\textbf{DIM $\Rightarrow$:}
Per il primo principio compatibilità:
$$ 
\begin{pmatrix}
c_1 \\ \vdots \\ c_n
\end{pmatrix}
=
y_1 \begin{pmatrix}
a_{11} \\ \vdots \\ a_{n1}
\end{pmatrix}
+...+
y_n \begin{pmatrix}
a_{1m} \\ \vdots \\ a_{nm}
\end{pmatrix}
$$
$$ (c_1,...,c_n) \; \text{DIPENDE DALLE COLONNE}$$
DA FINIRE
\textbf{DIM $\Leftarrow$:}
Partiamo da "i due ranghi sono uguali" allora per il teorema degli orlati hanno lo stesso ordine/grado $(L)$.\\
Consideriamo una matrice:
$$
\begin{pmatrix}
a_{11} & \dots & a_{1n} &\aug& c_1 \\
\vdots & \ddots & \vdots &\aug& \vdots \\
a_{n1} & \dots & a_{nm} &\aug& c_n 
\end{pmatrix}
$$
Consideriamo un minore fondamentale della matrice incompleta (base), ma dalla ipotesi lo è anche per la matrice completa.\\





%\section{Lezione 13 - 13/10/2023}

\subsection{Alberi Red - Black}
Gli alberi Red - Black, sono alberi binari di ricerca che associano dei colori ai loro nodi. 
%Come ogni tipo di struttura dati ha le sue caratteristiche. 
La colorazione ovviamente andrà a braccetto con alcune proprieta (vincoli) che di seguito andremo a definire.

\begin{itemize}	
	\item[1)] Ogni nodo deve essere rosso o nero.
	\item[2)] \textbf{I nodi foglie possono essere solo neri (NIL)}, quindi i nodi rossi potranno essere soltanto all'interno. 
	\item[3)] Ogni nodo rosso ha \textbf{solo} figli neri.
	\item[4)] Per ogni nodo X preso all'interno dell'albero, ogni percorso da X al nodo foglia contiene \textbf{lo stesso numero} di nodi neri.
\end{itemize}

\begin{figure}[H]
	\includegraphics[width=\textwidth]{AlberoRB} 
	\caption{Questo è un albero RB perché soddisfa tutti e 4 i vincoli}
\end{figure}

\paragraph{Non tutti gli alberi possono essere colorabili}
\mbox{}
\smallskip

Per vedere se un albero è colorabile ci sono delle considerazioni da fare:

\begin{itemize}
    \item Colorare subito le foglie e la radice di nero.
    \item Osservare se esiste un sottoalbero è visibilmente piu pesante di un altro. In tal caso l'albero è quasi sicuramente non colorabile. Teoricamente se la differenza di altezza di alberi è maggiore di due allora probabilmente non è colorabile.
\end{itemize}

\subsubsection{Altezza Nera di un albero R-B}
L'altezza nera, di un albero R-B, è il numero di nodi neri che, preso un nodo X, si contano da X fino alle foglie escludendo X.

L'altezza nera è sicuramente minore dell'altezza dell'albero e al massimo uguale.

Dimostriamo dunque che l'altezza è sicuramente:

$$h \le 2^{bh(x)}-1$$

Preso un nodo all'interno di un albero il numero di nodi interni non può essere più piccolo di un albero completamente nero.
$$ n i(x) \ge 2^{bh(x)} -1 $$

Dimostriamo per induzione

\begin{itemize}
	\item Base Induttiva: Albero di altezza zero, quindi il numero di nodi interni di un albero di $h=0$ è \textbf{zero}, andiamo a svolgere l'equazione con la base induttiva:
	$$ 0 \ge 2^{bh(x)}-1 \Rightarrow 2^0-1 = 0 \Rightarrow \text{ VERO }$$
	
	
	\item Caso Induttivo $h > 0$: L'albero contiene almeno un nodo interno. 
	Andiamo a scomporre il nostro albero come sottoalbero sx del figlio sinistro e sottoalbero dx del figlio destro.
	$$ ni(y) \ge 2^{bh(y)}-1 $$
	$$ ni(z) \ge 2^{bh(z)}-1 $$
	Noi sappiamo che 
	$$ ni(x)=1+ni(y)+ni(z) $$
	
	In questo caso, ragionando analiticamente possiamo dire che l'altezza nera di X, il nostro nodo padre del sottoalbero dipende dal fatto che y, il suo sottoalbero sinistro, sia nero o rosso.
	\begin{itemize}
		\item Nel caso in cui il nodo sia rosso, allora l'altezza nera di x e y sono uguali.
		\item Nel caso in cui il nodo sia nero, allora l'altezza nera di x e uguale a quella di y + 1.
	\end{itemize}
	
	Esplicitiamo $bh(y)$ dalle due equazioni perche ci interessa esplicitare tutto per $bh(x)$

	In questo caso vedremo che $bh(y) > bh(y) - 1$ poiche o e uguale, o e sicuramente maggiore di $bh(x) - 1$.
	
	Questo vale anche per z, dunque avremo:
	
	$$bh(y) \ge bh(x) -1$$
	$$bh(z) \ge bh(x) -1$$

	Questo vuol dire che scendendo di altezza, andro a diminuire al massimo di uno l'altezza del sottoalbero. Grazie a queste equazioni possiamo ritornare a ritroso alla tesi.
	

	Usiamo questo ragionamento matematico: Se io so che $n \ge m$, allora avro anche che $ 2^{n} \ge 2^{m}$ poiche l'esponenziale e crescente e non andiamo a modificare il risultato comunque finale. 
	
	Applichiamo dunque la stessa proprieta alle stesse equazioni scritte sopra. In tal caso avremo 
	
	$$bh(y) \ge bh(x) -1 \rightarrow 2^{bh(y)} \ge 2^{bh(x)-1} $$
	sottraiamo una stessa quantita a entrambi i membri
	$$2^{bh(y)}-1  \ge 2^{bh(x)-1}$$
	Dunque entrambi vedremo che la somma tra $2^{bh(y)} e 2^{bh(z)}$ sono maggiori o uguali di $2^{bh(x)-1}$.
	
	Il numero di nodi interni di X e dato da $1 + ni(y)+ ni(z)$. Sostituendo abbiamo che $1 + 2^{bh(x) - 1} -1 + 2^{bh(x)-1}-1 \rightarrow 2*2^{bh(x)-1}$. Il "-1" puo essere semplificato portando dentro il 2 moltiplicato avanti all'espressione. In tal modo avremmo che indipendentemente dall'altezza che io ho in entrata, il numero di nodi interni di quel sottoalbero e almeno uguale a $2^{bh(x)}-1$, quindi la tesi inziale e dimostrata.
	
	
\end{itemize}

\subsubsection{Considerazioni sull'altezza nera di un albero}

Se sappiamo che il numero di nodi n e maggiore o uguale a $2^bh -1$, possiamo intuitivamente e approssimativamente andare a trovare l'altezza nera dell'albero.

Nel caso in cui avesse tutti i nodi neri allora l'altezza nera e $\le h$, mentre se ha alternati rossi e neri, abbiamo il limite minimo dell'altezza nera, cioe $\frac{h}{2}$.

Dunque l'altezza nera e compresa tra : $\frac{h}{2} \le bh \le h$.

Usando la matematica e le nozioni della dimostrazione precedente...
$$bh \ge \frac{h}{2} \rightarrow 2^{bh} -1 \ge 2^{\frac{h}{2}-1}$$
Se sappiamo che n e $\ge 2^{bh}-1$ allora,
$$n \ge 2^{\frac{h}{2}-1}$$
$$n+1 \ge 2^{\frac{h}{2}}$$
$$\log_{2} h+1 \ge \frac{h}{2}$$
$$h \ge 2\log_{2} n+1$$

\subsubsection{Inserimento Albero R-B}
Gli algoritmi di inserimento e bilanciamento usati fino ad ora non andranno più bene per questo tipo di struttura. Nonostante abbiamo più libertà da un certo punto di vista, dobbiamo considerare che la proprieta 4 degli alberi Red-Black ci impedisce di fare degli inserimenti nella struttura dati in modo efficiente.
\smallskip

In particolare nell'inserimento di un valore in un nodo NIL, l'algoritmo deve occuparsi di creare il nodo, colorarlo e di creare e colorare a sua volta i figli NIL (di nero ovviamente).
\smallskip

Successivamente la colorazione del nodo k non è immediata e semplice, va considerato il colore del padre e non va rotto il vincolo dello stesso numero di nodi neri per ogni percorso dei sottoalberi.
\smallskip

Abbiamo due possibilità di colore all'inserimento. Solitamente per non creare problemi con il padre del sottoalbero a cui dobbiamo inserire, si inserisce nero. Anche in quel caso non e detto che l'inserimento del nero non abbia creato problemi per la proprieta 4 dei R-B.

Nei Red - Black la violazione di una di questi due algoritmi puo essere scoperta solo ricorsivamente.

\begin{lstlisting}[language=Java]
	InsertRB(T,k)
		if T != NIL then
			if T->key < k then
				T->dx = InsertRB(T->dx, k)
				T = BilanciaRBdx(T)
			else if T->key > k then
				T->sx = InsertRB(T->sx, k)
				T = BilanciaRBsx(T)
			else
				T = new_nodeRB(k,r) //Creazione nodo rosso (r) e figli a NIL
		return T
\end{lstlisting}

Questa funzione verra supportata dalle funzioni di bilanciamento che sono specificatamente scritte apposta per la R-B. In tal caso abbiamo 3 casistiche generali di problemi.
Nel caso in cui il nodo che andiamoa  inserire sia rosso...

\begin{itemize}
	\item Caso 1) Il padre rosso e il fratello rosso. Sia a destra che a sinistra del padre rosso.
	\item Caso 2) Inserimento a destra del sottoalbero il cui padre e rosso e il fratello nero.
	\item Caso 3) Inserimento a sinistra del sottoalbero il cui padre e rosso e il fratello e nero.
\end{itemize}

La risoluzione del Caso 1, si scambia il nodo padre rosso con il nonno nero, in tal caso abbiamo che i figli diventeranno perforza di cose neri. In questo modo non andiamo a violare la proprieta 4, poiche i percorsi sia a destra che a sinistra avranno lo stesso numero di nodi neri, ma andiamo soltanto ad aumentare l'altezza nera.

Il Caso 2 si risolve ruotando in modo tale da arrivare al caso 3.

Per il caso 3 ci conviene ruotare l'albero a destra e sostituire il nodo radice (precendetemente nero) con un nodo rosso (visto che era figlio di nero). In tal modo abbiamo la radice nera, i figli rossi e i sottoalberi non cambiano.

Aggiustare assolutamente
%\section{Lezione 14 - 26/04/2023}

\subsection{Propietà}
Consideriamo una funzione lineare $f: V \rightarrow W$ elenchiamo le seguenti propietà:

\begin{itemize}


\item[1)] $f(\underline{0}_v) = \underline{0}_w$


\item[2)] $f(h_1\underline{v}_1+...+h_n\underline{v}_n) = h_1f(\underline{v}_1)+...+h_nf(\underline{v}_n)$


\item[3)] $\underline{v} \; \text{dipende da} \; \underline{v}_1,...,\underline{v}_n \Rightarrow f(\underline{v}) \; \text{dipende da} \; f(\underline{v}_1),...,f(\underline{v}_n)$\\
$\textbf{DIM:}$
$$ \underline{v} = h_1\underline{v}_1+...+h_n\underline{v}_n $$
$$ f(\underline{v}) = f(h_1\underline{v}_1+...+h_n\underline{v}_n) =^{2)} h_1f(\underline{v}_1)+...+h_nf(\underline{v}_n)$$

\item[3.1)] $f$ conserva dipendenza lineare MA NON L'INDIPENDENZA
$$ \underline{v}_1,...,\underline{v}_n \;DIP\; \Rightarrow f(\underline{v}_1),...,f(\underline{v}_n) \;DIP\; $$
$$ \exists \underline{v}_i \;DIP\; \underline{v}_1,...,\underline{v}_{i+1},...,\underline{v}_n \Rightarrow f(\underline{v}_i) \;DIP DA\; f(\underline{v}_1),...,f(\underline{v}_{i+1}),...,f(\underline{v}_n)$$

\item[4)] $f(<S>) = <f(S)> = <f(\underline{v}_1),...,f(\underline{v}_n))>$\\

Andiamo a dimostare la prima uguaglianza nel solito modo:
\subitem DIM $\subseteq$:
$$ \underline{v} \in f(<S>) $$
$$ \exists \underline{w} \in <S> \underline{v} = f(\underline{w})$$
$$ \underline{w} = h_1\underline{v}_1+...+h_n\underline{v}_n$$
$$ f(\underline{w}) = h_1f(\underline{v}_1)+...+h_nf(\underline{v}_n) \in <f(S)>$$
 
\subitem DIM $\supseteq$:
$$ \underline{w} \in <f(S)> $$
$$ \underline{w} = h_1f(\underline{v}_1)+...+h_nf(\underline{v}_n)$$
$$ f(h_1\underline{v}_1+...+h_n\underline{v}_n) \in f(<S>)$$
$$ \underline{w} \in f(<S>) $$

\item[5)] $H \le V \rightarrow f(H) \le W$\\
Dimostriamo sia sottospazio vettoriale:

\subitem Non vuoto:
$$ \underline{0} \Rightarrow f(\underline{0})= \underline{0} \in H $$
\subitem Stabilità Somma:
$$ \underline{w},\underline{w}^{\prime} \in f(H) \;\;\;\;\; \underline{v},\underline{v}^{\prime} \in H $$
$$ \underline{w} = f(\underline{v}) \;\;\;\;\; \underline{w}^{\prime} = f(\underline{v}^{\prime}) $$
$$ \underline{w}+\underline{w}^{\prime} = f(\underline{v})+f(\underline{v}^{\prime})= f(\underline{v}+\underline{v}^{\prime}) \in f(H) $$

\subitem Stabilità Prodotto:
TODO: DA FARE COME ESERCIZIO (un giorno lo farò)

\end{itemize}

\subsection{Kernel}

Consideriamo la funzione lineare $f: V \rightarrow W$, denotiamo con $Im f$ il sottospazio immagine di $f$ cioé: $\{f(\underline{v})/\underline{v} \in V\}=f(V) = Im f$.\\

Andiamo a definire un altro insieme chiamato \textbf{Kernel} o anche detto \textbf{ker}, cioé l'insieme di tutti i valori del dominio che vanno a finire nel neutro nella fattispecie: $ ker f = {\underline{v} \in V / f(\underline{v}) = \underline{0}} $\\

Andiamo a dimostare che il ker sia un sottospazio:
\begin{itemize}
\item[Non vuoto:] Vero poiché il neutro gli appartiene

\item[Stabilità Somma:]
$$ \underline{v},\underline{v}^{\prime} \in ker f \Rightarrow \underline{v}+\underline{v}^{\prime} \in ker f $$

$$ f(\underline{v})+f(\underline{v}^{\prime}) = f(\underline{v}+ \underline{v}^{\prime}) = \underline{0}+\underline{0} = \underline{0} $$

\item[Stabile Prodotto:]
$$ h \in \mathbb{R} \;\;\;\;\; h\underline{v} \in ker f$$
$$ f(h\underline{v}) = hf(\underline{v}) = \underline{0} $$

\end{itemize}

Avendo definito questi due concetti possiamo sfruttare alcune propietà per capire più facilente l'iniettività o surriettività di un applicazione lineare:
\begin{itemize}
\item[•] Se l'immagine del dominio combacia col codominio allora la funzione \textbf{è surriettivita}
$$ f(V) = Im f = W \Leftrightarrow \text{f è surriettiva} $$

\item[•] Se il $ker f = \{\underline{0}\} \Leftrightarrow$ $f$ è \textbf{iniettiva.}\\

\subitem \textbf{DIM $\Rightarrow$:}
$$ \underline{v},\underline{w} \in V \; \text{con} \; f(\underline{v}) = f(\underline{w}) \Rightarrow \underline{v} = \underline{w} $$
$$ f(\underline{v}-\underline{w}) = f(\underline{v})-f(\underline{w}) = \underline{0} \Rightarrow \underline{v}-\underline{w} \in ker f \Rightarrow \underline{v}-\underline{w} = \underline{0} \Rightarrow \underline{v} = \underline{w}$$

\subitem \textbf{DIM $\Leftarrow$:}
$$ \underline{0} \in ker f \;\;\;\;\; \{\underline{0}\} \subseteq ker f $$
$$ \underline{v} \in ker f \Rightarrow f(\underline{v}) = \underline{0} \Rightarrow \underline{v} = \underline{0} $$
Tutti gli elementi combaciano con il neutro

\end{itemize}


\subsubsection{Esempi}
\begin{itemize}
\item[•] $\underline{0}_v: V \rightarrow V (\underline{v} \rightarrow \underline{0})$\\
$$ Im f = f(V) = \{\underline{0}\} $$
$$ Ker f = V$$ 
$$ \text{INIETTIVA E SURRIETTIVA } \Leftrightarrow V = \{\underline{0}\}$$

\item[•] $ id_v V \rightarrow V (\underline{v} \rightarrow \underline{v})  $
$$ Im f = f(id_v) = V \; INIETTIVA$$
$$ Ker f = \{\underline{0}\} \;  SURRIETTIVA$$

\item[•] $\mathbb{R}^2 \rightarrow \mathbb{R}^2$
$$ (x,y) \rightarrow \begin{pmatrix}1 & 2 \\ 0 & 1\end{pmatrix} \begin{pmatrix}x \\ y \end{pmatrix} = \begin{pmatrix}x+2y \\ y\end{pmatrix} $$

\subitem • Per verificare l'iniettività dobbiamo verificare il $ker f$ sia formato solo dal vettore nullo, quindi vediamo per quali valori di $x,y$ $f(x,y) = 0$:
$$ \systeme{x+2y=0, y=0} \Rightarrow \systeme{x=0,y=0} \; \text{È INIETTIVA} $$

\subitem • Per verificare la surrietività prendiamo un sistema di generatori, in questo caso quella canonica:
$$ <(1,0),(0,1)> \; dim=2$$
$$ f(\mathbb{R}^2) = <f(1,0),f(0,1)> = <(1,0),(2,1)>$$
Essendo indipendenti hanno $dim=2$ quindi è surriettiva visto che è rimasto un sistema di generatori.

\item[•] $f: \mathbb{R}_2[x] \rightarrow \mathbb{R}[x] (ax^2+bx+c \rightarrow 2ax+b)$\\

\subitem •  Per verificare l'iniettività dobbiamo verificare il $ker f$ sia formato solo dal vettore nullo, quindi vediamo per quali valori di $a,b,c$ $f(ax^2+bx+c) = 0$:
$$ \underline(v) \in \mathbb{R}_2[x] / f(\underline{v})= \underline{0}$$
$$ f(ax^2+bx+c) =  2ax+b = \underline{0} \Leftrightarrow a = 0 = b $$
Quindi per $ker f$ sarà formato da tutti i valori $c \in \mathbb{R}$ quindi non è iniettiva.
$$ ker f = \{c/c \in \mathbb{R}\} DIM=1$$

\subitem •  Per verificare la surrietività prendiamo un sistema di generatori, in questo caso quella canonica:
$$ <x^2,x,1> dim=3$$
$$ f(<x^2,x,1>) = <2x,1,0> = <2x,1> dim = 2 \; \textbf{NON È SURRIETIVA}$$

\item[•] $f: \mathbb{R}^2 \rightarrow \mathbb{R}^3 $
$$ (x,y) \rightarrow \begin{pmatrix} 1&2\\3&4\\0&1 \end{pmatrix} \begin{pmatrix} x \\ y \end{pmatrix} $$

\subitem •  Per verificare l'iniettività dobbiamo verificare il $ker f$ sia formato solo dal vettore nullo, quindi vediamo per quali valori di $x,y$ $f(x,y) = 0$:
$$ \begin{pmatrix} 1&2\\3&4\\0&1 \end{pmatrix} \begin{pmatrix} x \\ y \end{pmatrix} = \begin{pmatrix} 0 \\ 0 \end{pmatrix}  \Leftrightarrow \systeme{x+2y=0, 3x+4y=0, y=0} \Leftrightarrow \systeme{x=0,y=0} $$

\end{itemize}

\subsection{Altre propriepietà}
Consideriamo una generica funzione lineare $f: V \rightarrow W$:
\begin{itemize}
\item[•] Il monomorfismo (omomorfismo iniettivo) mantiene l'indipendenza, cioè:
$$  \underline{v}_1,...,\underline{v}_n \; INDIP. \Rightarrow f(\underline{v}_1),...,f(\underline{v}_n) \; INDIP.$$
$$ \text{BASE DI V} \rightarrow \text{BASE DI }f(V) $$
\textbf{DIM:}
Prendiamo una combinazione lineare dei vettori immagine ed eguagliamola all’elemento
neutro:
$$ f(h_1\underline{v}_1+...+h_n\underline{v}_n) = h_1f(\underline{v}_1)+...+h_nf(\underline{v}_n) = \underline{0}   $$
Possiamo leggerlo anche nel seguente modo:
$$ h_1\underline{v}_1+...+h_n\underline{v}_n \in ker f $$
Affinché questi vettori siano indipendenti bisogna dimostrare che tutti gli scalari siano necessariamente pari a zero ma essendo i vettori indipendenti per ipotesi ed avendolo uguagliato all'elemento neutro troviamo che tutti gli scalari sono nulli.

\item[•] Conserando sempre la funzione linieare possiamo definire la seguente equivalenza:
$$ dim(ker f) + dim(Im f) = dim (V) $$ 
La dimensione del dominio è uguale alla somma delle dimensioni dell'immagine del dominio e del ker di f.\\
\textbf{DIM:}\\
Ragioniamo per casi:
\subitem • $ker f = \{\underline{0}\}$\\
Quindi la funzione ha $dim=0$ ed è iniettiva (monomorfismo) per quanto visto sopra conserva la base (indipendenza) e quindi la dimensione quindi:
$$ 0 + dim(Im f) = dim (V) $$

\subitem • $ker f = V$:\\
Come abbiamo dagli esempi di prima l'unico caso in cui $ker f = V$ è quando $f$ è la funzione nulla e sempre per la precendente osservezione $dim (Im f) = 0$, quindi:
$$ dim (ker f) + 0 = dim (V) $$

\subitem • $\{0\} < ker f < V$\\
Prendiamo un base per il $ker$ ed estendiamo a $V$:
$$ \underbrace{\underline{e}_1,...,\underline{e}_m}_{\text{base di ker}},\underbrace{\underline{e}_{m+1},...,\underline{e}_n}_{\text{base estesa di V}}$$
Quindi abbiamo le seguenti dimensioni:
$$ \underbrace{dim(Ker f)}_m  + dim(Im f) = \underbrace{dim(V)}_n  $$
Quindi $dim(Im f) = n-m$, dimostriamo che $f(\underline{e}_{m+1},..,\underline{e}_n)$ sia base di $Im f$\\
Prendiamo un sistema di generatori per $Im f$:
$$ Im f = <f(\underline{e}_1),...,f(\underline{e}_m),f(\underline{e}_{m+1}),...,f(\underline{e}_n)>$$
Ma i vettori appartenenti al $ker f$ sono nulli quindi possono essere rimosso:
$$ Im f = <f(\underline{e}_{m+1}),...,f(\underline{e}_n)>$$
Non ci resta che dimostrare l'indipendenza:
$$ h_1f(\underline{e}_{m+1})+...+h_nf(\underline{e}_{n}) = \underline{0}$$
$$ f(h_1\underline{e}_{m+1}+...+h_n\underline{e}_{n}) = \underline{0} \Rightarrow$$
$$ \Rightarrow  h_1\underline{e}_{m+1}+...+h_n\underline{e}_{n} \in Ker f$$



\end{itemize}
Piccolo tips: perché una funzione sia iniettiva la dimensione del dominio sia minore o uguale alla dimensione del codominio.

\subsection{Cordinazione Associata}
Preso uno spazio vettoriale $V$ e un suo riferimento $(\underline{e}_1,...,\underline{e}_n)$ possiamo considerare la seguente funzione lineare $C_{\mathtt{R}}$:
$$ \underline{v} = h_1\underline{e}_1+...+h_n\underline{e}_n \rightarrow (h_1,...,h_n) \in \mathbb{R}^n $$
Andiamo a verificare che sia una funzione lineare:
\begin{itemize}
\item[Prodotto lineare:]
$$ f(h\underline{v}) = hf(\underline{v}) $$
$$ (h\underline{e}_1,...,h\underline{e}_n)= h(\underline{e}_1,...,\underline{e}_n)$$

\item[Somma lineare:]
$$ \underline{v} = h_1\underline{e}_1+...+h_n\underline{e}_n \;\;\; f(\underline{v})=(h_1,...,h_n) $$
$$ \underline{w} = k_1\underline{e}_1+...+k_n\underline{e}_n \;\;\; f(\underline{w})=(k_1,...,k_n) $$
$$ \underline{v}+\underline{w}= (h_1+k_1)\underline{e}_1+...+(h_n+k_n)\underline{e}_n $$
$$ f(\underline{v} +\underline{w}) = (h_1+k_1,...,h_n+k_n) $$
\end{itemize}

Inoltre la funzione è \textbf{iniettiva} poiché il $Ker C_{\mathtt{R}}= \{\underline{0}\}$.\\
È anche \textbf{suriettiva} poiché $C_{\mathtt{R}}(V) = <(1,0,...,0),(0,1,...,0),...,(0,0,...,1)> = \mathbb{R}^n$.\\
Quindi questa funziona è un \textbf{isomorfismo} che conserva l'indipendenza.
È grazie a questa funzione che precedentemente abbiamo considerato le terne associate.

\subsubsection{Corolarrio}
Ogni spazio vettoriale finitamente generato (diverso da 0) di dimensione $n$ è isomorfo ad $\mathbb{R}^n$:
$$ \forall V \neq \{\underline{0}\} \exists n: V \simeq \mathbb{R}^n $$

\subsubsection{Proposizione}
Sia $f: V \rightarrow W$ un isomorfismo (funzione lineare biettiva) esiste la sua inversa $f^{-1}: W \rightarrow V$ anche sarà un isomorfimo.\\
Dimostriamo che l'inversa sia una funzione lineare:
\begin{itemize}
\item[Somma lineare:]
$$\underline{w},\underline{w}^{\prime} \in W(codominio) \;\;\; f^{-1}(\underline{w}+\underline{w}^{\prime}) = f^{-1}(\underline{w})+f^{-1}(\underline{w}^{\prime})$$
Essendo per definizione surriettiva allora:
$$ \exists \underline{v},\underline{v}^{\prime} \in V \;\;\;  \underline{w}=f(\underline{v}) \;\;\; \underline{w}^{\prime}=f(\underline{v}^{\prime}) $$
Andando a sostituire esce:
$$ f^{-1}(f(\underline{v})+f(\underline{v}^{\prime})) = f^{-1}(f(\underline{v}+\underline{v}^{\prime}))$$
Quindi alla fine [DA RIVEREDER]:
$$ f^{-1}(f(\underline{v}+\underline{v}^{\prime})) = \underline{v}+\underline{v}^{\prime} = f^{-1}(\underline{w})+f^{-1}(\underline{w}^{\prime})$$

\item[Prodotto lineare:]
$$ h \in \mathbb{R} \;\;\; \underline{w} \in W \;\;\; \exists \underline{v} \in V = \underline{w} = f (\underline{v}) $$
$$ f^{-1}(h \underline{w}) = h f^{-1}(\underline{w}) = h \underline{v} $$
$$ f^{-1}(h f(\underline{v})) = f^{-1}(f(h\underline{v})) = h \underline{v}$$

\end{itemize}





\end{document}