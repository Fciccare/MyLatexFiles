\documentclass[12pt,a4paper]{article}
\usepackage[a4paper, total={6in, 9.5in}]{geometry}
\usepackage[utf8]{inputenc}
\usepackage{babel}
\usepackage{eucal}
\usepackage{color}
\usepackage{cancel}
\usepackage{svg}
\usepackage[utf8]{inputenc}
\usepackage{graphicx}
\graphicspath{ {./images/} }
\usepackage{import}
\usepackage[T1]{fontenc}
\usepackage{blindtext}
%\usepackage{amsmath}
\usepackage{mathrsfs}
\usepackage{amsfonts}
\usepackage{mathtools}
\usepackage{systeme}
\usepackage{eqparbox}
\newcommand{\verteq}{\rotatebox{90}{$\,=$}}
\newcommand{\equalto}[2]{\underset{\scriptstyle\overset{\mkern4mu\verteq}{#2}}{#1}}
\newcommand{\mystackrel}[3][T]{\stackrel{\eqmakebox[#1]{\scriptsize#2}}{#3}}
%\usepackage{lipsum}
\usepackage{listings}
\usepackage{hyperref}
\usepackage{soul}
\DeclareRobustCommand{\hlcyan}[1]{{\sethlcolor{cyan}\hl{#1}}}
\hypersetup{linktoc=all}
\title{Algoritmi e Strutture Dati 2023-24\\ (M. Benerecetti)}
\author{}
\date{}

\definecolor{backcolour}{rgb}{0.95,0.95,0.92}
\definecolor{codegreen}{rgb}{0,0.6,0}
\definecolor{dkgreen}{rgb}{0,0.6,0}
\definecolor{gray}{rgb}{0.5,0.5,0.5}
\definecolor{mauve}{rgb}{0.58,0,0.82}

%Define a custom style
\lstdefinestyle{Style01}{
    backgroundcolor=\color{backcolour},   
    commentstyle=\color{codegreen},
    basicstyle=\ttfamily\footnotesize,
    breakatwhitespace=false,         
    breaklines=true,                 
    keepspaces=true,                 
    numbers=left,       
    numbersep=5pt,                  
    showspaces=false,                
    showstringspaces=false,
    showtabs=false,                  
    tabsize=2,
}

\lstdefinestyle{Style02}{
  frame=tb,
  aboveskip=3mm,
  belowskip=3mm,
  showstringspaces=false,
  columns=flexible,
  basicstyle={\small\ttfamily},
  numbers=left,       
  numbersep=5pt,               
  backgroundcolor=\color{backcolour},
  numberstyle=\tiny\color{gray},
  keywordstyle=\color{blue},
  commentstyle=\color{dkgreen},
  stringstyle=\color{mauve},
  breaklines=true,
  breakatwhitespace=true,
  tabsize=2
}

% Use \lstset to make myStyle the global default
\lstset{style=Style02}

\begin{document}
\maketitle
\tableofcontents


\section{Algoritmo di Conteggio}
Descrivere un algoritmo che accetta come input un intero $N \ge 1 $ e produce in output il numero di coppie
ordinate $i,j \in \mathbb{N} \; \, \; (i,j): 1 \le i \le j \le \mathbb{N} $\\
Esempio:\\
\begin{itemize}
\item Input:N=$4$ 
\item Output:$10$ \{(1,1),(1,2),(1,3),(1,4),(2,2),(2,3),(2,4),(3,3),(3,4),(4,4)\}
\end{itemize}

\begin{lstlisting}[language=Python]
Conta(N):
	ris = 0;
	for i=1 to N do
		for j=1 to N do
			if i<=j then
				ris = ris+1
	return ris
\end{lstlisting}

Andiamo a definire per ogni riga un costo:
\begin{itemize}
\item 2) Assegnamento costante, 1 operazione elementare
\item 3) Al primo giro: Assegnamento + confronto (2 operazioni elementari), successivi giri: Incremento+confronto (2 operazione elementari)
\item 4) idem $3$
\item 5) 2 letture + confronto (3 operazioni elementari)
\item 6) lettura+scrittura+assegnamento (3 operazioni elementari)
\item 7) 1 operazione elementare
\end{itemize}

Ognuna di queste operazioni (righe) vengono eseguite più di una volta, quindi il costo sarà maggiore, andiamo ad esprimerlo:\\

\begin{itemize}
\item 2) Costo = 1 (fuori dal ciclo)
\item 3) La testa viene eseguita $n+1$ volte poiché abbiamo anche l'ultima operazione per uscire dal ciclo, quindi Costo = $2*(n+1) = 2 * \sum_{i=1}^{N+1} 1$
\item 4) Questo for verrà ripetuto $N$ volte poiché il corpo del for viene eseguito $N$ volte, quindi il suo costo sarà:
$$ \underbrace{2}_{\text{costo dell'operazione}}*\underbrace{\sum_{i=1}^{N}}_{\text{for esterno}} \underbrace{\sum_{j=1}^{N+1} 1}_{\text{for interno}}$$
\item 5) L'if stando in entrambi i for avrà un costo di: $3*\sum_{i=1}^{N} \sum_{j=1}^{N+1} 1$
\item 6) Quando $i=1$ il numero di esecuzioni della linea 5 è $N$, quando $i=2$ il numero di esecuzioni sarà $N-1$, perché quando $j=1$ non entriamo, quando $i=3$ il numero di esecuzioni sarà $N-2$, e cosi via, quindi in generale quando i è uguale ad un certo numero $k(con k<=N)$ il numero di volte in cui è soddisfatta la condizione dell’if è $N-(K - 1)$:
\item 7) Costo = 1 (fuori dal ciclo)
\end{itemize}
\section{Lezione 03 - 21-09-2023}

\subsection{Algoritmo v3}
L'algoritmo può essere anche migliorato, riusciendo ad arrivare ad una complessità \textbf{lineare}, nel seguente modo:

\begin{lstlisting}[language=Python]
int Max_seq_sum_3(int N, array a[])
	maxsum = 0
	sum = 0
	for j=1 to N
		if (sum + a[j] > 0) then
			sum = sum + a[j]
		else
			sum = 0
		maxsum = max(maxsum,sum)
	return maxsum
\end{lstlisting}
Il ragionamento è il seguente:
Se prendiamo un insieme di numeri da sommare, (da i ad a), possiamo controllare se esso è positivo o negativo.
Nel caso in cui $\sum_{e=i}^{a} A[e]$ risultasse positiva, andiamo a espandere il nostro range fintantochè il risultato della sommatoria riamanga positivo.
Nel caso in cui invece il risultato fosse negativo, non ci conviene tenere traccia dei numeri più piccoli di quel range, dato che se quella sommatoria è minore del numero successivo alla sommatoria, non ha senso tenerne conto. E quindi invece ha senso tenere traccia del numero successivo. Da quel numero poi sommare i numeri successivi continuando il processo sopracitato.


\subsection{Strutture Dati - Insieme Dinamico}
Vediamo come rappresentare un insieme di dati dinamico $S$ (con insieme dinamico si intende una collezione di elementi variabile nel tempo, quindi è possibile aggiungere o rimuovere elementi);%Rubata da Simone Cerrone
$$ S=\{a_1,a_2,...,a_n\} \;\;\; n \ge 0  $$
Andiamo a definire alcune operazioni:
\begin{itemize}
\item Insert$(S,a) \rightarrow S^{\prime}$ ($ S^{\prime} = S \cup \{x\}$)
\item Deletes$(S,a) \rightarrow S^{\prime}$ ($S^{\prime} = S \backslash \{x\}$)
\item Search$(S,a) \rightarrow \{True,False\}$
\item Massimo$(S) \rightarrow a$
\item Minimo$(S) \rightarrow a$
\item Successore$(S,a) \rightarrow a^{\prime}$
\item Predecessore$(S,a) \rightarrow a^{\prime}$
\end{itemize}
\section{Lezione 04 - 17/03/2023}

\subsection{Spazi Vettoriali su R}
Sia $V$ un insieme non vuoto, definiamo due operazioni:
\begin{enumerate}
\item [] Interna $ +: VxV -> V $ (somma vettoriale) 
\item [] Esterna $ \cdot: RxV -> V $ (scalare per un vettore)  R è campo
\end{enumerate}
Posto $(V,+,\cdot)$ si dice \textbf{spazio vettoriale su R} $\Leftrightarrow$

\begin{enumerate}
\item $ (V,+) $ è un gruppo abeliano, quindi:
	\subitem Associatività
	\subitem Commutatività
	\subitem Neutro
	\subitem Tutti gli elementi invertibili
\item $ \forall \underline{v} \in V $ tale che $ \underline{v} \cdot 1 = \underline{v}$ (associtività mista)
\item $ \forall h,k \in R$, $\forall \underline{v} \in V$ tale che $ (hk)\underline{v} = h(k\underline{v})$
\item $ \forall h,k \in R$, $\forall \underline{v} \in V$ tale che $ (h+k)\cdot\underline{v} = h\cdot\underline{v} + k\cdot\underline{v}$ (distrub. tra $\cdot$ e $+$ in $R$)
\item $ \forall h,k \in R$, $\forall \underline{v} \in V$ tale che $ h(\underline{v}+\underline{w}) = h\cdot\underline{v} + h\cdot\underline{v}$  (distrub. tra $\cdot$ e $+$ in $V$)
\end{enumerate}

\subsection{Esempi Spazi Vettoriali}

\subsubsection{Spazio Vettoriale numerico di ordine n}
Verifichiamo che $(R^n,+,\cdot)$ sia uno spazio vettoriale, ma prima facciamo un esempio:
$$ (1,2,3) + (0,1,2) = (1,3,5) \:\:\:\:\:\:\: 3(3,2,4)=(9,6,12)$$
Andiamo a verificare che sia spazio vettoriale:
\begin{enumerate}
\item $ (R^n, +) $ gruppo abeliano:
	\subitem * Associatività e Commutatività banalmente eraditati da $+$
	\subitem * Neutro: $\underline{0} = (0,0,...,0)$
	\subitem * Inverso: $-(x_1,...,x_n) = (-x_1,-x_2,...,-x_n)$
\item Banale ereditatà di $\cdot$
\item $ (hk)(x_1,...,x_n) = (hkx_1,...,hkx_n) = h(kx_1,...,kx_n) = h(k(x_1,...,x_n))$
\item DA DIMOSTARE
\item DA DIMOSTARE
\end{enumerate}

\subsubsection{Spazio Vettoriale di una matrice di ordine m,n}
Possiamo considerare $(M_{m,n}(R), +, \cdot)$ come una lunga riga, quindi si accomuna al caso precedente.
\subsubsection{Spazio Vettoriale polinomiale}

\subsubsection{Spazio Vettoriale polinomiale di al più n}

\subsubsection{Spazio Vettoriale dei vettori geometrici in un punto O}



 




\section{Lezione 05 - 16-03-2023}

\subsection{Definizioni simboli Insiemestici ed Eventi}
\begin{center}

\begin{tabular}{ |c|c|c| } 
 \hline
 Begin & Algebra degli Insiemi & Logica degli Eventi \\ 
 $\Omega$ & Insieme universo & Spazio Campione \\ 
 $A$ & Insieme & Evento \\
 $A^C$ & Complementare di A & Negato di A \\
 $A \cup B$ & Unione di A e B & OR degli eventi, deve verificarli almeno uno tra A e B \\
 $ A \cap B $ & Intersezione tra A e B & AND degli eventi, devono verificarsi entrambi \\
 $ \bigcup_{k=1}^n A_k $ & Unione finita & n verifica almeno una tra $A_1, A_2,...,A_n$ \\
 $ \bigcup_{k=1}^{\infty} A_k $ & Unione numerabile & boh \\
 $ \bigcap_{k=1}^n A_k$ & Intersezione finita & Si verifica se tutti gli eventi $A_1,...,A_n$ si verificano \\
 $ \bigcap_{k=1}^{\infty} A_k$ & Unione numerabile & boh  \\
 $ \emptyset $ & Insieme Vuoto & Evento Impossibile \\
 $ A \cap B = \emptyset $ & A e B sono disgiunti & Eventi Incompatibili \\
 $ A \subset B $ & A contenuto in B & Il verificare di A implica il verificare di B \\
 $ \uplus_k A_k = \Omega $ & Ricomprimento disgiunto (partizione) & $A_1, A_2,...,A_n$ eventi neccessari\\
 \hline
\end{tabular}

\end{center}


\subsection{Esempio Lancio Moneta 1}
Poniamo caso che vogliamo descrivere l'evento che al terzo lancio di una moneta esca Testa, per prima cosa scegliamo un spazio campione:
$$ \Omega = {\{T,C\}}^{N} $$
Una moneta ha solo due casi, testa oppure croce, ora descriviamo che testa esca al terzo lancio:
$$ T_3 = \{ (w_1,w_2,...) \in \Omega: w_3 = "T" \} $$ 
abbiamo descritto questo eveno tramite propietà degli insiemi, nel caso volessimo esprimere lo stesso concetto ma per le croci ci basta fare il comlemento:
$$ T_3^C = C_3 = \{ (w_1,w_2,...) \in \Omega: w_3 = "C" \}  $$

\subsection{Esempio Lancio Moneta 2}
Poniamo invecere di voler complicare le cose, voglia esprimere l'evento che escano due Testa prima di due Croci, chiamiamo questo evento $A$, questo evento ha infinite possibilità, facciamo alcuni esempi:
$$ A_2 = T_1, T_2, \Omega $$
$$ A_3 = C_1, T_2, T_3, \Omega $$
$$ A_4 = T_1, C_2, T_3, T_4, \Omega $$
$$ A_5 = C_1, T_2, C_3, T_4, T_5 \Omega $$
Possiamo fare alcune osservazioni, $A_2, A_3$ sono incompatibili, non possono verificarci contemporaneamente, invece $A_5$ è incompatibile con $A_2,A_3,A_4$ .
Possiamo esprimere il verificarsi dell'evento $A$ in vari modi:\\
$$ A = A_2 \cup A_3 $$
$$ A = A_2 \cup A_3 \cup A_4 $$
$$ A = A_2 \uplus A_3  \uplus A_4 $$
$$ A = A_2 \uplus A_3 \uplus A_4 \uplus A_5 $$
Possiamo esprimere questo evento $A$ tramite \textbf{Unione Numberabile}:
\begin{equation*}
A_n =
\begin{cases}
C_1,T_2,...,C_{n-2},T_{n-1},T_n  \:\:\:\:\:\text{n dispari inizia con una croce}\\
T_1,C_2,...,C_{n-2},T_{n-1},T_n  \:\:\:\:\:\text{n pari inizia con una testa}
\end{cases}
\Rightarrow A = \bigcup_{n=2}^{\infty} \:\:\:\:\: \text{è un evento}
\end{equation*}

\subsection{Famiglia/Classi}
\blindtext

\subsection{Algebra e Sigma Algebra}
Preso un $\Omega$ spazio campione e un $a$ (a tondo), classe non vuota di sottinsiemi di $\Omega$ allora:
$$ a \: \text{è un algebra} \Leftrightarrow $$
$$ i) A \in a \Rightarrow A^C \in a \:\:\: \text{(a è chiusa rispetto il complemento)} $$
$$ ii) A_1, A_2 \in a \Rightarrow A_1 \cup A_2 \in a \:\:\: \text{(a è chiusa rispetto l'unione di due elementi)} $$
C'è un anche una sua variante chiamanta Sigma(numerabile) Algebra che si definisci così:
$$ \sigma-algebra \Leftrightarrow $$
$$ i)uguale $$
$$ ii) n \in N, A_n \in a \Rightarrow \bigcup_{n=1}^{\infty} A_n \in a$$
\subsubsection{Osservazioni}
Posto $a=\{\{2,3\}, \{6\}, \{4,5\}\}$, osserviamo i seguenti esempi:
\begin{center}
$ \{4,5\} \subseteq a $ SBAGLIATO\\
$ \{4,5\} \in a $ CORRETTO\\
$ \{\{4,5\}\} \subseteq a $ CORRETTO
\end{center}

\subsubsection{Casi Particolari}
Poniamo $A \subseteq \Omega$, si definisce \textbf{algebra(sigma) banale}, $a$ posto come:
$$ a = \{\emptyset, \Omega \}$$
È l'unica algebra a due elementi, ovviamente entrambe le propietà sono banalmente dimostrate poiché:
$$ \Omega^C = \emptyset $$
$$ \Omega \cup \emptyset = \Omega \in a $$
Gli elementi $\emptyset$ e $\Omega$ sono neccessari per essere un \textbf{algebra}.
Poniamo caso di un $a=\{A, A^c\}$ questa non è un algebra poiché $A \cup A^c = \Omega \not \in a$, se aggiunssimo solo $\Omega$ non sarebbe rispettata la prima condizione poiché $ \Omega^c = \emptyset \not \in a$.\\
Ricapitolando:
$$ a=\{A, A^C\} \:\:\: \textbf{non è algebra} \:\:\:\:\: a=\{A,A^C,\emptyset,\Omega\} \:\:\:\:\:\textbf{è algebra (sigma)}$$
Per contrapposizione la (sigma) algebra più grande è $P(\Omega)$, tutte le altre algebra(sigma) sono sottoinsiemi di $P(\Omega)$

\subsection{Propietà (conseguenze)}
\begin{enumerate}
\item $a$ è una algebra (sigma) $\Rightarrow \: \emptyset,\Omega \in \: a$ (come abbiamo osservato prima)\\Tutti gli elementi dell'algebra banale devono essere presenti in ogni algebra(sigma).
\item L'unione finita di elementi di un algebra (sigma) appartiene comunque ad $a$\\ Per $ii)$ abbiamo visto come l'unione si applica per due elementi, ma essendo $\cup$ associativa nel caso di $n-elementi$ basta operarli a due a due e quindi portare questa propietà fino a n elementi.
\item $ Sigma\:algebra \Rightarrow Algebra \:\:MA\:\: Sigma\: algebra \not \Leftarrow Algebra $\\
Questo poiché un unione finita da 0 a $+\infty$ non appartiene a tutte le algebra, cose che invece accade con le sigma algebra.
\end{enumerate}










\end{document}