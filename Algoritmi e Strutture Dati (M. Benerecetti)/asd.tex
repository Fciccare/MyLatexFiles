\documentclass[12pt,a4paper]{article}
\usepackage[a4paper, total={6in, 9.5in}]{geometry}
\usepackage[utf8]{inputenc}
\usepackage{babel}
\usepackage{eucal}
\usepackage{color}
\usepackage{cancel}
\usepackage{svg}
\usepackage[utf8]{inputenc}
\usepackage{graphicx}
\graphicspath{ {./images/} }
\usepackage{import}
\usepackage[T1]{fontenc}
\usepackage{blindtext}
%\usepackage{amsmath}
\usepackage{mathrsfs}
\usepackage{amsfonts}
\usepackage{mathtools}
\usepackage{systeme}
\usepackage{eqparbox}
\newcommand{\verteq}{\rotatebox{90}{$\,=$}}
\newcommand{\equalto}[2]{\underset{\scriptstyle\overset{\mkern4mu\verteq}{#2}}{#1}}
\newcommand{\mystackrel}[3][T]{\stackrel{\eqmakebox[#1]{\scriptsize#2}}{#3}}
%\usepackage{lipsum}
\usepackage{listings}
\usepackage{hyperref}
\usepackage{soul}
\DeclareRobustCommand{\hlcyan}[1]{{\sethlcolor{cyan}\hl{#1}}}
\hypersetup{linktoc=all}
\title{Algoritmi e Strutture Dati 2023-24\\ (M. Benerecetti)}
\author{}
\date{}

\definecolor{backcolour}{rgb}{0.95,0.95,0.92}
\definecolor{codegreen}{rgb}{0,0.6,0}
\definecolor{dkgreen}{rgb}{0,0.6,0}
\definecolor{gray}{rgb}{0.5,0.5,0.5}
\definecolor{mauve}{rgb}{0.58,0,0.82}

%Define a custom style
\lstdefinestyle{Style01}{
    backgroundcolor=\color{backcolour},   
    commentstyle=\color{codegreen},
    basicstyle=\ttfamily\footnotesize,
    breakatwhitespace=false,         
    breaklines=true,                 
    keepspaces=true,                 
    numbers=left,       
    numbersep=5pt,                  
    showspaces=false,                
    showstringspaces=false,
    showtabs=false,                  
    tabsize=2,
}

\lstdefinestyle{Style02}{
  frame=tb,
  aboveskip=3mm,
  belowskip=3mm,
  showstringspaces=false,
  columns=flexible,
  basicstyle={\small\ttfamily},
  numbers=left,       
  numbersep=5pt,               
  backgroundcolor=\color{backcolour},
  numberstyle=\tiny\color{gray},
  keywordstyle=\color{blue},
  commentstyle=\color{dkgreen},
  stringstyle=\color{mauve},
  breaklines=true,
  breakatwhitespace=true,
  tabsize=2
}

% Use \lstset to make myStyle the global default
\lstset{style=Style02}

\begin{document}
\maketitle
\tableofcontents


\section{Lezione 02 - 08/03/2023}

\subsection{Regola Moltiplicativa}
Se una procedura di scelta si può suddividere in $r$ sottoprocedure allora il numero $n$ delle possibili scelte è dato da:
$$ n = n_1*n_2*...*n_r$$
Dove $i=1,2,...,r$ rappresenta il numero delle possibili scelte nella sottoprecedura i-sima.\\
\subsubsection{Esempio Cartellini Camicie}
Vogliamo sapere quanti cartellini delle camicie dobbiamo fabbricare avendo i seguenti dati:
4 Taglie, 2 Foggie, 7 Colori.\\
Usando la regola moltiplicativa poniamo $r=3$ avendo tre possibili varianti, $n_1=4$ per le taglie, $n_2=2$ per le foggie, $n_3=7$ per i colori, ora calcoliamo il totale:
$$ n = n_1*n_2*n_3 = 4*2*7 = 56 \:\:\: \textbf{CARTELLINI} $$

\subsection{Fattoriale}
%Sia $n$ un intero positivo. Il prodotto dei primi $n$ interi positivi è chiamato fattoriale di n e si pone come
Il fattoriale di $n>=0$ si esprime come $n!$ ed è definita come il prodotto di tutti i numeri precendenti, definiamo tramite ricorsione:
\begin{equation*}
n! = 
\begin{cases}
1 \: \: \: \text{SE} \: \: \: n=0\\
n*(n-1)! \: \: \: \text{SE} \: \: \: n>0
\end{cases}
\end{equation*}
Esempio: 
$$6! = 1*2*3*4*5*6 = 720$$
$$ \frac{13!}{11!} = \frac{13*12*\cancel{11!}}{\cancel{11!}} = 13*12 = 156 $$
$$ \frac{n!}{(n-1)!} = \frac{n(n-1)!}{(n-1)!} = n $$
\newpage

\subsection{Coifficiente Binomiale}
Presi $n e k$ con $k<=n$, possiamo definire il cofficiente binomiale in questo modo:
$$ \binom{n}{k} = \frac{n!}{k!(n-k)!} $$
%Esempio:
$$ \binom{6}{4} = \frac{6!}{4!(6-4)!} = \frac{6!}{4!*2!} = \frac{6*5*\cancel{4!}}{\cancel{4!}*2!} = \frac{\cancel{6}^3*5}{\cancel{2}} = 3*5 = 15 $$

\subsubsection{Propietà del C.B. con esempi}
Andiamo ad elencare alcune propietà del coifficiente binomiale con i rispettivi esempi:
\begin{description}
  \item [Propietà 01] 
  	$$ \binom{n}{n} = 1 $$
	$$\binom{5}{5} = \frac{\cancel{5!}}{\cancel{5!}*\equalto{(5-5)}{0!=1}!} = 1 $$ 
	
  \item [Propietà 02] 
  	$$ \binom{n}{n-1} = n $$
	$$\binom{5}{4} = \frac{5*\cancel{4!}}{\cancel{4!}*\equalto{(5-4)}{1}!} = 5 $$ 
  \item [Propietà 03]
  $$ \binom{n}{k} = \binom{n}{n-k} $$
  $$ \binom{12}{4} = \frac{12!}{4!*\equalto{(12-4)!}{8!}} = \frac{\cancel{12}^{\cancel{3}}*11*\cancel{10}^5*9*\cancel{8!}}{\cancel{2}*\cancel{3}*\cancel{4}*\cancel{8!}} = 5*9*11 = 495 = \frac{12!}{8!*\equalto{(12-8)!}{4!}} = \binom{12}{8} $$
  \item [Propietà 04 Se $k<n$ ]
  $$ \binom{n}{k} = \binom{n-1}{k-1} + \binom{n-1}{k} $$
  $$ \binom{}{}$$
  \item [Propietà 05 ($n=6, k=3$)]
  $$ \binom{n+1}{k} = \binom{n}{k} + \binom{n}{k-1}  $$
\begin{equation*}
\resizebox{\textwidth}{!}
{%
$\binom{7}{3} = \frac{7*\cancel{6}*5*\cancel{4!}}{\cancel{3!}*\cancel{4!}} = 7*5 = 35 = 20+15 =\frac{\cancel{2}*\cancel{3}*4*5*\cancel{6}}{\cancel{6} * \cancel{6}} + \frac{\cancel{6}^3*5*\cancel{4!}}{\cancel{2}*\cancel{4!}} = \frac{6!}{3!*3!} + \frac{6!}{2!*4!} = \binom{6}{3} + \binom{6}{2}$%
}
\end{equation*}
\end{description}

Un possibile uso del coifficiente binomiale è quello di poter sapere il numero dei sottoinsiemi di ordine $k$ con $n$ valori.\\
Esempio poniamo di avere un insieme $S=\{1,2,3,4\}$ con cardilinità $\#S = 4$, vogliamo sapere quanti sono tutti i possibili sottoinsiemi di ordine due:

$$ \binom{4}{2} = \frac{4!}{2!*(4-2)!} = \frac{\cancel{4}^2*3*\cancel{2!}}{\cancel{2}*\cancel{2!}} = 2*3 = 6$$

$$ T={ \{1,2\}, \{1,3\}, \{1,4\}, \{2,3\}, \{2,4\}, \{3,4\}} \: \: \#T=6 $$


\subsection{Coifficiente Multinomiale}
Sia $n$ un intero posi+tivo e $n_1,n_2...n_r$ interi tali che $n_1+n_2+...+n_r = n$, possiamo scrivere il coifficiente multinomilae in questo modo:
$$ \binom{n}{n_1,n_2,...,n_r} = \frac{n!}{n_1!*n_2!*...*n_r!} $$
%Esempio:
$$ \binom{7}{2,3,2} = \frac{7!}{2!*3!*2!} = \frac{7*6*5*\cancel{4}*\cancel{3!}}{\cancel{4}*\cancel{3!}} = 210 \:\:\:(2+3+2 = 7) $$
\newpage
\subsection{Problema del Contare}
Sia $S$ un insieme costituito da un numero $n$ finito di elementi distinti. In problemi coinvolgenti la selezione occorre distungere il caso in cui questa è effettuata con o senza ripetizioni. Si può inoltre porre o meno l'attenzione sull'ordine con cui gli elementi di S si presentano nella selezioni.

\subsection{Disposizioni e Combinazioni}
Per ovviare al problema del contare andiamo a definire le seguenti classificazioni:\\\\
\textbf{Disposizione:} è una selezione dove l'ordinamento è \textbf{IMPORTANTE}.\\
Possiamo suddividerla in:\\
Disposizione: è ammessa la \textbf{ripetizione} di qualunque elemento\\
Diposizione Semplice: \textbf{non è amessa} la ripezioni\\\\
\textbf{Combinazioni: } è una selezione dove l'ordinamente \textbf{non è IMPORTANTE}.\\
Possiamo suddividerla in:\\
Combinazioni: è ammessa la \textbf{ripetizione} di qualunque elemento\\
Combinazioni Semplice: \textbf{non è amessa} la ripezioni\\\\

\subsection{Disposizioni semplici/ripetizioni}
\label{sec:disposizioni}
Per calcolare tutte le k-disposizioni con ripetizione di S usiamo questa formula:
$$ D^{(r)}_{n,k} = n^k$$ 
%$$ D^{(r)}_{n,k} = n^k \: \: \: \: \: (k>=n)$$ 

Per calcolare tutte le k-disposizioni semplici di S usiamo questa formula:

$$ D_{n,k} = \frac{n!}{(n-k)!} \: \: \: \: \: (k<=n)$$ 

\begin{center}
($n$ cardinalità dell'insieme, $k$ la lunghezza della disposizione)
\end{center}

\subsubsection{Esempio di Disposizione}
Poniamo caso di voler sapere le possibili di dispozioni normali e semplici di un dato insieme di lettere.
Per semplicità consideriamo l'insieme $S=\{c,a\}$, poniamo caso che vogliamo sapere tutte le possibili parole di lunghezza $2$.\\
Quindi $n = \#S = 2$ e $k = 2$, allora:

$$ D^{(r)}_{n,k} = n^k = 2^2 = 4 = \{(c,c),(a,a),(c,a),(a,c)\} $$
$$ D_{n,k} = \frac{n!}{(n-k)!} = \frac{2!}{0!} = 2! = 2 = \{(c,a), (a,c)\}$$ 



















\section{Lezione 03 - 21-09-2023}

\subsection{Algoritmo v3}
L'algoritmo può essere anche migliorato, riusciendo ad arrivare ad una complessità \textbf{lineare}, nel seguente modo:

\begin{lstlisting}[language=Python]
int Max_seq_sum_3(int N, array a[])
	maxsum = 0
	sum = 0
	for j=1 to N
		if (sum + a[j] > 0) then
			sum = sum + a[j]
		else
			sum = 0
		maxsum = max(maxsum,sum)
	return maxsum
\end{lstlisting}
Il ragionamento è il seguente:
Se prendiamo un insieme di numeri da sommare, (da i ad a), possiamo controllare se esso è positivo o negativo.
Nel caso in cui $\sum_{e=i}^{a} A[e]$ risultasse positiva, andiamo a espandere il nostro range fintantochè il risultato della sommatoria riamanga positivo.
Nel caso in cui invece il risultato fosse negativo, non ci conviene tenere traccia dei numeri più piccoli di quel range, dato che se quella sommatoria è minore del numero successivo alla sommatoria, non ha senso tenerne conto. E quindi invece ha senso tenere traccia del numero successivo. Da quel numero poi sommare i numeri successivi continuando il processo sopracitato.


\subsection{Strutture Dati - Insieme Dinamico}
Vediamo come rappresentare un insieme di dati dinamico $S$ (con insieme dinamico si intende una collezione di elementi variabile nel tempo, quindi è possibile aggiungere o rimuovere elementi);%Rubata da Simone Cerrone
$$ S=\{a_1,a_2,...,a_n\} \;\;\; n \ge 0  $$
Andiamo a definire alcune operazioni:
\begin{itemize}
\item Insert$(S,a) \rightarrow S^{\prime}$ ($ S^{\prime} = S \cup \{x\}$)
\item Deletes$(S,a) \rightarrow S^{\prime}$ ($S^{\prime} = S \backslash \{x\}$)
\item Search$(S,a) \rightarrow \{True,False\}$
\item Massimo$(S) \rightarrow a$
\item Minimo$(S) \rightarrow a$
\item Successore$(S,a) \rightarrow a^{\prime}$
\item Predecessore$(S,a) \rightarrow a^{\prime}$
\end{itemize}
\section{Lezione 04 - 17/03/2023}

\subsection{Spazi Vettoriali su R}
Sia $V$ un insieme non vuoto, definiamo due operazioni:
\begin{enumerate}
\item [] Interna $ +: VxV -> V $ (somma vettoriale) 
\item [] Esterna $ \cdot: RxV -> V $ (scalare per un vettore)  R è campo
\end{enumerate}
Posto $(V,+,\cdot)$ si dice \textbf{spazio vettoriale su R} $\Leftrightarrow$

\begin{enumerate}
\item $ (V,+) $ è un gruppo abeliano, quindi:
	\subitem Associatività
	\subitem Commutatività
	\subitem Neutro
	\subitem Tutti gli elementi invertibili
\item $ \forall \underline{v} \in V $ tale che $ \underline{v} \cdot 1 = \underline{v}$ (associtività mista)
\item $ \forall h,k \in R$, $\forall \underline{v} \in V$ tale che $ (hk)\underline{v} = h(k\underline{v})$
\item $ \forall h,k \in R$, $\forall \underline{v} \in V$ tale che $ (h+k)\cdot\underline{v} = h\cdot\underline{v} + k\cdot\underline{v}$ (distrub. tra $\cdot$ e $+$ in $R$)
\item $ \forall h,k \in R$, $\forall \underline{v} \in V$ tale che $ h(\underline{v}+\underline{w}) = h\cdot\underline{v} + h\cdot\underline{v}$  (distrub. tra $\cdot$ e $+$ in $V$)
\end{enumerate}

\subsection{Esempi Spazi Vettoriali}

\subsubsection{Spazio Vettoriale numerico di ordine n}
Verifichiamo che $(R^n,+,\cdot)$ sia uno spazio vettoriale, ma prima facciamo un esempio:
$$ (1,2,3) + (0,1,2) = (1,3,5) \:\:\:\:\:\:\: 3(3,2,4)=(9,6,12)$$
Andiamo a verificare che sia spazio vettoriale:
\begin{enumerate}
\item $ (R^n, +) $ gruppo abeliano:
	\subitem * Associatività e Commutatività banalmente eraditati da $+$
	\subitem * Neutro: $\underline{0} = (0,0,...,0)$
	\subitem * Inverso: $-(x_1,...,x_n) = (-x_1,-x_2,...,-x_n)$
\item Banale ereditatà di $\cdot$
\item $ (hk)(x_1,...,x_n) = (hkx_1,...,hkx_n) = h(kx_1,...,kx_n) = h(k(x_1,...,x_n))$
\item DA DIMOSTARE
\item DA DIMOSTARE
\end{enumerate}

\subsubsection{Spazio Vettoriale di una matrice di ordine m,n}
Possiamo considerare $(M_{m,n}(R), +, \cdot)$ come una lunga riga, quindi si accomuna al caso precedente.
\subsubsection{Spazio Vettoriale polinomiale}

\subsubsection{Spazio Vettoriale polinomiale di al più n}

\subsubsection{Spazio Vettoriale dei vettori geometrici in un punto O}



 




\section{Lezione 05 - 16-03-2023}

\subsection{Definizioni simboli Insiemestici ed Eventi}
\begin{center}

\begin{tabular}{ |c|c|c| } 
 \hline
 Begin & Algebra degli Insiemi & Logica degli Eventi \\ 
 $\Omega$ & Insieme universo & Spazio Campione \\ 
 $A$ & Insieme & Evento \\
 $A^C$ & Complementare di A & Negato di A \\
 $A \cup B$ & Unione di A e B & OR degli eventi, deve verificarli almeno uno tra A e B \\
 $ A \cap B $ & Intersezione tra A e B & AND degli eventi, devono verificarsi entrambi \\
 $ \bigcup_{k=1}^n A_k $ & Unione finita & n verifica almeno una tra $A_1, A_2,...,A_n$ \\
 $ \bigcup_{k=1}^{\infty} A_k $ & Unione numerabile & boh \\
 $ \bigcap_{k=1}^n A_k$ & Intersezione finita & Si verifica se tutti gli eventi $A_1,...,A_n$ si verificano \\
 $ \bigcap_{k=1}^{\infty} A_k$ & Unione numerabile & boh  \\
 $ \emptyset $ & Insieme Vuoto & Evento Impossibile \\
 $ A \cap B = \emptyset $ & A e B sono disgiunti & Eventi Incompatibili \\
 $ A \subset B $ & A contenuto in B & Il verificare di A implica il verificare di B \\
 $ \uplus_k A_k = \Omega $ & Ricomprimento disgiunto (partizione) & $A_1, A_2,...,A_n$ eventi neccessari\\
 \hline
\end{tabular}

\end{center}


\subsection{Esempio Lancio Moneta 1}
Poniamo caso che vogliamo descrivere l'evento che al terzo lancio di una moneta esca Testa, per prima cosa scegliamo un spazio campione:
$$ \Omega = {\{T,C\}}^{N} $$
Una moneta ha solo due casi, testa oppure croce, ora descriviamo che testa esca al terzo lancio:
$$ T_3 = \{ (w_1,w_2,...) \in \Omega: w_3 = "T" \} $$ 
abbiamo descritto questo eveno tramite propietà degli insiemi, nel caso volessimo esprimere lo stesso concetto ma per le croci ci basta fare il comlemento:
$$ T_3^C = C_3 = \{ (w_1,w_2,...) \in \Omega: w_3 = "C" \}  $$

\subsection{Esempio Lancio Moneta 2}
Poniamo invecere di voler complicare le cose, voglia esprimere l'evento che escano due Testa prima di due Croci, chiamiamo questo evento $A$, questo evento ha infinite possibilità, facciamo alcuni esempi:
$$ A_2 = T_1, T_2, \Omega $$
$$ A_3 = C_1, T_2, T_3, \Omega $$
$$ A_4 = T_1, C_2, T_3, T_4, \Omega $$
$$ A_5 = C_1, T_2, C_3, T_4, T_5 \Omega $$
Possiamo fare alcune osservazioni, $A_2, A_3$ sono incompatibili, non possono verificarci contemporaneamente, invece $A_5$ è incompatibile con $A_2,A_3,A_4$ .
Possiamo esprimere il verificarsi dell'evento $A$ in vari modi:\\
$$ A = A_2 \cup A_3 $$
$$ A = A_2 \cup A_3 \cup A_4 $$
$$ A = A_2 \uplus A_3  \uplus A_4 $$
$$ A = A_2 \uplus A_3 \uplus A_4 \uplus A_5 $$
Possiamo esprimere questo evento $A$ tramite \textbf{Unione Numberabile}:
\begin{equation*}
A_n =
\begin{cases}
C_1,T_2,...,C_{n-2},T_{n-1},T_n  \:\:\:\:\:\text{n dispari inizia con una croce}\\
T_1,C_2,...,C_{n-2},T_{n-1},T_n  \:\:\:\:\:\text{n pari inizia con una testa}
\end{cases}
\Rightarrow A = \bigcup_{n=2}^{\infty} \:\:\:\:\: \text{è un evento}
\end{equation*}

\subsection{Famiglia/Classi}
\blindtext

\subsection{Algebra e Sigma Algebra}
Preso un $\Omega$ spazio campione e un $a$ (a tondo), classe non vuota di sottinsiemi di $\Omega$ allora:
$$ a \: \text{è un algebra} \Leftrightarrow $$
$$ i) A \in a \Rightarrow A^C \in a \:\:\: \text{(a è chiusa rispetto il complemento)} $$
$$ ii) A_1, A_2 \in a \Rightarrow A_1 \cup A_2 \in a \:\:\: \text{(a è chiusa rispetto l'unione di due elementi)} $$
C'è un anche una sua variante chiamanta Sigma(numerabile) Algebra che si definisci così:
$$ \sigma-algebra \Leftrightarrow $$
$$ i)uguale $$
$$ ii) n \in N, A_n \in a \Rightarrow \bigcup_{n=1}^{\infty} A_n \in a$$
\subsubsection{Osservazioni}
Posto $a=\{\{2,3\}, \{6\}, \{4,5\}\}$, osserviamo i seguenti esempi:
\begin{center}
$ \{4,5\} \subseteq a $ SBAGLIATO\\
$ \{4,5\} \in a $ CORRETTO\\
$ \{\{4,5\}\} \subseteq a $ CORRETTO
\end{center}

\subsubsection{Casi Particolari}
Poniamo $A \subseteq \Omega$, si definisce \textbf{algebra(sigma) banale}, $a$ posto come:
$$ a = \{\emptyset, \Omega \}$$
È l'unica algebra a due elementi, ovviamente entrambe le propietà sono banalmente dimostrate poiché:
$$ \Omega^C = \emptyset $$
$$ \Omega \cup \emptyset = \Omega \in a $$
Gli elementi $\emptyset$ e $\Omega$ sono neccessari per essere un \textbf{algebra}.
Poniamo caso di un $a=\{A, A^c\}$ questa non è un algebra poiché $A \cup A^c = \Omega \not \in a$, se aggiunssimo solo $\Omega$ non sarebbe rispettata la prima condizione poiché $ \Omega^c = \emptyset \not \in a$.\\
Ricapitolando:
$$ a=\{A, A^C\} \:\:\: \textbf{non è algebra} \:\:\:\:\: a=\{A,A^C,\emptyset,\Omega\} \:\:\:\:\:\textbf{è algebra (sigma)}$$
Per contrapposizione la (sigma) algebra più grande è $P(\Omega)$, tutte le altre algebra(sigma) sono sottoinsiemi di $P(\Omega)$

\subsection{Propietà (conseguenze)}
\begin{enumerate}
\item $a$ è una algebra (sigma) $\Rightarrow \: \emptyset,\Omega \in \: a$ (come abbiamo osservato prima)\\Tutti gli elementi dell'algebra banale devono essere presenti in ogni algebra(sigma).
\item L'unione finita di elementi di un algebra (sigma) appartiene comunque ad $a$\\ Per $ii)$ abbiamo visto come l'unione si applica per due elementi, ma essendo $\cup$ associativa nel caso di $n-elementi$ basta operarli a due a due e quindi portare questa propietà fino a n elementi.
\item $ Sigma\:algebra \Rightarrow Algebra \:\:MA\:\: Sigma\: algebra \not \Leftarrow Algebra $\\
Questo poiché un unione finita da 0 a $+\infty$ non appartiene a tutte le algebra, cose che invece accade con le sigma algebra.
\end{enumerate}








\section{Lezione 06 - 24-03-2023}

\subsection{Propietà Sottospazio Vettoriale}

\begin{itemize}
\item[$W \underline{<} V$ è stabile rispetto alla somma di $n$ oggetti]
Siano $\underline{w}_1, ... , \underline{w}_n \in W$ si ha $w_1 + w_2 \in W \Rightarrow$ 
\item[Famiglia di sottospazi vettoriali]
Sia $ \mathbb{L}$ una famiglia di sottospazi di $V$, l'intersezione dei sottospazi della famiglia $\mathbb{L}$ è un sottospazio e si indica: 
$$ \bigcap_{L \in \mathbb{L}} L $$
L'intersezione di una qualunque famiglia di sottospazi è un sottospazio.\\
Dimostriamolo: 
	\subitem Neutro: Il neutro è un elemento comune, quindi è sempre contenuto.
	\subitem Stab $+$: Siano $ \underline{v},\underline{w} \in \bigcap_{L \in \mathbb{L}} L \Rightarrow \forall L \in \mathbb{L} \Rightarrow \underline{v},\underline{w} \in L \Rightarrow \underline{v}+\underline{w} \in \bigcap_{L \in \mathbb{L}} L $
	\subitem Stab $\cdot$: Siano $ \underline{v} \in \bigcap_{L \in \mathbb{L}} L, h \in \mathbb{R} \Rightarrow \forall L \in \mathbb{L} \Rightarrow \underline{v},\underline{w} \in L \Rightarrow \underline{v}+\underline{w} \in \bigcap_{L \in \mathbb{L}} L $
\end{itemize}

\subsection{Sottospazio Generato}
Sia $ S \subseteq V $, indicheremo con $<S>$ il \textbf{sotto spazio generato da S}.\\
$$ <S> = \bigcap_{L \in \mathbb{L}_s} L$$
In altri termini: è il più piccolo sottospazio rispetto all'intersezione.

\subsubsection{Esempi:}
\begin{itemize}
\item[•] $ <H> = H $ SEMPRE!
\item[•] $ <\{\underline{0}\} = \{\underline{0}\} $
\item[•] $ <V> = V $ 
\item[•] $ <\emptyset> = {0} $ Singleton dell'elemento neutro, poiché appartiene ad ogni elemento.
\end{itemize}

$ S= H \cup K $ con $H,K \underline{<} V$
$$ <H \cup K> = H+K = \{\underline{h}+\underline{k} / \underline{h} \in H, \underline{k} \in K \} $$
Dim:
Come sempre per dimostrare l'uguaglianza dobbiamo dimostare la doppia inclusione:
$$ <H \cup K> \subseteq \textbf{al contrario } H+K = \{\underline{h}+\underline{k} / \underline{h} \in H, \underline{k} \in K \} $$
non ho capito\\

Dimostriamo che sia spazio vettoriale:
\begin{itemize}
\item[Neutro] $$ \underline{0} = \underline{0}^{\text{preso da H}} + \underline{0}^{\text{preso da K}} $$
\item[Stabile $+$] $$ (\underline{h} + \underline{k}) + (\underline{h}^' + \underline{k}^') \in H+K $$
$$ (h+h^') + (k+k^') $$
\item[Stabile $\cdot$] $$ \alpha(\underline{h}+\underline{k}) = \alpha\underline{h} + \alpha \underline{k}  $$
\end{itemize}

\subsubsection{Esempio}
$$ H=\{(0,k) / x \in \mathbb{R}\} \;\;\; K=\{(y,0) / y \in \mathbb{R} \} $$
$$ <H \cup K> = H+K = (0+y, x+0) = \mathbb{R}^2 $$

\subsection{Propietà Sottospazio Generato}
Posto $H,K \underline{<} V$, allora valgono le seguenti propietà:
\begin{itemize}
\item[•] $ H \quad K = H \cap K = \{ \underline{0} \} \;(\text{neutro}) $ Si dicono in somma diretta.
\item[•] $ H + K = V$ allora $H,K$ si dicono supplementari
\item[•] $ H \quad K = V$ allora si dicono complementari (in altri termini devono essere in somma diretta e supplementari).\footnote{È un concetto un po' strano, perché vuol dire somma normale (quindi caso 2), ma ricandoci che l'intersezione da il neutro (quindi caso 1)}
\end{itemize}

\subsubsection{Esempio}
Posti $\{ \underline{0} \}$ e $V$:
\begin{itemize}
\item[Somma diretta]: $ \{ \underline{0} \} \quad V = \{ \underline{0} \} \cap V = \{ \underline{0} \}$
\item[Supplementari]: $  \{ \underline{0} \} + V = V$
\item[Complementare]: Dato che è sia somma diretta che supplementare
\end{itemize}

\subsection{Dipendenza/Indipendenza Lineare}
Sia $V$ uno spazio e vettore e siano $\underline{v}_1, \underline{v}_2, ..., \underline{v} \in V$, sono detti \textbf{linearmente dipendenti} (o legati) $\Leftrightarrow$
$$ \exists \alpha_1, \alpha_2, ..., \alpha_n \neq (0,0,...,0) = \alpha_1\underline{v_1} + ... + \alpha_n+\underline{v}_n = \underline{0}$$
La loro combinazione lineare deve essere il vettore nullo.
Se tali scalari non esistono allora si dice che sono \textbf{linearmente indipendenti} (o liberi), l'unica soluzione valida è quella formata da tutti zero: $0\underline{v}_1+...+0\underline{v}_n = \underline{0}$.
$$ \textbf{Se non sono dipendenti} \Rightarrow \textbf{Sono indipendenti}$$

\subsubsection{Esempio}

\subsection{Propietà dipendenza lineare}
\begin{itemize}
\item[1)] $\underline{0}$ dipende smpre da qualunque sistema
$$ \underline{0} = 0\underline{v}_1+...+0\underline{v}_n $$
\item[2)] $ \forall i \underline{v}_i $ dipende da $ \underline{w}_1,...,\underline{w}_n$ \footnote{Una specie di transitività della dipendenza}
$$  $$
\end{itemize}










\end{document}