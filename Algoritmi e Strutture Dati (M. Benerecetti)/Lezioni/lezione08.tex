\section{Lezione 08 - 03/10/2023}
\subsection{Nomi Utilizzati delle visite}
Vi sono due tipo di visita che analizziamo per gli alberi:
\begin{itemize}
	\item \textbf{DFS} o visite in \textbf{profondità} sono i tipi di visite che, in generale, tendono a visitare prima i figlio dei padri, si sviluppano "in verticale". Sono intrinsecamente ricorsivi.
	\item \textbf{BFS} o visite in \textbf{ampiezza} sono i tipi di visite che tendono ad andare per livelli, visitando i nodi che sono tra loro "fratelli" e "cugini".
\end{itemize}

\subsection{Visita In Ampiezza Albero}
Come già accennato, la visita in ampiezza di un albero prevede la visita dei nodi in modo "orizzontale". Questa visita di prim'occhio potrebbe sembrare piu complessa, poiche rispetto alle altre che abbiamo visto fino ad ora non è possibile effettuarla con un algoritmo ricorsivo/iterativo e basta. Infatti, per questa visita sarà necessaria una struttura dati astratta già studiata in precedenza: \textbf{la coda}.
Iniziando la visita dalla radice, inseriremo i figli del nodo corrente all'interno della coda (se presenti), per poi proseguire la visita con un nodo estratto dalla coda, che diventa così il nodo corrente, e così via, fin quando la coda non sarà vuota.

\begin{lstlisting}[language=Java]
	BFS(T):
	Q=NIL
	Q=Accoda(Q,T)
	while(Q!=NIL) DO
		X=Testa(Q)
		"Visita del nodo"
		Q=Accoda(Q,X->sx)
		Q=Accoda(Q,X->dx)
		Q=Decode(Q) //Toglie della coda il primo	
\end{lstlisting}

Ma quanto può essere complesso fare una cosa del genere? Le operazioni all'interno e prima del while sono a tempo costante, poichè sono operazioni di assegnazione. Il numero di volte in cui viene eseguito il while, però, dipende dalla quantità di nodi all'interno dell'albero, dunque è \textbf{lineare sul numero di nodi}.
Per quanto riguarda la memoria occupata dalla coda, invece:
\begin{itemize}
	\item Nel caso migliore l'albero è degenere (ogni nodo ha un solo figlio ad eccezione della foglia), nella coda sarà presente al più un nodo per volta, quindi è un costo di memoria \textbf{costante}.
	\item Nel caso migliore l'albero è bilanciato (ogni nodo ha 2 figli ad eccezione delle foglie), il momento in cui vi sarà più memoria utilizzata sarà quando tutte le foglie si troveranno in coda: circa n/2 nodi. Cioè nel caso più sfortunato si occuperà memoria \textbf{lineare sul numero di nodi}.   
\end{itemize}

\subsection{Albero Binario di Ricerca}
E' un tipo di albero ordinato. Preso qualsiasi nodo, si ha che il suo dato è:
\begin{itemize}
	\item Maggiore o uguale a tutti i dati contenuti nei nodi del suo sottoalbero sinistro
	\item Minore o uguale a tutti i dati contenuti nei nodi del suo sottoalbero destro
\end{itemize} 
Il suo acronimo è \textbf{BST} (Binary Search Tree). L'algoritmo di ricerca per questo albero è uno dei piu efficienti.

\subsubsection{Search Ricorsiva - BST}

\begin{lstlisting}[language=Java]
	SearchBSTr(T,k)
	ret=T
	if T!= NIL then
		if T->key < k then
			ret = SearchBSTr(T->dx, k)
		else if T->key > k then
			ret = SearchBSTr(T-sx, k)
	return ris	
\end{lstlisting}

Questo algoritmo sicuramente ci permetterà di cercare in modo piu efficiente un dato, poichè, un po' come la ricerca binaria (ma non propriamente), ci permette di andare a "dimezzare" l'insieme di ricerca ad ogni confronto.\\

Esiste anche la versione iterativa di questo algoritmo:

\begin{lstlisting}[language=Java]
	SearchBSTi(T,k)
	Tmp = T
	while Tmp != NIl && Tmp-> key != k do
		if Tmp->key < k then
			Tmp=Tmp->dx
		else
			Tmp=Tmp->sx
	return Tmp	
\end{lstlisting}

Proprio come l'algoritmo ricorsivo andremo a dividere l'albero in due, ma ci avvarremo di un puntatore temporaneo per risalire nell'albero (questo puntatore viene memorizzato implicitamente nella versione ricosrsiva all'interno dello stack di attivazione).

\subsubsection{Min - BST}
La ricerca del minimo in un albero binario di ricerca è molto semplice. Seguendo la regola secondo cui "l'elemento piu piccolo si trova a sinistra", andremo a scendere a sinistra finchè sarà possibile, l'elemento che non ha figlio sinistro conterrà il dato più piccolo dell'albero.

\begin{lstlisting}[language=Java]
	MinR(T)
	ret = T
	if ret->sx != NIL then
		ret= MinR(ret->sx)
	return ret
\end{lstlisting}

\subsubsection{Algoritmo del Successore Ricorsivo - BST}
Il successore di un numero è il primo numero piu grande dello stesso. Non è detto che un successore sia presente all'interno dell'albero (caso in cui tutti i valori nell'albero sono minori o uguali al valore di cui si cerca il successore) L'algoritmo diventa un po' piu complesso: vi sono le 3 casistiche:
\begin{itemize}
	\item Caso \textbf{T->key = k}. L'elemento successore si troverà sicuramente nel sottoalbero destro, poichè cerchiamo un valore più grande. In particolare, è il piu piccolo elemento del sottoalbero destro, quindi \textbf{Min(T->dx)}
	\item Caso \textbf{T->key < k}. L'elemento successore si troverà sicuramente nel sottoalbero destro, poichè cerchiamo un valore più grande. Non sappiamo nient'altro però, quindi eseguiamo l'algoritmo sulla radice del sottoalbero destro: \textbf{Succ(T->dx)}.
	\item Caso \textbf{T->key > k}. L'elemento successore potrebbere trovarsi nel sottoalbero sinistro, poichè il valore corrente è più grande, quindi eseguiamo l'algoritmo sulla radice del sottoalberso sinistro \textbf{Succ(T->sx, k)}, se la ricerca non dovesse andare a buon fine, allora il vaore corrente T è il successore.
\end{itemize}

\begin{lstlisting}[language=Java]
	SuccR(T,k)
	ret = NIL
	if ret != NIL then
		if ret->key = k then
			ret = Min(ret->dx)
		else if ret->key < k then
			ret = SuccR(ret->dx,k)
		else 
			ret = SuccR(T->sx,k)
			if ret= NIL then
				ret = T
	return ret
\end{lstlisting}

\subsubsection{Algoritmo del Successore Iterativo - BST}

Per questo tipo di algoritmo, dobbiamo ragionare in modo diverso.
E' necessario fare controlli che nella versione ricorsiva vengono omessi poichè queste informazioni vengono ricavate dal punto di ripresa dell'esecuzione di ogni chiamata. 
Bisogna fare particolare attenzione al caso in cui il valore di cui vogliamo il successore non ha figli destri, poichè è necessario risalire fino a trovare un padre "da sinistra" (la terza casistica).
Nell'algoritmo il valore tmp viene usato per visitare l'albero. Quando si scende a sinistra, viene salvato prima il nodo corrente nella variabile ret, poichè se non si trova il successore in quel sottoalbero sinistro, allora è il nodo salvato in ret il successore.

\begin{lstlisting}[language=Java]
	SuccI(T,k)
	Tmp = T
	ret = NIL
	while Tmp != NIL andd TMP->key != k then
		if Tmp->key < k then
			Tmp = Tmp->dx
		else 
			ret = Tmp
			Tmp = Tmp->sx
	if Tmp != NIL && Tmp->dx != NIL then
		ret = Min(Tmp->dx)
	return ret
\end{lstlisting}
