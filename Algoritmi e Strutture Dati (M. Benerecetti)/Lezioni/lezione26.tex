\section{Lezione 26 - 21/11/2023}
\subsection{TODO}

\subsection{Algoritmi visita Grafi Pesati}
Dato un grafo pesato tramite un Algoritmo di visita BFS possiamo andare a calcolare la stima del peso di un percorso $\pi$, ne possiamo distinguere due:
\paragraph{Algoritmo di Dijkstra} è usato solo per grafi con pesi non negativi.
\paragraph{Algoritmo Bellman-Ford} per grafi pesati arbitrari.
Entrambi sono basati sul concetto di "rilassamento" degli archi.

\subsection{Rilassamento}
Il rilassamento è una tecnica che consente di stimare il percorso con peso minore.\smallskip

Consideriamo la funzione:
$$ d: V \rightarrow \mathbb{R} $$
che ad ogni vertice viene associato un reale che corrisponde a una stima del peso.
\begin{lstlisting}[language=Java]
relax(u,v,w):
    if d[v] > d[u] + w(u,v) then //d: distanza, w: peso
        d[v]=d[u]+w(u,v)
        pred[v]=u
\end{lstlisting}
Se la nuova stima è migliore aggiorno, ovviamente esistono casi in cui la stima è migliore anche se bisogna percorrere più archi.

\subsection{Propietà Percorsi Minini su Grafi Pesati}
\subsubsection{Lemma 1}
Dato $G=(V,E,w)$ e $\pi=v_1v_2...v_{k-1}v_k$ un percorso minimo da $v_1$ a $v_k$
$$ \forall 1 \le i \le j \le k, \pi_{ij}=v_iv_{i+1}...v_j$$
$$ \pi_{ij} \text{ è un percorso minimale da } v_i \text{ a } v_j$$

\paragraph{Dimostazione}