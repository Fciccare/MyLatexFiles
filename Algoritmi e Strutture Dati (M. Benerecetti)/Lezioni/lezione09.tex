\subsection{Lezione 9 - 05/10/2023}

\subsubsection{Successore Iterativo - BST}

Per questo tipo di algoritmo, dobbiamo ragionare in modo diverso. In questo caso non iterativo non possiamo permetterci di omettere determinati controlli a posteriori. In particolare il controllo nel caso in cui il valore di cui vogliamo il successore non ha figli destri ed e una foglia. In questo caso particolare non abbiamo la possibilita di risalire a ritroso ricorsivamente ma dobbiamo tenere traccia ogni volta che il nodo scende a sinistra, segnandoci il puntatore di quest'ultimo.

\begin{lstlisting}[language=Java]
	SuccI(T,k)
	Tmp = T
	ret = NIL
	while Tmp != NIL andd TMP->key != k then
	if Tmp->key < k then
	Tmp = Tmp->dx
	else 
	ret = Tmp
	Tmp = Tmp->sx
	if Tmp != NIL && Tmp->dx != NIL then
	ret = Min(Tmp->dx)
	return ret
\end{lstlisting}


\subsubsection{Predecessore Ricorsivo - BST}
L'algoritmo del predecessore e simile al successore strutturalmente parlando ma invertendo segni e qualche operazione

\begin{lstlisting}[language=Java]
	PredR(T,k)
		ret = NIL
			if ret != NIL then
				if ret->key = k then
					ret = Max(ret->sx)
				else if ret->key < k then
					ret = PredR(ret->dx,k)
				else 
					ret = PredR(T->sx,k)
					if ret = NIL then
						ret = T
		return ret
\end{lstlisting}

\subsubsection{Insert Ricorsiva - BST}

L'algoritmo dell'inserimento in un albero in un albero binario di ricerca puo vantare del fatto che e piu facile trovare il nodo nel quale si puo aggiungere il valore che abbiamo in input alla funzione. Ci bastera semplicemente scorrere a destra o a sinistra il nostro puntatore per poi arrivare nel primo punto NIL favorevole e "returnare" a cascata i puntatori dei padri.

\begin{lstlisting}[language=Java]
	InsertR(T,k)
	ret = T
		if T = NIL then
			ret = new_node(k)
		else if T->key < k then
			T->dx = InsertR(T->dx,k)
		else if T->key > k then
			T->sx = InsertR(T->sx,k)
	return ret
\end{lstlisting}

\subsubsection{New Node - Generico}
La funzione new node va a creare un nuovo nodo dinamico all'interno dell'albero.

\begin{lstlisting}[language=Java]
	new_node(k)
		ret = alloca_nodo() //andiamo a restituire il puntatore a nuovo nodo allocato in memoria a ret
		ret->key = k
		ret->sx = NIL
		ret->dx = NIL
		return ret
\end{lstlisting}

\subsubsection{Insert Iterativa - BST}
Versione iterativa della insert prevede dei controlli in piu per quanto riguarda la ricerca e inserimento. In questo caso specifico abbiamo bisogno di un puntatore in piu che ci segue nello scorrimento, chiamato \textbf{P} e sta a indicare il Padre del nodo a cui stiamo scorrendo.

\begin{lstlisting}[language=Java]
	InsertI(T,k)
	ret = T
	P = NIL
	Tmp = T
	while Tmp != NIL && Tmp->key != k do
		P = Tmp
		if Tmp->key < k then
			Tmp = Tmp->dx
		else
			Tmp = Tmp->sx
	
	If Tmp = NIL then
		x = new_node(k)
		if P->key < k then
			P->dx = x
		else
			P->sx = x
	else 
		P-sx = x
	return ret
\end{lstlisting}

\subsubsection{DeleteR - BST}
La delete prevede la delete del nodo e la restituzione dell'albero con quel nodo mancante. A primo acchitto non sembra un'operazione cosi difficile ma dobbiamo come sempre andare a ragionare per casi.

\begin{itemize}
	\item Caso albero vuoto. In questo caso dobbiamo semplicemente restituire T, il puntatore (vuoto) alla radice dell'alberoe
	\item Caso albero non vuoto. In questo caso la radice del sottoalbero ha un sottoalbero destro e sinistro.
		\begin{itemize}
			\item Se T->key < k then Delete(T->dx,k)
			\item Se T->key > k then Delete(T->sx,k)
			\item Se T->key = k, dobbiamo distinguere dei casi
			\begin{itemize}
				\item Nel caso in cui il nodo non ha figli allora, si puo procedere all'eliminazione del nodo
				\item Il caso in cui il nodo o ha figlio destro o figlio sinistro e basta.
				\item Il caso in cui il nodo ha entrambi i figli collegati. In questo caso specifico deleghiamo la distruzione del nodo a un'altra funzione chiamata \textbf{Stacca(T->dx,k)}.
			\end{itemize}
		\end{itemize}
\end{itemize}

In particolare questa funzione sara formata in questo modo:

\begin{lstlisting}[language=Java]
	StaccaMin(T,P)
		ret = T
		If T != NIL then
			ret = StaccaMin(T->sx,P)
			if ret = NIL then
				if P != NIL then
					if P->sx = T then
						P->sx = T->dx
					else
						P->dx = T->dx
		return ret
\end{lstlisting}


