\subsection{Lezione 9 - 05/10/2023}

\subsubsection{Successore Iterativo - BST}

Per questo tipo di algoritmo, dobbiamo ragionare in modo diverso. In questo caso non iterativo non possiamo permetterci di omettere determinati controlli a posteriori. In particolare il controllo nel caso in cui il valore di cui vogliamo il successore non ha figli destri ed e una foglia. In questo caso particolare non abbiamo la possibilita di risalire a ritroso ricorsivamente ma dobbiamo tenere traccia ogni volta che il nodo scende a sinistra, segnandoci il puntatore di quest'ultimo.

\begin{lstlisting}[language=Java]
	SuccI(T,k)
	Tmp = T
	ret = NIL
	while Tmp != NIL andd TMP->key != k then
	if Tmp->key < k then
	Tmp = Tmp->dx
	else 
	ret = Tmp
	Tmp = Tmp->sx
	if Tmp != NIL && Tmp->dx != NIL then
	ret = Min(Tmp->dx)
	return ret
\end{lstlisting}

\subsubsection{Predecessore Ricorsivo - BST}
TODO

\subsubsection{Insert Ricorsiva - BST}

L'algoritmo dell'inserimento in un albero in un albero binario di ricerca puo vantare del fatto che e piu facile trovare il nodo nel quale si puo aggiungere il valore che abbiamo in input alla funzione. Ci bastera semplicemente scorrere a destra o a sinistra il nostro puntatore per poi arrivare nel primo punto NIL favorevole e "returnare" a cascata i puntatori dei padri.

\subsubsection{Algoritmo del Successore Iterativo - BST}

Per questo tipo di algoritmo, dobbiamo ragionare in modo diverso. In questo caso non iterativo non possiamo permetterci di omettere determinati controlli a posteriori. In particolare il controllo nel caso in cui il valore di cui vogliamo il successore non ha figli destri ed e una foglia. In questo caso particolare non abbiamo la possibilita di risalire a ritroso ricorsivamente ma dobbiamo tenere traccia ogni volta che il nodo scende a sinistra, segnandoci il puntatore di quest'ultimo.

\begin{lstlisting}[language=Java]
	InsertR(T,k)
	ret = T
	if T = NIL then
		ret = new_node(k)
	else if T->key < k then
		T->dx = InsertR(T->dx,k)
	else if T->key > k then
		T->sx = InsertR(T->sx,k)
	return ret
\end{lstlisting}

\subsubsection{New Node - Generico}

