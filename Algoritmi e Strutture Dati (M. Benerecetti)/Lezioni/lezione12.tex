\section{Lezione 12 12/10/2023}
\subsection{Cancellazione albero AVL}

Per la cancellazione di un elemento nell'albero AVL dobbiamo sempre fare delle considerazioni simili a quelle dell'aggiunta di un elemento nell'AVL.
\medskip

Nel caso specifico pero c'e bisogno di considerare che la DeleteAVL del nodo.
Nella diminuzione di un elemento K del sottoalbero, \textbf{il padre} dell'elemnto K puo essere attaccato facilmente a uno dei due sottoalberi di sinistra o destra.
Nel farlo andiamo a diminuire di pesantezza uno dei due sottoalberi rendendo il successivo piu pesante. In quel caso la strategia che potremmo attuare e quella del ribilanciamento, ma condizionato dal fatto che ci troviamo in un AVL.\\

I tre algoritmi sono concettualmente gli stessi che usiamo per il bilanciamento di un BST.

\subsubsection{DeleteAVL}
Questa funzione e pressoche simile a quella vista negli Alberi binari di ricerca con l'aggiunta del bilanciamento.

\begin{lstlisting}[language=Java]
	DeleteAVL(T,k)
		if T != NIL then
			if T->key > k then
				T->sx = DeleteAVL(T->sx,k)
				T = BilanciaDx(T)
			else if T->key < k then
				T->dx = DeleteAVL(T->dx,k)
				T = BilanciaSx(T)
			else 
				T = DeleteRootAVL(T)
		return T
\end{lstlisting}

\subsubsection{DeleterootAVL}
La delete Root, chiamata quando viene trovato il nodo da eliminare, non solo copiera il valore del piu piccolo elemento del sottoalbero destro di T, ma andra a bilanciare nuovamente a sinistra (perche piu pesante) il sottoalbero di T.

	\begin{lstlisting}[language=Java]
		DeleteRootAVL(T)
			if T != NIL then
				tmp = T
				if T->sx = NIL then
					T = T->dx
				else if T->dx = NIL then
					T = T->sx
				else
					tmp = stacca\_minAVL(T->dx,T)
					T->key = tmp->key
					T = BilanciaSx(T)
					dealloca(tmp)
			return T
	\end{lstlisting}

\subsubsection{Stacca\_min}
Per lo staccamin abbiamo bisogno di una variabile in piu in questo caso perche non ci e possibile andare a salvare l'elemento da cancellare dopo il bilanciamento, che si perderebbe nell'albero. In tal caso l'uso di una variabile e il controllo successivo se si trova nel nodo figlio di destra o sinistra ci verra in aiuto.

\begin{lstlisting}[language=Java]
	Stacca\_minAVL(T,P)
		if T != NIL then
			If T->sx != NIL then
				ret = Stacca\_minAVL(T->sx, T)
				newt = BilanciaDx(T)
			else
				ret = T
				newt = T->dx
			if T = P->sx then
				P->sx = newt
			else
				P->dx = newt
		return ret
\end{lstlisting}

Le operazioni di cancellazione andranno a effettuare una operazione di diminuizione dell'altezza di un albero e conseguentemente un'altra a cascata per il bilanciamento, che solitamente cambia le dimensioni dell'albero finale. Questo solitamente non ci dara problemi, ma solo in caso di alberi AVL ci dara problemi, poiche ci fara perdere le proprieta di quest'ultimo.
\medskip
In un esempio di AVL minimo la cancellazione di un nodo nel sottoalbero meno pesante andrebbe a comportare un ribilanciamento che va a peggiorare la differenza delle altezze tra alberi, portandola a 2, e qundi perdendo la proprieta di albero.


