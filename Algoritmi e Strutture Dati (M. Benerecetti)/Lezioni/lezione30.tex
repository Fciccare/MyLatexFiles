\section{Lezione 30 - 30/11/2023}
\subsection{Insertion Sort}

\begin{lstlisting}[language=Java]
InsertionSort(A, N): //A: array, N: dimensione max array 
    //primo valore gia' "ordinato", partiamo dal secondo
    for j=2 to N do
        i=j-1 //valore precedente
        x=A[j] //valore "corrente"
        //se il valore precedente e' maggiore del corrente
        while (i>0 AND A[i]>x) do 
            A[i+1]=A[i] //nella cella corrente metto il valore prec.
            i=i-1 //mettiamo i al prec. prec.
        A[i+1]=x
\end{lstlisting}


\subsubsection{Caso Peggiore}
Il caso peggiore è quando la sequenza è totalmente in disordine cioé quando i valori sono in ordine decrescente, in questo caso $t_j=j$ e quindi avremo:
$$ 4n-3+2\sum_{j=2}^n (j+1) + \sum_{j=2}^n j $$
Andiamo a spezzare la prima sommatoria:
$$ 4n-3+2\sum_{j=2}^n j - 2\sum_{j=2}^n 1 + \sum_{j=2}^n j $$
Accorpiamo la prima e terza sommatoria
$$ 4n-3+3\sum_{j=2}^n j - 2\sum_{j=2}^n 1 $$
La prima sommatoria è la somma dei prima n numeri naturali (formula di gauss) meno uno perché partiamo da 2, la seconda è banalemente n tranne uno sempre perché cominciamo da due:
$$ 4n -3 +3(\frac{n(n+1)}{2}-1) - 2(n-1) = \theta(n^2) $$
Nel caso peggiore è \textbf{quadratica}

\subsubsection{Caso Medio}
Nel caso medio andiamo a fare appunto la media dei possibili valori di $t_j=1,2,...,j$:
$$ \frac{\sum_{i=1}^j i}{j} = \frac{1}{\cancel{j}}*\frac{\cancel{j}(j+1)}{2} = \frac{j+1}{2} $$

Ora che sappiamo il valore di $j$ andiamo a sostituirlo qua:
$$ 4n-3+2\sum_{j=1}^n (j-1) + \sum_{j=2}^n j $$
Quindi avremo:
$$ 4n-3+2\sum_{j=1}^n (\equalto{\frac{j+1}{2}-1}{\frac{j-1}{2}}) + \sum_{j=2}^n \frac{j+1}{2} $$
Andiamo a spezzare le sommatorie:
$$ 4n-3+2\sum_{j=2}^{n} \frac{j}{2} - 2 \sum_{j=2}^n \frac{1}{2} + \sum_{j=2}^{n} \frac{j}{2} + \sum_{j=2}^{n} \frac{1}{2}$$max
Sommiamo le sommatorie simili:
$$ 4n-3+3\sum_{j=2}^{n} \frac{j}{2} - \sum_{j=2}^n \frac{1}{2} $$
Andiamo a sviluppare le due sommatorie, per $\frac{j}{2}$ possiamo portale il mezzo fuori dalla sommatoria dato che non dipende da $j$:
$$ 4n-3+\frac{3}{2}\sum_{j=2}^{n} j - \sum_{j=2}^n \frac{1}{2} $$
La prima sommatoria è sempre la formula di gauss meno uno, invece la seconda è sempre n meno uno ma diviso due:
$$ 4n-3+\frac{3}{2}(\frac{n(n+1)}{2} -1) - \frac{n-1}{2} = \theta(n^2)$$
Il caso medio è \textbf{quadratico}

