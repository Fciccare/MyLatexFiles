\section{Lezione 21 - 02/11/2023}

\subsection{Visita in Profondità - DFS}
La Depth-First Search (DFS) è un algoritmo di esplorazione dei grafi il cui funzionamento consiste nel visitare prima i nodo più lontani dal nodo sorgente e poi tornando progressivamente verso il punto di partenza. 
\smallskip

Rispetto alla BFS vista in precedenza, la DFS viene impiegata per scopi diversi. Mentre la BFS è utilizzata per scoprire percorsi minimi, la DFS può essere usato per verifica la ciclicità di un Grafo.
\smallskip
Come per la BFS useremo di "strutture ausiliarie":
\begin{itemize}
    \item Col[ ]: $V \rightarrow \{b,g,n\}$ (array dei colori)
    \item Pred[ ]: $V \rightarrow V$ (array dei predec., cioè quale vertice è stato scoperto da chi)
    \item d[ ]: $V \Rightarrow \mathbb{N}[\{1,2,...,2n\}]$ (il tempo in cui il vertice è stato scoperto)
    \item f[ ]: $V \Rightarrow \mathbb{N}[\{1,2,...,2n\}]$ (il tempo in cui il vertice è stato visitato)
\end{itemize}
\smallskip

Dato che per la DFS vogliamo visitare tutti i vertici di un grafo, dato che non sempre un vertice sorgente ci permette di farlo, andiamo a definire due funzioni:

\begin{lstlisting}[language=Java]
Init(G) //G:grafo
    for each v in V do //ogni vertice del Grafo
        col[v]=b  
        prev[v]=NIL
    tempo=1 //variabile globale
\end{lstlisting}
La funzione DFS ci permette di far ripartire la visita da un nuovo vertice sorgente (bianco) in modo da poter esplorare ogni parte del grafo. 
\begin{lstlisting}[language=Java]
DFS(G) //G:grafo
    Init(G)
    for each v in V do //ogni vertice del Grafo
        if col[v]=b then 
            DFS_Visit(G,v)
\end{lstlisting}
Ogni volta che troviamo un nuovo nodo (bianco) andiamo a richiamare ricorsivamente la visita in modo da scendere in Profondità nel grafo e poi successivamente "risalire".
\begin{lstlisting}[language=Java]
DFS_Visit(G, s) //G:grafo, s: vertice sorgente
    col[s]=g //coloriamo di grigio 
    d[s]=tempo++ //scoperto a tempo corrente (+1)
    for each v in ADJ[s] do
        if col[v]=b then
            pred[v]=s
            DFS_Visit(G,v) //visita ricorsiva
    f[s]=tempo++ //visitato a tempo corrente (+1)
    col[s]=n
\end{lstlisting}

\paragraph{Costo} Come visto per la BFS possiamo anche qua ogni nodo viene visatato una solo volta quindi il costo sarà sempre |V|+|E|=|G|
