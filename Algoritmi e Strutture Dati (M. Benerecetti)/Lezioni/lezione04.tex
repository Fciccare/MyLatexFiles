\section{Lezione 04 - 23/09/2023}
\subsection{Array non ordinato}
Abbiamo un insieme $S$ che vogliamo rappresentare in $A$.\\
$$ |S| = |A| $$
La funzione $A.free$ ci restituirà la posizione della prima cella libera.\\


\subsubsection{Ricerca}
Dato un array e un elemento da cercare, restituisce la posizione in cui è presente il valore
\begin{lstlisting}[language=Java]
Search(A,e) //A: array in input, e: elemento da cercare
	pos = A.free-1
	while A[pos] != e and pos >= 0 do
		pos = pos-1
	return pos
\end{lstlisting}
Ci posizionamo all'ultima cella piena e mano a mano tornando all'indietro andiamo a cercare il valore che abbiamo in input, se non è presente nell'array andiamo a restistuire il valore $-1$. 
La complessità computazione sarà \textbf{lineare} nello specifico:
$$ T_s(n)=c \cdot n $$
Dove $c$ è il costo fisso delle operazione e $n$ è quante volte si ripete il ciclo.

\subsubsection{Inserimento}
Dato un array e un elemento da inserire, andiamo a verificare tramite la funzione \textbf{ricerca} se l'elemento non sia già presente poiché non vogliamo elementi duplicati.
\begin{lstlisting}[language=Java]
Inserimento(A,e) //A: array in input, e: elemento da cercare
	pos = search(A,e) //-1: non trovato, >=0: indice del valore
	if pos = -1 then
		if(A.free < length(A)) then
			A=resize(A)
		A[A.free]=e
		A.free=A.free+1
\end{lstlisting}
La complessità computazione sarà \textbf{lineare} nello specifico:
$$ T_i(n)=2c \cdot n + c^{\prime} $$
\newpage

\subsubsection{Cancellazione}
Dato un array e un elemento da cancellare, andiamo a verificare tramite la funzione \textbf{ricerca} se l'elemento sia presente in modo da portelo eliminare.
\begin{lstlisting}[language=Java]
Delete(A,e) //A: array in input, e: elemento da cercare
	pos = search(A,e) //-1: non trovato, >=0: indice del valore
	if pos >= 0 then //Elemento trovato
		A[pos]=A[A.free-1]
		A.free=A.free-1
\end{lstlisting}

\subsection{Array Ordinato}
Uno dei grosso problemi dell'array visto in precendenza era la ricerca, poiché al più avevo costo \textbf{lineare} cioè la lunghezza di tutto l'array, con l'array ordinario andiamo a sopperire a questo problema ma ovviamente aggiungendone degli altri.

\subsubsection{Ricerca Binaria}
La ricerca binaria è un algoritmo applicabile solo ad array ordinato che permette una ricerca molto rapida.\\
L'algoritmo è simile al metodo usato per poter trovare una parola sul dizionario: sapendo che il vocabolario è ordinato alfabeticamente, l'idea è quella di iniziare la ricerca non dal primo elemento, ma da quello centrale, cioè a metà del dizionario. Si confronta questo elemento con quello cercato:
\begin{itemize}
\item se corrisponde, la ricerca termina indicando che l'elemento è stato trovato;
    \item se è superiore, la ricerca viene ripetuta sugli elementi precedenti (ovvero sulla prima metà del dizionario), scartando quelli successivi;
    \item se invece è inferiore, la ricerca viene ripetuta sugli elementi successivi (ovvero sulla seconda metà del dizionario), scartando quelli precedenti.
\end{itemize}
Andiamo a definirlo per ricorsione:
\begin{lstlisting}[language=Java]
BinSearch(A, e, i, j) //i: punto inizio, j: punto fine
	if i <= j then
		q=(i+j)/2 //punto centrale
		if A[q] > e then
			i=BinSearch(A, e, i, q-1) //ricerca a "destra"
		else if A[q] < e then
			i=BinSearch(A, e, q+1, j) //ricerca a "sinistra"
		else return q //trovato
	else return -1 //non trovato
\end{lstlisting}
Per studiare la complessità di questo algoritmo (ricorsivo), andiamo a rappresentare le varie chiamate tramite un "albero degenere" in cui ogni nodo ha un solo figlio e così via, il costo di ogni nodo sarà $c$, ad ogni chiamatà andiamo a dimezzare la lunghezza dell'array $n... n/2... n/4....$ il massimo delle chiamate possono essere $h+1$ dove $h$ è l'altezza dell'albero andiamo a generallizare con:
$$ \frac{n}{2^h} \;\;\;\;\; h: \text{è il numero dei richiami} $$
Dobbiamo trovare il valore di $h$ per il quale l'intervallo si riduce a $1$ elemento, in quanto l'elemento desiderato sarà stato trovato. Quindi, dobbiamo risolvere l'equazione:
$$ \frac{n}{2^h} = 1$$
Per risolvere questa equazione per $h$, possiamo moltiplicare entrambi i lati per $2^h$:

$$n = 2^h$$

Ora, per isolare $h$, possiamo applicare il logaritmo in base 2 ad entrambi i lati:

$$h = log_2(n)$$

Quindi, il numero di chiamate ricorsive necessarie per trovare l'elemento desiderato è logaritmico rispetto alla dimensione dell'array $n$. Pertanto, la complessità computazionale dell'algoritmo di ricerca binaria è $O(log n)$ nel caso peggiore.\\
Quindi la ricerca binaria (logaritimica) è più efficente di quella normale (lineare).

\subsubsection{Inserimento/Cancellazione}
Abbiamo visto come in un array ordinato la ricerca (binario) è molto efficemente ma abbiamo come contro che l'inserimento/cancellazione hanno costo peggiore poiché dobbiamo mantenere l'ordinamento, quindi se vogliamo inserire un elemento dobbiamo andare a spostare i valpri per creare un posto disponibile.

\subsection{Resize}
La resize 