\section{Lezione 13 - 13/10/2023}

\subsection{Alberi Red - Black}
Gli alberi Red - Black, sono alberi che associano dei colori ai loro nodi. Come ogni tipo di struttura dati ha le sue caratteristiche. La colorazione ovviamente andra a braccetto con alcune proprieta che di seguito andremo a definire.

\begin{itemize}	
	\item Ogni nodo deve essere rosso o nero.
	\item I nodi rossi potranno essere soltanto all'interno. In poche parole \textbf{i nodi foglie possono essere solo neri}.
	\item Ogni nodo rosso ha \textbf{solo} figli neri.
	\item Per ogni nodo X preso all'interno dell'albero, ogni percorso da X al nodo foglia contiene \textbf{lo stesso numero} di nodi neri.
\end{itemize}

Foto

\subsubsection{Non tutti gli alberi possono essere colorabili}

Per vedere se un albero è colorabile ci sono delle considerazioni da fare:

\begin{itemize}
    \item Colorare subito le foglie e la radice di nero.
    \item Osservare se esiste un sottoalbero è visibilmente piu pesante di un altro. In tal caso l'albero è quasi sicuramente non colorabile. Teoricamente se la differenza di altezza di alberi è maggiore di due allora probabilmente non è colorabile.
\end{itemize}

\subsubsection{Altezza Nera di un albero R-B}
L'altezza nera, di un albero R-B, è il numero di nodi neri che, preso un nodo X, si contano da X fino alle foglie escludendo X.

L'altezza nera è sicuramente minore dell'altezza dell'albero e al massimo uguale.

Dimostriamo dunque che l'altezza è sicuramente:

$$h \le 2^{bh(x)}-1$$

Preso un nodo all'interno di un albero il numero di nodi interni non può essere più piccolo di un albero completamente nero.
$$ n i(x) \ge 2^{bh(x)} -1 $$

Dimostriamo per induzione

\begin{itemize}
	\item Base Induttiva: Albero di altezza zero, quindi il numero di nodi interni di un albero di $h=0$ è \textbf{zero}, andiamo a svolgere l'equazione con la base induttiva:
	$$ 0 \ge 2^{bh(x)}-1 \Rightarrow 2^0-1 = 0 \Rightarrow \text{ VERO }$$
	
	
	\item Caso Induttivo $h > 0$: L'albero contiene almeno un nodo interno. 
	Andiamo a scomporre il nostro albero come sottoalbero sx del figlio sinistro e sottoalbero dx del figlio destro.
	$$ ni(y) \ge 2^{bh(y)}-1 $$
	$$ ni(z) \ge 2^{bh(z)}-1 $$
	Noi sappiamo che 
	$$ ni(x)=1+ni(y)+ni(z) $$
	

\end{itemize}

