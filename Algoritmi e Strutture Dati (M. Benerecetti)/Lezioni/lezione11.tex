\section{Lezione 11 - 10/10/2023}
%\subsection{Bilaciamento Albero AVL}
%Dato che l'inserimento/cancellazione può sbilanciare l'albero cioé 

\subsection{Altezza}
Per aiutarci nel bilanciamento andiamo a scrivere una funzione ausilaria che ci sarà molto utile:
\begin{lstlisting}[language=Java]
Altezza(T)
    ret=-1
    if T != NIL then
        sx=Altezza(T->sx)
        dx=Altezza(T->dx)
        ret=1+max(sx,dx)
return ret
\end{lstlisting}
Questa funzione è \textbf{lineare} ma nel nostro caso voglio che sia logaritmo, quindi per ovviare a questo problema andiamo a inserire un valore $h$ all'interno di ogni nodo.

\begin{lstlisting}[language=Java]
Altezza(T)
    if T = NIL then
        ret=-1
    else
        ret=T->h
return ret
\end{lstlisting}

\subsection{Inserimento AVL}
Andiamo a vedere come scrivere un algoritmo per l'inserimento, essendo l'AVL un albero binario di ricerca sfruttiamo un ragionamento simile, ma con l'aggiunta del bilanciamento per rispettare la condizone degli AVL.
\begin{lstlisting}[language=Java]
InsertAVL(T, k)
if T != NIL then
    if T-key < k then
        T->dx=InsertAVL(T->dx, k)
        T=Bilanciadx(T)
    else if T->key > k then
        T->sx=InsertAVL(T->sx, k)
        T=Bilanciasx(T)
else 
    T=newnodo(k)
    T->h=0
return T
\end{lstlisting}

\subsection{Bilaciamento}
Dato che l'inserimento/cancellazione può sbilanciare un albero abbiamo bisogno di bilanciare l'albero in modo da far rispettare sempre la condizione, abbiamo diversi casi di bilanciamento andiamo ad esaminarli:

\subsubsection{Bilanciamento Sinistro}
\begin{lstlisting}[language=Java]
Bilanciasx(T)
if T != NIL then
    if Altezza(T->sx) - Altezza(T->dx) = 2 then
        if Altezza(T->sx->sx) > Altezza(T->dx->dx) then
            T=Rotazionesx(T)
        else 
            T->sx=Rotazionedx(T->sx)
            T=Rotazionesx(T)
    else 
        T->h=1+max(Altezza(T->sx), Altezza(T->dx))
return T
\end{lstlisting}

