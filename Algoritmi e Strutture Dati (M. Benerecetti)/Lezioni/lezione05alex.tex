\section{Lezione 05 - 26/09/2023}

\subsection{Sviluppo di un Algoritmo Ricorsivo}
Per lo sviluppo di un algoritmo ricorsivo avremo sicuramente bisogno di una strategia diversa da quella di uno sviluppo di uno iterativo. Solitamente la strategia che si effettua e scrivere dapprima l'algoritmo in forma \textbf{iterativa} e poi convertirlo in forma \textbf{ricorsiva}.\\
Nel nostro caso andremo ad affrontare un esempio di conversione da iterativo a ricorsivo di un algoritmo gia sviluppato, cioe la BinSearch(), in modo tale da comprendere le modalita di conversione.\\

\begin{lstlisting}[language=Python]
BinSearch(A, k) 
	ret = -1
	i=0
	j= length(A)-1
	while i<j and ret=-1 do
		q=(i+j)/2
		if A(q) < k then
			i = q+1
		else if A(q) > k then
			j=q-1
		else
			ret = q
	return ret
\end{lstlisting}

Come vediamo ad ogni esecuzione del while, avremo un nuovo campionamento del nostro array, similarmente a come facevamo per la versione ricorsiva, dove invece \textbf{ogni chiamata ricorsiva corrispondeva a un nuovo campionamento}.\\

E facile e veloce notare che un algoritmo iterativo gioca sulle ripetizioni (o iterazioni) date dal ciclo (while in questo caso). Differentemente da questo approccio, l'algoritmo ricorsivo non presenta ripetizioni.

Per crearne uno si ragiona per casi, a seconda di quali informazioni abbiamo all'interno della funzione.\\

Nel nostro caso la funzione ha in input un Array (A) e un numero (k) da trovare all'interno.

Suddividiamo l'intero procedimento in casistiche piu generali, tra le quali:
\begin{enumerate}
\item La sequenza e vuota $(i>j) \rightarrow$ In questo caso l'unica cosa da fare e di far tornare il risultato alla funzione chiamante, con un return negativo della chiamata (nel nostro caso di ragionameto return -1).

\item La sequenza non e vuota : e in questo caso la risposta e piu complessa da dare ma la andiamo a studiare nelle sue sottocasistiche.
	\begin{enumerate}
	\item Se A(q) < k allora il risultato e uguale a Search(A, q+1, j)
	\item Se A(q) > k allora il risultato e uguale a Search(A, i, q-1)
	\item Il risultato e Q
	\end{enumerate}
\end{enumerate}

\begin{lstlisting}[language=Java]
BinSearch(A, k, i, j) //i: punto inizio, j: punto fine
	ris = -1
	if i <= j then
		q=(i+j)/2 //punto centrale
		if A[q] > k then
			ris = BinSearch(A, k, i, q-1) //ricerca a "destra"
		else if A[q] < e then
			ris = BinSearch(A, k, q+1, j) //ricerca a "sinistra"
		else ris = q //trovato
	return ris //non trovato
\end{lstlisting}

\subsection{Le liste}

Le liste sono una sequenza di nodi collegati, cioe aree di memoria suddivise in 2 parti:
\begin{itemize}
\item Dato
\item riferimento ad altro nodo
\end{itemize}

Le liste possono essere operate sia in modalita iterativa che in modalita ricorsiva.

Una sommatoria puo essere definita sia come numero ripetitivo di somme. Oppure la andremo a definire in forma induttiva come:

\[
\sum_{i=1}^{n}i =
\begin{cases}
    0 \text{ se } n=0\\
   	n+\sum_{i=1}^{n-1} i \text{ } \text{se } n \ge 1
\end{cases}
\]


%\begin{lstlisting}[language=Java]
%Somma(A,i,n)
%	if n=0 then 
%		ret = 0
%	else
%		x = Somma(A,1,n-1)
%		ret = n+x
%	return ret
%\end{lstlisting}



Nel nostro caso possiamo dire che la lista L, in input puo essere dunque:
\begin{itemize}
\item $\emptyset$ [NIL]
\item Lista di elementi rappresentabile da una sola cella contenente chiave e puntatore a successivo 
\end{itemize}

Un elemento della lista e dunque il piu piccolo elemento nella quale la lista puo essere decomposta.
\\
Un algoritmo di ricerca in questo tipo di struttura dati devfinito \textbf{sequenziale} non e consigliabile per l'elevata quantita di tempo che potrebbe impiegare il processore nello scorrere tutte le celle.

\subsubsection{Algoritmo iterativo per la ricerca di un valore nella lista dinamica}

Comunque sia iterativamente parlando potremmo scrivere questo algoritmo:

\begin{lstlisting}[language=Java]
Search(L,k)
	ris = NIL
	
\end{lstlisting}