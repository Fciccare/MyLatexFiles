\subsection{Lezione 14 17/10/2023}

\subsubsection{Bilanciamento All'inserimento in Albero R-B}

\begin{lstlisting}[language=Java]
	BilanciaInsSx(T)
		If !NIL(T->sx) and (!NIL(T->sx->sx) or !NIL(T->sx->dx)) then
			v = ViolazioneSxInsRB(T->sx,T->dx)
			Case V of :
				1: T = Caso1(T)
				2: T = Caso2(T)
				3: T= Caso3(T)
		return T
\end{lstlisting}

La violazione verifica se esiste una violazione di regola numero 4 (degli R-B), mandando il sottoalbero sinistro e destro rispettivamente all'interno di una funzione controllo che restituira il tipo di caso di violazione che stiamo riscontrando in quel sottoalbero.

\subsubsection{Controllo Violazione a Sinstra Inserimento R-B}

\begin{lstlisting}[language=Java]
	ViolazioneSxInsRB(s,d)
		v=0
		If s->col = R then
			if d->col = R then
				if s->sx->col = R or s->dx->col = R then
					v=1
			else if s->dx->col = R then
				v=2
			else if s->sx->col = R then
				v=3
		return v
\end{lstlisting}