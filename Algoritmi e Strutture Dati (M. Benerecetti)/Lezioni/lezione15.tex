\section{Lezione 15 - 19/10/2023}
\subsubsection{Conversione Ricorsiva->Iterativo}
%Rubato da Simone Cerrone + ChatGPT
Sappiamo che in una chiamata ricorsiva, ogni istanza mantiene le proprie variabili, e prima di procedere con la successiva chiamata ricorsiva, deve conservare il proprio contesto. Questo contesto si può salvare memorizzando le variabili utilizzate prima di passare alla chiamata successiva. 
\smallskip

Quando la chiamata ricorsiva figlia completa la sua esecuzione, la memoria precedentemente occupata può essere liberata, e il controllo ritorna alla chiamata precedente. Questo processo segue una logica LIFO (Last-In-First-Out), il che significa che l'ultima chiamata effettuata è quella a cui verrà restituito il controllo per prima. Per emulare efficacemente questo meccanismo in un'iterazione, è necessario utilizzare una struttura dati di tipo stack.
\smallskip

Il numero di iterazioni del blocco di codice in una chiamata ricorsiva dipende dall'input fornito, quindi non è possibile conoscere in anticipo quante iterazioni saranno necessarie per un determinato algoritmo. Questo implica che non è appropriato utilizzare un ciclo for, ma è invece necessario implementare un ciclo while per gestire dinamicamente il processo ricorsivo in base all'input specifico.
\subsection{Struttura Conversione}

\begin{lstlisting}[language=Java]
	while start || st != NIL do //start:partenza, st: ripresa
    if start then
      Simula codice fino alla prima chiamata ricorsiva

      Se caso base -> Termina simulazione chiamata (start=false)

      Se chiamata ricorsiva -> Salva contesto e instanza parametri formali
    else 
      Riprende contesto da stack

      Individua punto di ritorno

      Se richiesta chiamata
        Se necessario aggiorna stack imposta parametri formali (start=true)
      Altrimenti tornare e pulire lo stack
\end{lstlisting}

