\section{Lezione 08 - 31/03/2023}

\subsection{Boh}
\blindtext

\subsection{Spazio Vettoriale Finitamente Generato}
Sia $V$ spazio vettoriale si dice \textbf{finitamente generato} $ \Leftrightarrow $
$$ \exists \underline{v}_1,...,\underline{v}_n \in V / V = <\underline{v}_1,...,\underline{v}_n> $$
$$ <\underline{v}_1,...,\underline{v}_n> \; \textbf{generatori} $$
Uno spazio vettoriale è finitamente generabile se ogni
elemento di $V$ può essere scritto come combinazione lineare di $\underline{v}_1,...,\underline{v}_n$.\\

\subsubsection{Esempi}
\begin{itemize}
\item[•] $$ \mathbb{R}^3 = < (1,0,0),(0,1,0),(0,0,1) > $$
\item[•] $$ \mathbb{R}^4 = \mathbb{R}_{2,2} = < \begin{pmatrix}
1 & 0 \\
0 & 0 \\
\end{pmatrix}
\begin{pmatrix}
0 & 1 \\
0 & 0 \\
\end{pmatrix}
\begin{pmatrix}
0 & 0 \\
1 & 0 \\
\end{pmatrix}
\begin{pmatrix}
0 & 0 \\
0 & 1 \\
\end{pmatrix} >  $$
\item[•] $$ \mathbb{R}_2[x] = <1,x,x^2> (ax^2+bx+c)$$
\item[•] $$ \mathbb{R}[x] \; \text{NON È FINITAMNETE GENERATO}$$

\end{itemize}

\subsection{Base}
Sia $V$ uno spazio vettoriale finitamente generato \footnote{D'ora in poi sarà standard e sarà abbreviato in S.V. F.G}, un sistema indipendente di generatori è detto base $\Leftrightarrow$ \\
$\underline{v}_1,...,\underline{v}_n$ hanno le suguenti propietà:
\begin{itemize}
\item[1)] Sono indipendenti
\item[2)] $ V = < \underline{v}_1,...,\underline{v}_n > $
\end{itemize}

\subsubsection{Esempi}


\subsection{Riferimenti}




