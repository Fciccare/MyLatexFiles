\section{Lezione 08 - 31/03/2023 (da migliorare)}

\subsection{Boh}
\blindtext

\subsection{Spazio Vettoriale Finitamente Generato}
Sia $V$ spazio vettoriale si dice \textbf{finitamente generato} $ \Leftrightarrow $
$$ \exists \underline{v}_1,...,\underline{v}_n \in V / V = <\underline{v}_1,...,\underline{v}_n> $$
$$ <\underline{v}_1,...,\underline{v}_n> \; \textbf{generatori} $$
Uno spazio vettoriale è finitamente generabile se ogni
elemento di $V$ può essere scritto come combinazione lineare di $\underline{v}_1,...,\underline{v}_n$.\\

\subsubsection{Esempi}
\begin{itemize}
\item[•] $$ \mathbb{R}^3 = < (1,0,0),(0,1,0),(0,0,1) > $$
\item[•] $$ \mathbb{R}^4 = \mathbb{R}_{2,2} = < \begin{pmatrix}
1 & 0 \\
0 & 0 \\
\end{pmatrix}
\begin{pmatrix}
0 & 1 \\
0 & 0 \\
\end{pmatrix}
\begin{pmatrix}
0 & 0 \\
1 & 0 \\
\end{pmatrix}
\begin{pmatrix}
0 & 0 \\
0 & 1 \\
\end{pmatrix} >  $$
\item[•] $$ \mathbb{R}_2[x] = <1,x,x^2> (ax^2+bx+c)$$
\item[•] $$ \mathbb{R}[x] \; \text{NON È FINITAMNETE GENERATO}$$

\end{itemize}

\subsection{Base}
Sia $V$ uno spazio vettoriale finitamente generato \footnote{D'ora in poi sarà standard e sarà abbreviato in S.V. F.G}, un sistema indipendente di generatori è detto base $\Leftrightarrow$ \\
$\underline{v}_1,...,\underline{v}_n$ hanno le suguenti propietà:
\begin{itemize}
\item[1)] Sono indipendenti
\item[2)] $ V = < \underline{v}_1,...,\underline{v}_n > $
\end{itemize}

\subsubsection{Esempi}

\begin{itemize}
\item[•] $$ \mathbb{R}^2 = < (1,0),(0,1),(0,0) > \; \text{non è una base poiché no indipendente}$$
\item[•] $$ \mathbb{R}^2 =  < (1,0),(0,1) > \; \text{base (canonica e riferimento)} $$
\item[•] $$ \mathbb{R}^n =  < (1,0,...,0),...,(0,0,...,1) > \; \text{base (canonica)} $$
\item[•] $$ \mathbb{R}_{2,2} =  < \begin{pmatrix}
1 & 0 \\
0 & 0 \\
\end{pmatrix}
\begin{pmatrix}
0 & 1 \\
0 & 0 \\
\end{pmatrix}
\begin{pmatrix}
0 & 0 \\
1 & 0 \\
\end{pmatrix}
\begin{pmatrix}
0 & 0 \\
0 & 1 \\
\end{pmatrix} > \; \text{base (canonica e riferimento)} $$
\item[•] $$ \mathbb{R}^2  = < (2,3),(1,2) > \; \text{base (non canonica)} $$
\item[•] $$ \mathbb{R}[x] = <x^2,x,1> \; \text{base (canonica e riferimento)} $$
\end{itemize}

\subsection{Riferimenti}
Un \textbf{riferimento} è una \textbf{base ordinata}.\\
Negli esempi sopra un riferemento è una base con quell'ordine di elementi.

\subsection{Base Estratta}
Sia $V = <\underline{v}_1,...,\underline{v}_n>$ da $\underline{v}_1,...,\underline{v}_n$ posso estrarre una base.\\
Poniamo $\mathbb{R}^2 = (1,0),(0,1),(2,3)$, possiamo levare $(2,3)$.\\
Supponiamo $\underline{v}_1,...,\underline{v}_n$ dipendenti.\\
$ \exists \underline{v}_i $ dipendoni dai rimanenti da $\underline{v}_{i+1},...,\underline{v}_n$
$$ \mathbb{R} = <1> \; \text{base} $$
$$ \mathbb{R}^2 = (1,3),(0,1),(4,6),(5,1),(0,5),(5,6) $$
$$ (1,3),(0,1) \; \text{base estratta}$$
$(1,3),(0,1)$ è base estratta poiché $(4,6),(5,1),(0,5),(5,6) $ poiché sono tutti proposizionali e quindi si possono ricavare da $(1,3),(0,1)$

\subsection{Lemma di Steinz (no dim)}
Se ho $m$ vettori linearmente indipendenti contenuti in un sottospazio generato da $n$
vettori allora il numero di generatori è maggiore o uguale del numero di vettori indipendenti.\\
Sia $V$ spazio vettoriale:
$$ \underline{v}_1,...,\underline{v}_m \; \text{indip.} \; \in <\underline{w}_1,...,\underline{w}_n> $$
$$ m \le n $$

\subsection{Conseguenze Lemma di Steinz}
\begin{itemize}
\item[•] Tutte le basi hanno lo stesso numeri di elementi (vettori)\\
Dim:
\item[•] 
\end{itemize}

\subsection{Dimensione}
La dimensione di una spazio vettoriale $V$ si scrive nel seguente modo $\text{dim}(V)$ indica la \textbf{cardinalità di una base}.\\
Esempi:
\begin{itemize}
\item[• ] $\text{dim}(\mathbb{R}^2) = 2$
\item[• ] $\text{dim}(\mathbb{R}_{n,m}) = n*m$
\item[• ] $\text{dim}(\mathbb{R}_n[x]) = n+1$
\item[• ] $\text{dim}(\mathbb{R}[x]) = \infty$
\item[• ] $\text{dim}(V) = n$
\item[• ] $V=\{\underline{0}\} \; \text{dim}(V) = 0$
\end{itemize}

\subsection{Proposizioni (con dim)}
Sia $V_n$ spazio vettoriale e $n$ la sua dimensione, i seguenti enunciati sono tra loro equivalementi per $\underline{v}_1,...,\underline{v}_n$:
\begin{itemize}
\item[1)] Base
\item[2)] Sistema di Generatori miniale
\item[3)] Sistema di Generatori di ordine minimo
\item[4)] Sistema indipendente massimale
\item[5)] Sistema indipendente di ordine massimo
\end{itemize}



