\section{Lezione 10 - 12/04/2023}

\subsection{Componenti}
Sia $V$ spazio vettoriale prendiamo un riferimento (base ordinata): $\mathbb{R} = (\underline{e}_1,...,\underline{e}_n)$, e prendiamo un vettore $\underline{v} \in V$ allora $\underline{v}$ si potrà scrivere come combinazione lineare: 
$$\underline{v} = h_1 \underline{e}_1+...+h_n\underline{e}_n$$
Chiamiamo i cofficienti $h$ come \textbf{componenti}, più nello specifico:
$$ (h_1,...,h_n) \; \text{n-pla delle componenti} \; \underline{v} \; \text{nel riferimento} \; \mathbb{R} $$
\subsubsection{Esempi}
TODO: INSERIRE ESEMPIO

\subsection{Cambiamento di Riferimento (Formula di Passaggio)}
Presi due riferimenti vogliamo portare un vettore scritto come combinazione linere delle componenti del "vecchio" riferimento al "nuovo" riferimento.\\
Consideriamo: 
$$ \mathbb{R}_1 = (\underline{e}_1,...,\underline{e}_n) \; \text{VECCHIO RIFERIMENTO} $$
$$ \mathbb{R}_2 = (\underline{f}_1,...,\underline{f}_n) \;  \text{NUOVO RIFERIMENTO}$$
Consideriamo un vettore $\underline{v}$ del vecchio riferimento:
$$ \underline{v} = k_1\underline{e}_1+...+k_n\underline{e}_n $$
Come abbiamo visto prima le componenti di questo vettore sono: $(k_1,..,k_n)$\\
Cominciamo ad esperimere $\underline{e}_1,..,\underline{e}_n$ come combinazione lineare del nuovo riferimento:
$$\underline{e}_1 = h_{1,1}\underline{f}_1+...+h_{1,n}\underline{f}_n $$
$$ ... $$
$$\underline{e}_n = h_{n,1}\underline{f}_1+...+h_{n,m}\underline{f}_n $$
Il primo pedice indica il vettore, il secondo lo scorrimento.\\
Andiamo a sostituire la nuova comb. lin. di $\underline{e}_1$ in $\underline{v}$:
$$ \underline{v} = k_1(h_{1,1}\underline{f}_1,+...+h_{1,n}\underline{f}_n)+..+k_n(h_{n,1}\underline{f}_1,+...+h_{n,m}\underline{f}_n) $$
Ora mettiamo in evidenza $\underline{f}$:
$$ \underline{v} = \underline{f}_1\equalto{(k_1 h_{1,1}+...+k_n h_{1,n})}{k_1^{\prime}}+...+\underline{f}_n\equalto{(k_1 h_{n,1}+...+k_n h_{n,m})}{k_n^{\prime}} $$
Quindi $(k_1^{\prime},...,k_n^{\prime})$ le componenti di $\underline{v}$ in $\mathbb{R}_2$
In definitiva avremmo:
$$ k_1^{\prime} = (k_1 h_{1,1}+...+k_n h_{1,n})  $$
$$ ... $$
$$ k_n^{\prime} = (k_1 h_{n,1}+...+k_n h_{n,m})  $$

\subsubsection{Matrice di Passaggio}
Un altro metodo oltre la "formula di passaggio" è la "matrice di passaggio"\\
Andiamo ad esprimere 

\subsubsection{Esempio}
Posti in $\mathbb{R}^3$ consideriamo i seguenti riferimenti:
$$ \mathbb{R}_1 = ((1,2,3),(4,5,6),(0,0,2) \; \text{VECCHIO RIFERIMENTO} $$
$$ \mathbb{R}_2 = ((0,1,0),(0,1,1),(1,1,1)) \;  \text{NUOVO RIFERIMENTO}$$
Usiamo il metodo della "matrice di passaggio", cominciamo col esprimere i vecchie riferimenti in favori dei nuovi:
$$ (1,2,3) = -1(0,1,0) + 2(0,1,1)+1(1,1,1) $$
$$ (4,5,6) = -1(0,1,0) + 2(0,1,1)+4(1,1,1) $$
$$ (0,0,2) = -2(0,1,0) + 2(0,1,1)+0(1,1,1) $$
Ora costruiamo la matrice (per colonna):
$$\begin{pmatrix}
-1 & -1 & -2 \\
2 & 2 & 2 \\
1 & 4 & 0 \\
\end{pmatrix}$$
Ora basta fare il prodotto righe per colonne di un qualsiasi vettore che vogliamo "trasportare":
$$ 
\begin{pmatrix}
-1 & -1 & -2 \\
2 & 2 & 2 \\
1 & 4 & 0 \\
\end{pmatrix}
\begin{pmatrix}
1 \\
0 \\
0 \\
\end{pmatrix}
=
\begin{pmatrix}
-1 \\
2 \\
1 \\
\end{pmatrix}
$$

\subsection{Ultima Osservazione Spazi Vettoriali}
Consideriamo: 
$$ \mathbb{R}^3: < (1,1,1),(0,1,5),(2,3,4),(2,2,2),(4,5,6) >$$
Vogliamo trovare una base (procediamo con la riduazione a gradini):
$$ 
\begin{pmatrix}
1 & 1 & 1 \\
0 & 1 & 5 \\
2 & 3 & 4 \\
2 & 2 & 2 \\
4 & 5 & 6 \\
\end{pmatrix}
\rightarrow
\begin{pmatrix}
1 & 1 & 1 \\
0 & 1 & 5 \\
0 & 1 & 2 \\
0 & 0 & 0 \\
0 & 1 & 2 \\
\end{pmatrix}
\rightarrow
\begin{pmatrix}
1 & 1 & 1 \\
0 & 1 & 5 \\
0 & 1 & 2 \\
0 & 0 & 0 \\
0 & 0 & 0 \\
\end{pmatrix}
\rightarrow
\begin{pmatrix}
1 & 1 & 1 \\
0 & 1 & 2 \\
0 & 1 & 5 \\
0 & 0 & 0 \\
0 & 0 & 0 \\
\end{pmatrix}
\rightarrow
\begin{pmatrix}
1 & 1 & 1 \\
0 & 1 & 2 \\
0 & 0 & 3 \\
0 & 0 & 0 \\
0 & 0 & 0 \\
\end{pmatrix}
$$
Quindi la base sarà:
$$ <(1,1,1),(0,1,2),(0,0,3) $$
\begin{itemize}
\item[•] Avendo $3$ pivot $\Rightarrow dim = 3$
\item[•] Le righe non nulle di una matrice a gradini \textbf{sono indipendenti}
\end{itemize}

\subsection{Determinanti (matrice quadrata)}
Il determinante esiste solo e solamente per matrici quadrate; il determinante si definisce per ricorsione\footnote{Attenzione da non confondere "dimostrazione per induzione" con "definizione per induzione"}.\\

\subsubsection{Matrice Complementare}
La matrice complemenare di $A(i,j)$ consiste nell'eliminare la $i$ riga e $j$ colonna.\\
Esempio $A(2,2)$:
\begin{center}
\stackMath
\stackinset{c}{}{c}{0\baselineskip}{\rule{4.4\baselineskip}{.4pt}}{%
\stackinset{c}{0\baselineskip}{c}{}{\rule{.4pt}{4.0\baselineskip}}{%
\begin{pmatrix}
2 & 3 & 4 \\
1 & 0 & 0 \\
5 & 0 & 1 \\
\end{pmatrix}}}
$\Rightarrow
\begin{pmatrix}
2 & 4 \\
5 & 1 
\end{pmatrix}$
\end{center}

\subsubsection{Complemento Algebrico}
Il complemento algebrico di elemento $a_{i,j}$ della matrice è definito:
$$ A_{i,j} = (-1)~{i+j} det(A(i,j)) $$
Si indica con un "A grande" : $A_{i,j}$\\
Esempio, consideriamo una matrice:
$$ 
\begin{pmatrix}
a_{1,1} & a_{1,2} & \dots & a_{1,n} \\
a_{2,1} & a_{2,2} & \dots & a_{2,n} \\
\vdots & \vdots & \ddots & \vdots \\
a_{n,1} & a_{n_2} & \dots & a_{n,n} \\
\end{pmatrix}
$$
Allora possiamo ricavare il determinante:
$$ a_{1,1}A_{1,1}+a_{1,2}A_{1,2}+...+a_{1,n+1}A_{1,n+1} = det A $$
$$ a_{2,1}A_{2,1}+a_{2,2}A_{2,2}+...+a_{2,n+1}A_{2,n+1} = det A $$
$$ ... $$
$$ a_{n,1}A_{n,1}+a_{n,2}A_{n,2}+...+a_{n,n+1}A_{n,n+1} = det A $$
\begin{center}
\textbf{TUTTI QUESTI VALORI SONO UGUALI}
\end{center}
Questo procedimento è stato effettuato sulle righe, ma si può applicare uguale alle colonne.

\subsection{Determinante per Casi}

\subsubsection{Matrice 2x2}
Per le matrici $2x2$ il determinante si ottiene come diagonale primare - diagonale secondaria\\
Per le matrici $2x2$ il determinante si ottiene come diagonale primare - diagonale secondaria\\
\begin{equation}
\begin{vmatrix}
a & b \\
c & d
\end{vmatrix}
= a*d - b+c
\end{equation}

\subsubsection{Matrice 3x3 - Regola di Sorrus}
Avendo una matrice $3x3$ "duplichiamo" la prima e la seconda colonna  e li posiniamo infondo
$$
\begin{vmatrix}
a_{11} & a_{12} & a_{13}\\
a_{21} & a_{22} & a_{23}\\
a_{31} & a_{32} & a_{33}
\end{vmatrix}
\rightarrow
\begin{pmatrix}
a_{11} & a_{12} & a_{13} & a_{11} & a_{12}\\
a_{21} & a_{22} & a_{23} & a_{21} & a_{22}\\
a_{31} & a_{32} & a_{33} & a_{31} & a_{32}
\end{pmatrix}
$$
Da cui otteniamo:
$$ a_{11}a_{22}a_{33} + a_{12}a_{23}a_{31} + a_{13}a_{21}a_{32} - (a_{13}a_{22}a_{31} + a_{11}a_{23}a_{23} + a_{12}a_{21}a_{33})$$
Esempio:
$$ 
\begin{vmatrix}
1 & 1 & 1 \\
0 & 1 & 1 \\
2 & 2 & 2
\end{vmatrix}
= 2+2+0-(2+0+2) = 0
$$

\subsection{Propietà sul Determinante}
Dimostrazioni(dove neccessarie) ed esempi omessi (per ora)
\begin{itemize}
\item[•] Se una riga dipende dalle rimanenti allora il determinante è zero
\item[•] Se una matrice ha due righe uguale allora il determinante è zero
\item[•] Se il determinante è diverso da zero allora righe indipendenti
\item[•] Se scambiamo due righe il determinante cambia di segno
\item[•] Teorema di Cauchy-Binet: $det(AB) = det(A) * det(B)$
\item[•] $det A$ = $det A^t$
\end{itemize}

