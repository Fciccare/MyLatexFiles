\section{Lezione 14 - 26/04/2023}

\subsection{Propietà}
Consideriamo una funzione lineare $f: V \rightarrow W$ elenchiamo le seguenti propietà:

\begin{itemize}


\item[1)] $f(\underline{0}_v) = \underline{0}_w$


\item[2)] $f(h_1\underline{v}_1+...+h_n\underline{v}_n) = h_1f(\underline{v}_1)+...+h_nf(\underline{v}_n)$


\item[3)] $\underline{v} \; \text{dipende da} \; \underline{v}_1,...,\underline{v}_n \Rightarrow f(\underline{v}) \; \text{dipende da} \; f(\underline{v}_1),...,f(\underline{v}_n)$\\
$\textbf{DIM:}$
$$ \underline{v} = h_1\underline{v}_1+...+h_n\underline{v}_n $$
$$ f(\underline{v}) = f(h_1\underline{v}_1+...+h_n\underline{v}_n) =^{2)} h_1f(\underline{v}_1)+...+h_nf(\underline{v}_n)$$

\item[3.1)] $f$ conserva dipendenza lineare MA NON L'INDIPENDENZA
$$ \underline{v}_1,...,\underline{v}_n \;DIP\; \Rightarrow f(\underline{v}_1),...,f(\underline{v}_n) \;DIP\; $$
$$ \exists \underline{v}_i \;DIP\; \underline{v}_1,...,\underline{v}_{i+1},...,\underline{v}_n \Rightarrow f(\underline{v}_i) \;DIP DA\; f(\underline{v}_1),...,f(\underline{v}_{i+1}),...,f(\underline{v}_n)$$

\item[4)] $f(<S>) = <f(S)> = <f(\underline{v}_1),...,f(\underline{v}_n))>$\\

Andiamo a dimostare la prima uguaglianza nel solito modo:
\subitem DIM $\subseteq$:
$$ \underline{v} \in f(<S>) $$
$$ \exists \underline{w} \in <S> \underline{v} = f(\underline{w})$$
$$ \underline{w} = h_1\underline{v}_1+...+h_n\underline{v}_n$$
$$ f(\underline{w}) = h_1f(\underline{v}_1)+...+h_nf(\underline{v}_n) \in <f(S)>$$
 
\subitem DIM $\supseteq$:
$$ \underline{w} \in <f(S)> $$
$$ \underline{w} = h_1f(\underline{v}_1)+...+h_nf(\underline{v}_n)$$
$$ f(h_1\underline{v}_1+...+h_n\underline{v}_n) \in f(<S>)$$
$$ \underline{w} \in f(<S>) $$

\item[5)] $H \le V \rightarrow f(H) \le W$\\
Dimostriamo sia sottospazio vettoriale:

\subitem Non vuoto:
$$ \underline{0} \Rightarrow f(\underline{0})= \underline{0} \in H $$
\subitem Stabilità Somma:
$$ \underline{w},\underline{w}^{\prime} \in f(H) \;\;\;\;\; \underline{v},\underline{v}^{\prime} \in H $$
$$ \underline{w} = f(\underline{v}) \;\;\;\;\; \underline{w}^{\prime} = f(\underline{v}^{\prime}) $$
$$ \underline{w}+\underline{w}^{\prime} = f(\underline{v})+f(\underline{v}^{\prime})= f(\underline{v}+\underline{v}^{\prime}) \in f(H) $$

\subitem Stabilità Prodotto:
TODO: DA FARE COME ESERCIZIO (un giorno lo farò)

\end{itemize}

\subsection{Kernel}

Consideriamo la funzione lineare $f: V \rightarrow W$, denotiamo con $Im f$ il sottospazio immagine di $f$ cioé: $\{f(\underline{v})/\underline{v} \in V\}=f(V) = Im f$.\\

Andiamo a definire un altro insieme chiamato \textbf{Kernel} o anche detto \textbf{ker}, cioé l'insieme di tutti i valori del dominio che vanno a finire nel neutro nella fattispecie: $ ker f = {\underline{v} \in V / f(\underline{v}) = \underline{0}} $\\

Andiamo a dimostare che il ker sia un sottospazio:
\begin{itemize}
\item[Non vuoto:] Vero poiché il neutro gli appartiene

\item[Stabilità Somma:]
$$ \underline{v},\underline{v}^{\prime} \in ker f \Rightarrow \underline{v}+\underline{v}^{\prime} \in ker f $$

$$ f(\underline{v})+f(\underline{v}^{\prime}) = f(\underline{v}+ \underline{v}^{\prime}) = \underline{0}+\underline{0} = \underline{0} $$

\item[Stabile Prodotto:]
$$ h \in \mathbb{R} \;\;\;\;\; h\underline{v} \in ker f$$
$$ f(h\underline{v}) = hf(\underline{v}) = \underline{0} $$

\end{itemize}

Avendo definito questi due concetti possiamo sfruttare alcune propietà per capire più facilente l'iniettività o surriettività di un applicazione lineare:
\begin{itemize}
\item[•] Se l'immagine del dominio combacia col codominio allora la funzione \textbf{è surriettivita}
$$ f(V) = Im f = W \Leftrightarrow \text{f è surriettiva} $$

\item[•] Se il $ker f = \{\underline{0}\} \Leftrightarrow$ $f$ è \textbf{iniettiva.}\\

\subitem \textbf{DIM $\Rightarrow$:}
$$ \underline{v},\underline{w} \in V \; \text{con} \; f(\underline{v}) = f(\underline{w}) \Rightarrow \underline{v} = \underline{w} $$
$$ f(\underline{v}-\underline{w}) = f(\underline{v})-f(\underline{w}) = \underline{0} \Rightarrow \underline{v}-\underline{w} \in ker f \Rightarrow \underline{v}-\underline{w} = \underline{0} \Rightarrow \underline{v} = \underline{w}$$

\subitem \textbf{DIM $\Leftarrow$:}
$$ \underline{0} \in ker f \;\;\;\;\; \{\underline{0}\} \subseteq ker f $$
$$ \underline{v} \in ker f \Rightarrow f(\underline{v}) = \underline{0} \Rightarrow \underline{v} = \underline{0} $$
Tutti gli elementi combaciano con il neutro

\end{itemize}


\subsubsection{Esempi}
\begin{itemize}
\item[•] $\underline{0}_v: V \rightarrow V (\underline{v} \rightarrow \underline{0})$\\
$$ Im f = f(V) = \{\underline{0}\} $$
$$ Ker f = V$$ 
$$ \text{INIETTIVA E SURRIETTIVA } \Leftrightarrow V = \{\underline{0}\}$$

\item[•] $ id_v V \rightarrow V (\underline{v} \rightarrow \underline{v})  $
$$ Im f = f(id_v) = V \; INIETTIVA$$
$$ Ker f = \{\underline{0}\} \;  SURRIETTIVA$$

\item[•] $\mathbb{R}^2 \rightarrow \mathbb{R}^2$
$$ (x,y) \rightarrow \begin{pmatrix}1 & 2 \\ 0 & 1\end{pmatrix} \begin{pmatrix}x \\ y \end{pmatrix} = \begin{pmatrix}x+2y \\ y\end{pmatrix} $$

\subitem • Per verificare l'iniettività dobbiamo verificare il $ker f$ sia formato solo dal vettore nullo, quindi vediamo per quali valori di $x,y$ $f(x,y) = 0$:
$$ \systeme{x+2y=0, y=0} \Rightarrow \systeme{x=0,y=0} \; \text{È INIETTIVA} $$

\subitem • Per verificare la surrietività prendiamo un sistema di generatori, in questo caso quella canonica:
$$ <(1,0),(0,1)> \; dim=2$$
$$ f(\mathbb{R}^2) = <f(1,0),f(0,1)> = <(1,0),(2,1)>$$
Essendo indipendenti hanno $dim=2$ quindi è surriettiva visto che è rimasto un sistema di generatori.

\item[•] $f: \mathbb{R}_2[x] \rightarrow \mathbb{R}[x] (ax^2+bx+c \rightarrow 2ax+b)$\\

\subitem •  Per verificare l'iniettività dobbiamo verificare il $ker f$ sia formato solo dal vettore nullo, quindi vediamo per quali valori di $a,b,c$ $f(ax^2+bx+c) = 0$:
$$ \underline(v) \in \mathbb{R}_2[x] / f(\underline{v})= \underline{0}$$
$$ f(ax^2+bx+c) =  2ax+b = \underline{0} \Leftrightarrow a = 0 = b $$
Quindi per $ker f$ sarà formato da tutti i valori $c \in \mathbb{R}$ quindi non è iniettiva.
$$ ker f = \{c/c \in \mathbb{R}\} DIM=1$$

\subitem •  Per verificare la surrietività prendiamo un sistema di generatori, in questo caso quella canonica:
$$ <x^2,x,1> dim=3$$
$$ f(<x^2,x,1>) = <2x,1,0> = <2x,1> dim = 2 \; \textbf{NON È SURRIETIVA}$$

\item[•] $f: \mathbb{R}^2 \rightarrow \mathbb{R}^3 $
$$ (x,y) \rightarrow \begin{pmatrix} 1&2\\3&4\\0&1 \end{pmatrix} \begin{pmatrix} x \\ y \end{pmatrix} $$

\subitem •  Per verificare l'iniettività dobbiamo verificare il $ker f$ sia formato solo dal vettore nullo, quindi vediamo per quali valori di $x,y$ $f(x,y) = 0$:
$$ \begin{pmatrix} 1&2\\3&4\\0&1 \end{pmatrix} \begin{pmatrix} x \\ y \end{pmatrix} = \begin{pmatrix} 0 \\ 0 \end{pmatrix}  \Leftrightarrow \systeme{x+2y=0, 3x+4y=0, y=0} \Leftrightarrow \systeme{x=0,y=0} $$

\end{itemize}

\subsection{Altre propriepietà}
Consideriamo una generica funzione lineare $f: V \rightarrow W$:
\begin{itemize}
\item[•] Il monomorfismo (omomorfismo iniettivo) mantiene l'indipendenza, cioè:
$$  \underline{v}_1,...,\underline{v}_n \; INDIP. \Rightarrow f(\underline{v}_1),...,f(\underline{v}_n) \; INDIP.$$
$$ \text{BASE DI V} \rightarrow \text{BASE DI }f(V) $$
\textbf{DIM:}
Prendiamo una combinazione lineare dei vettori immagine ed eguagliamola all’elemento
neutro:
$$ f(h_1\underline{v}_1+...+h_n\underline{v}_n) = h_1f(\underline{v}_1)+...+h_nf(\underline{v}_n) = \underline{0}   $$
Possiamo leggerlo anche nel seguente modo:
$$ h_1\underline{v}_1+...+h_n\underline{v}_n \in ker f $$
Affinché questi vettori siano indipendenti bisogna dimostrare che tutti gli scalari siano necessariamente pari a zero ma essendo i vettori indipendenti per ipotesi ed avendolo uguagliato all'elemento neutro troviamo che tutti gli scalari sono nulli.

\item[•] $$ dim(ker f) + dim(Im f) = dim (V) $$ 
La dimensione del dominio è uguale alla somma delle dimensioni dell'immagine del dominio e del ker di f.\\
\textbf{DIM:}\\
Ragioniamo per casi:
\subitem • $ker f = \{\underline{0}\}$\\
Quindi la funzione ha $dim=0$ ed è iniettiva (monomorfismo) per quanto visto sopra conserva la base (indipendenza) e quindi la dimensione quindi:
$$ 0 + dim(Im f) = dim (V) $$

\subitem • $ker f = V$:\\
Come abbiamo dagli esempi di prima l'unico caso in cui $ker f = V$ è quando $f$ è la funzione nulla e sempre per la precendente osservezione $dim (Im f) = 0$, quindi:
$$ dim (ker f) + 0 = dim (V) $$

\subitem • $\{0\} < ker f < V$\\
\blindtext

\end{itemize}


\subsection{Cordinazione Associata}
Preso uno spazio vettoriale $V$ e un suo riferimento $(\underline{e}_1,...,\underline{e}_n)$ possiamo considerare la seguente funzione lineare:
$$ \underline{v} = h_1\underline{e}_1+...+h_n\underline{e}_n \rightarrow (h_1,...,h_n) \in \mathbb{R}^n $$



