\section{Lezione 01 - XX/03/2023}

\subsection{Definizioni di base}

\subsubsection{Prodotto Cartesiano}

Presi $S,T \neq \emptyset$, possiamo definire il prodotto cartesiano:
\begin{equation*}
SxT = \{(s,t)/ s \in S, t \in T\}
\end{equation*}
\begin{equation*}
S^2 = SxS = \{(s,t)/ s \in S, t \in T\}
\end{equation*}
Da non confendere con la definizione di diagonale: $ S^2 = SxS = \{(s,s)/ s \in S\} $.

\subsubsection{Coppie}
La definizione di coppia è la seguente:
\begin{equation*}
(s,t) = \{\{s,t\}, \{s\}\}
\end{equation*}

Negli insiemi l'ordine non conta $ \{s,t\} = \{t,s\}$, invece nelle coppie è rilevante, infatti due coppie sono uguali se e solo  sono ordinatamente uguali:
$$ (s,t) = (s^\prime, t^\prime) \Leftrightarrow s=s^\prime, t=t^\prime $$

Andiamo a dimostrare questa affermazione: 

\begin{itemize}
\item DIM $\Leftarrow$: BANALE
\item DIM $\Rightarrow$
$ (s,t) = (s^\prime, t^\prime) \Leftrightarrow \{\{s,t\}, \{s\}\} = \{\{s^\prime,t^\prime\}, \{s^\prime\}\} $\\
Ragioniamo per casi:
	\begin{itemize}
		\item[a] SE $s=t$:\\
			$$ \text{Sx:} \{\{s,t\}, \{s\}\} \Rightarrow \{\{s,s\},\{s\}\} \Rightarrow \{s\} $$
			$$ \text{Dx:} \{\{s^\prime,t^\prime\}, \{s^\prime\}\} \Rightarrow \{\{s^\prime,s^\prime\},\{s^\prime\}\} \Rightarrow \{s^\prime\} $$
		\item[b] SE $s \neq t$:\\
			Usiamo le definizioni di uguaglianza tra insiemi:
			$$\{s\} = \{s^\prime\} \Rightarrow s=s^\prime$$
			$$\{s,t\} = \{s^\prime,t^\prime\} \wedge s=s^\prime \:\: \Rightarrow t = t^\prime$$
	\end{itemize}
\end{itemize}

% $ a \bullet b $



\subsubsection{Operaziona Interna}



\subsubsection{Operaziona Esterna}

\subsubsection{Prodotto Scalare Standard}

\subsubsection{Matrice in R}

