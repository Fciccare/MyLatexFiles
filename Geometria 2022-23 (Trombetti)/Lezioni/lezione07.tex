\section{Lezione 07 - 29/03/2023}

\subsection{Sottospazi Equivalenti}
Siano $S_1,S_2 \le V $ si dicono \textbf{equivalente} se e solo se generano lo stesso sottospazio vettoriale quindi:
$$ \Leftrightarrow <S_1> = <S_2> $$
($<S_1>,<S_2>$ si dicono sistema di generatori)

\subsubsection{Esempi}
Presi $\underline{v}, \underline{v}, \underline{0}, \underline{w}$ equivale a $\underline{v}, \underline{w}$?\\
Dobbiamo andare a verificare che ogni elemento di $S_1$ si possa scrivere come combinazione lineare di $S_2$, quindi dobbiamo andare a verificare la doppia inclusione.\\
In questo possiamo notare come vale l'equivalenza poiché, possiamo levare la doppia ripetizione di $\underline{v}$ dal $S_1$, e $\underline{0}$ essendo il neutro deve essere necessariamene presente per essere sottospazio, quindi vale la doppia inclusione.\\

Ponendoci in $\mathbb{R}^3$ consideriamo il seguente sottospazio:
$$ <(1,2,1),(2,4,2),(0,0,1),(1,2,50)> $$
Possiamo notare come $(2,4,2)$ e $(1,2,50)$ sono combinazioni lineari, questo ci permette di eleminarli, quindi:
$$ <(1,2,1),(0,0,1)> \; \; \textbf{Base}$$
Questi due elementi sono indipendenti poiché l'unica combinazione possibile è $h=k=0$.\\

Consideriamo $\mathbb{R}^n$ come sempre possiamo considerare la matrice come una lunga riga.
$$
\begin{pmatrix}
1 & 2 & 1 \\
2 & 4 & 2 \\
0 & 0 & 1 \\
1 & 2 & 50\\
\end{pmatrix}
\rightarrow
\begin{pmatrix}
1 & 2 & 1 \\
0 & 0 & 0 \\
0 & 0 & 1 \\
0 & 0 & 49\\
\end{pmatrix}
\rightarrow
\begin{pmatrix}
1 & 2 & 1 \\
0 & 0 & 1 \\
0 & 0 & 49 \\
0 & 0 & 0\\
\end{pmatrix}
\rightarrow
\begin{pmatrix}
1 & 2 & 1 \\
0 & 0 & 1 \\
0 & 0 & 0 \\
0 & 0 & 0 \\
\end{pmatrix}
$$
Le trasformazioni di riga ($E_1,E_2,E_3,E_4$) mantendono i sottospazi.\\
Le righe non nulla di una matrice sono sempre sistemi indipendenti.\\
Poniamoci in $\mathbb{R}_2[x]$ e prendeniamo:
$$ <x^2+2x+1, 2x^2+4x+2, 1, x^2+2x+50> $$
Possiamo consideraro anche solo i termini senza le incognite:
$$ <(1,2,1),(2,4,2),(0,0,1),(1,2,50)> $$
Possiamo portarla in forma matriciale:
$$ 
\begin{pmatrix}
1 & 2 & 1 \\
2 & 4 & 2 \\
0 & 0 & 1 \\
1 & 2 & 50 \\
\end{pmatrix}
$$
E possiamo portarla a gradini:
$$
\begin{pmatrix}
1 & 2 & 1 \\
0 & 0 & 1 \\
0 & 0 & 0 \\
0 & 0 & 0 \\
\end{pmatrix}
$$
Quello che ci viene alla fine è:
$$ <x^2+2x+1, 1> $$

\subsection{Osservazioni sulla in/dipendenza}
\subsubsection{Dipendenza}
Sia $v_1,...,v_n \in V$ \footnote{Consideriamo $n \ge 2$ perché se $n = 1$ i casi di riducono unicamente a: $ \underline{v} \neq \underline{0} $ indipendente e $ \underline{v} = \underline{0} $ dipendente} sono linearmente dipendenti $ \Leftrightarrow \exists i: \underline{v}_i $ dipende dai rimanenti.
Dim:\\
Per ipotesi sappiamo: 
$$ \exists h_1,...,h_n \neq (0,0,...0): h_1\underline{v}_1 + ... + h_n\underline{v}_n = \underline{0} $$
Supponiamo $h_1 \neq 0$ allora:
$$ \underline{v_1} = h_1^{-1}(-h_2\underline{v}_2 + ... + (-h_n)\underline{v}_n $$
Quindi $\underline{v}_1$ è combinazione lineare di $\underline{v}_2,...,\underline{v}_n$ quindi dipende da questi vettori.\\
(vale anche il viceversa).

\subsubsection{Indipendenza}
Per scrivere l'indipendenza ci basta unicamente fare il negato della dipendenza:\\
$ v_1,...,v_n \in V $ sono indipendenti $ \Leftrightarrow \forall \underline{v}_i: \underline{v}_i$ non dipende da: $\underline{v}_1,...,\underline{v}_n $\\

\subsubsection{Propietà}
Sia $V$ spazio vettoriale, definiamo le seguenti propietà:
\begin{itemize}
\item[•] Se $\underline{v}_1,...,\underline{v}_n$ dipendono aggiungere $\underline{w}_1,...,\underline{w}_n$ fa rimanere la dipendenza.

\item[•] Se $\underline{v}_1,...,\underline{v}_n, \underline{w}_1,...,\underline{w}_n$ sono indipendenti allora andando a restringere a $\underline{v}_1,...,\underline{v}_n$ rimane indipendente.

\item[•] Se $\underline{v}_1,...,\underline{v}_n$ sono indipendenti allora c'è \textbf{l'unicità di scrittura}.\\
Dim:
$$ \textbf{da aggiungere} $$

\item[•] Presi $W_1,W_2 \le V$ e per ipotesi in somma diretta $W_1 \cap W_2 = \{\underline{0}\}$ e presi $\underline{0} \neq \underline{v} \in W_1$ e $\underline{0} \neq \underline{w} \in W_2$ allora $\underline{v}$ e $\underline{w}$ \textbf{sono indipendenti}.\\
Dim:
$$ \textbf{da aggiungere} $$

\item[•] Generalizziamo il caso precendente

\item[•] La somma direta implica l'unicità di scrittura

\end{itemize}






