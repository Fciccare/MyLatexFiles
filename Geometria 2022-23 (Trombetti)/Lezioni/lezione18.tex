\section{Lezione 18 - 12/05/2023}

\subsection{Ultimo Criterio di Diagonalizazzione (Completo)}
Consideriamo l'endomorfimo $f: V_n \rightarrow V_n$ è diagonalizzabile $\Leftrightarrow$:
\begin{itemize}
\item[1)]Il polinomio caratteristico ha tutte le radici in $\mathbb{R}$
\item[2)]La molteplicità algebrica combancia con quella geometrica
\end{itemize}
\textbf{DIM:}
$$ 
\exists \underbrace{\mathtt{R}}_{\underline{e}_1,...,\underline{e}_n} di V_n/ M_{\mathtt{R}}(f) \; \text{è diagonale} 
$$
Sapendo che $f(\underline{e}_1) = a_1\underline{e}_1$ la matrice associata è:
$$ 
\begin{pmatrix}
a_1 & 0 & 0\\
0 & \ddots & 0 \\
0 & 0 & a_n
\end{pmatrix}
$$
Andiamo a calcolarci il polinomio caratteristico:
$$ 
det \begin{pmatrix}
a_1 & 0 & 0\\
0 & \ddots & 0 \\
0 & 0 & a_n
\end{pmatrix}
= (a_1-x)(a_2-x)...(a_n-x)
$$
Essendo diagonale il determinante non è che il prodotto della diagonale.\\
Il polinomio è già scomposto ed di grado $1$, ed ha tutti i fattori reali quindi abbiamo verificato il primo punto.\\
\textbf{Andiamo a verificare il secondo punto:}\\

