\section{Lezione 18 - 12/05/2023}

\subsection{Ultimo Criterio di Diagonalizazzione (Completo)}
Consideriamo l'endomorfimo $f: V_n \rightarrow V_n$ è diagonalizzabile $\Leftrightarrow$:
\begin{itemize}
\item[1)]Il polinomio caratteristico ha tutte le radici in $\mathbb{R}$
\item[2)]La molteplicità algebrica combancia con quella geometrica
\end{itemize}
\textbf{DIM:}
$$ 
\exists \underbrace{\mathtt{R}}_{\underline{e}_1,...,\underline{e}_n} di V_n/ M_{\mathtt{R}}(f) \; \text{è diagonale} 
$$
Sapendo che $f(\underline{e}_1) = a_1\underline{e}_1$ la matrice associata è:
$$ 
\begin{pmatrix}
a_1 & 0 & 0\\
0 & \ddots & 0 \\
0 & 0 & a_n
\end{pmatrix}
$$
Andiamo a calcolarci il polinomio caratteristico:
$$ 
det \begin{pmatrix}
a_1 & 0 & 0\\
0 & \ddots & 0 \\
0 & 0 & a_n
\end{pmatrix}
= (a_1-x)(a_2-x)...(a_n-x)
$$
Essendo diagonale il determinante non è che il prodotto della diagonale.\\
Il polinomio è già scomposto ed di grado $1$, ed ha tutti i fattori reali quindi abbiamo verificato il primo punto.\\
\textbf{Andiamo a verificare il secondo punto:}\\
Consideriamo un polinomio di grado massimo $n$, possiamo riscriverlo nel seguente modo:
$$ (a_1-x)(a_2-x)...(a_n-x) = (a_1-x)^{n_1}...(a_n-x)^{n_m} $$
TODO: RIASCOLTARE AUDIO

\textbf{DIM $\Leftarrow$:}
Consideriamo $m$ autovalori e le loro rispettive molteplicità algebriche che combaciano con quelle geometriche:
$$ \lambda_1,...,\lambda_m \; \text{autovalori} $$
$$ n_1,...,n_m \; \text{moltep. algebrica = geometria} $$
$$ n_1+...+n_m = n $$
La \textbf{molteplicità algebrica sarà:} $n_1+...+n_m=n$\\
Andiamo a calcolarci quella geometrica, sappiamo che la molt. geo. è compresa tra $1 \le m_g(h) \le m_a(h)$, quindi sappiamo che:
$$ dim(V(a_1)),...,dim(V(a_n)) $$
Sappiamo che ogni di questi autospazi ci sono gli autovettori:
$$ \underline{e}_1,...,{\underline{e}_i}_1 $$


\subsubsection{Esempio/Esercizi}
Consideriamo il seguente endomorfismo $f: \mathbb{R}^3 \rightarrow \mathbb{R}^3$ $(x,y,z) \rightarrow (3y,3x,3t)$, e consideriamo come sempre il riferimento canonico $\mathtt{R}=((1,0,0),(0,1,0),(0,0,1))$, consideriamo la matrice di passaggio e facciamone il determinante:
$$
\begin{pmatrix}
0 & 3 & 0 \\ 3 & 0 & 0 \\ 0 & 0 & 3
\end{pmatrix} 
det \begin{pmatrix}
0-t & 3 & 0 \\ 3 & 0-t & 0 \\ 0 & 0 & 3-t
\end{pmatrix} 
=(3-t)(t^2-9)= (3-t)(t-3)(t+3)
$$
Avendo trovato il polinomio caratteristico ed avendolo scomposto andiamo a trovare le soluzioni:
$$ t=3,3 $$
$$ M_a(-3) = 1 = M_g(-3)$$
$$ M_a(3) = 2$$
Andiamo a calcolare la molteplicità geometrica di $3$:
$$
\begin{pmatrix}
-3 & 3 & 0 \\ 3 & -3 & 0 \\ 0 & 0 & 0
\end{pmatrix}
$$
Consideriamo il sistema associato:
$$ 
\systeme{-3x+3y=0, 3x-3y=0} \rightarrow \systeme{3x=y,y=y}
$$
Quindi il sistema delle soluzioni sarà:
$$
\overline{S} = \{(y,y,z)/y,z \in \mathbb{R}\}
$$
Quindi sappiamo che la $dim=2$ quindi $<(1,1,0),(0,0,1)>$
Andiamo a sostituire anche $-3$ nella matrice:
$$ 
\begin{pmatrix}
3 & 3 & 0  \\ 3 & 3 & 0 \\ 0 & 0 & 6
\end{pmatrix}
\begin{pmatrix}
x \\ y \\ z
\end{pmatrix}
=
\begin{pmatrix}
0 \\ 0 \\ 0
\end{pmatrix}
$$
Quindi il sistema associato sarà:
$$ 
\systeme{3x+3y = 0, 3x+3y=0, 6z=0} \Rightarrow \systeme{x=-y,x=-y, z=0}
$$
L'insieme delle soluzioni sarà:
$$ 
\overline{S}=\{(-y,-y,0)/ y \in \mathbb{R} \} = <(-1,-1,0)>
$$
\textbf{Quindi alla fine:}
$$ V(3) \uplus V(-3) = \mathbb{R}^3 $$

\subsection{Diagonalizzazione Matrice}
Fin ad ora abbiamo sempre visto la diagonalizzazione degli endomorfismi, invece ora parliamo di matrici.\\
Come già sappiamo una matrice $A \; \text{è diagona.} \Leftrightarrow \sim \text{diagonale}$.
Possiamo dire la matrice associata è diagonalizzabile se o solo se:
$$ M(\mathtt{R}) \Leftrightarrow F_A \; \text{è diagonal.}$$
$$ \mathbb{R}^n \rightarrow \mathbb{R}^n $$
$$ X \rightarrow AX $$
Diamo un paio di definizioni:
\begin{itemize}
\item[Autovettore di A] 
$$ AX= \lambda X \; X \neq 0  \;\;\; F_A(x) = \lambda X $$

\item[Autospazio di $\lambda$]
$$ V(\lambda) = \{X/ AX = \lambda X \} $$

\end{itemize}
Diamo una definizione più corretta:
$$ F_A \; \text{diagonaliz.} \Leftrightarrow \exists \mathtt{R}=(\underline{e}_1,...,\underline{e}_n) \; \text{rif. di} \; \mathbb{R}^n $$
Quindi:
$$ M_{\mathtt{R}}(F_A) = D \sim M_{\mathtt{R}}(F_A) = A $$
$$ P^{-1}AP = D $$


\subsubsection{Esempio/Esercizio}
Consideriamo la funzione matriciale $F_A: \mathbb{R}^3 \rightarrow \mathbb{R}^3, (x,y,z) \rightarrow A\begin{pmatrix}
x \\ y \\ z
\end{pmatrix}$\\
Consideriamo la matrice $A$ associatà già osservata in precendenza:
$$
\begin{pmatrix}
0 & 3 & 0 \\ 3 & 0 & 0 \\ 0 & 0 & 3
\end{pmatrix}
$$
Avendo già trovato tutto il necessario dall'esercizio precendente abbiamo:
$$ p_x = det A = (3-t)(t^2-9) $$ 
$$ \underbrace{(1,1,0),(0,0,1)}_3,\underbrace{(1,-1,0)}_{-3} $$
Costruiamo la matrice per colonne:
$$ 
p = \begin{pmatrix}
1 & 0 & 1 \\ 1 & 0 & -1 \\ 0 & 1 & 0
\end{pmatrix}
$$
Quindi la matrice:
$$
{\begin{pmatrix}
1 & 0 & 1 \\ 1 & 0 & -1 \\ 0 & 1 & 0
\end{pmatrix}}^{-1}
\begin{pmatrix}
0 & 3 & 0 \\ 3 & 0 & 0 \\ 0 & 0 & 2
\end{pmatrix}
\begin{pmatrix}
1 & 0 & 1 \\ 1 & 0 & -1 \\ 0 & 1 & 0
\end{pmatrix}
= 
$$
Quindi la matrice associatà sarà:
$$
=\begin{pmatrix}
3 & 0 & 0 \\
0 & 3 & 0 \\
0 & 0 & -3
\end{pmatrix}
$$


\subsection{Prodotto Diretto (Esterno)}
%Simone Cerrone:
Vediamo di seguito un procedimento per la costruzione di ulteriori spazi vettoriali ed è il concetto di prodotto
esterno. Il prodotto che fino ad ora avevamo visto è il prodotto diretto interno (dove abbiamo già lo spazio
vettoriale e scrivevamo un prodotto diretto da qualcosa interno allo spazio vettoriale).\\
Presi due spazi vettoriali $V_1, V_2$ prendiamo il prodotto cartesiano:
$$ V = V_1xV_2 = \{(\underline{v}_1,\underline{v}_2)/ \underline{v}_1 \in V_1, \underline{v}_2 \in V_2 \} $$
Andiamo a definire somma e prodotto esterni:
\begin{itemize}
\item[•]$+:$
$$ (\underline{v}_1, \underline{v}_2) + (\underline{v}_1^{\prime}, \underline{v}_2^{\prime} = (\underline{v}_1+\underline{v}_1^{\prime}, \underline{v}_2+\underline{v}_2^{\prime})$$

\item[•]$\cdot:$
$$ h(\underline{v}_1, \underline{v}_2) = (h\underline{v}_1, h \underline{v}_2$$
\end{itemize}
Possiamo dimostrare che sia un sotto spazio (omesso)\\

\subsubsection{Esempio}
$$ R_2[x]xM_{2,2}(\mathbb{R}) $$
Prendiamo i riferimenti canonici:
$$
((a_2x^2+a_1x+a_0), \begin{pmatrix}
b_1 & b_2 \\ b_3 & b_4
\end{pmatrix}
$$
Quindi le basi sono
$$ (1,0),(x,0),(x^2,0),(0, \begin{pmatrix}1&0\\0&0\end{pmatrix}),(0, \begin{pmatrix}0&1\\0&0\end{pmatrix}),(0, \begin{pmatrix}0&0\\1&0\end{pmatrix}), (0, \begin{pmatrix}0&0\\0&1\end{pmatrix})  $$
Risuleterà essere isomorfo ad $\mathbb{R}^7$ poiché:
$$ dim(R_2[x])+dim(M_{2,2}(\mathbb{R}) = dim(\mathbb{R}^7) $$
$$ 3+4=7$$






