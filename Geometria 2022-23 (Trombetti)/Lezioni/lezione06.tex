\section{Lezione 06 - 24-03-2023}

\subsection{Propietà Sottospazio Vettoriale}

\begin{itemize}
\item[$W \underline{<} V$ è stabile rispetto alla somma di $n$ oggetti]
Siano $\underline{w}_1, ... , \underline{w}_n \in W$ si ha $w_1 + w_2 \in W \Rightarrow$ 
\item[Famiglia di sottospazi vettoriali]
Sia $ \mathbb{L}$ una famiglia di sottospazi di $V$, l'intersezione dei sottospazi della famiglia $\mathbb{L}$ è un sottospazio e si indica: 
$$ \bigcap_{L \in \mathbb{L}} L $$
L'intersezione di una qualunque famiglia di sottospazi è un sottospazio.\\
Dimostriamolo: 
	\subitem Neutro: Il neutro è un elemento comune, quindi è sempre contenuto.
	\subitem Stab $+$: Siano $ \underline{v},\underline{w} \in \bigcap_{L \in \mathbb{L}} L \Rightarrow \forall L \in \mathbb{L} \Rightarrow \underline{v},\underline{w} \in L \Rightarrow \underline{v}+\underline{w} \in \bigcap_{L \in \mathbb{L}} L $
	\subitem Stab $\cdot$: Siano $ \underline{v} \in \bigcap_{L \in \mathbb{L}} L, h \in \mathbb{R} \Rightarrow \forall L \in \mathbb{L} \Rightarrow \underline{v},\underline{w} \in L \Rightarrow \underline{v}+\underline{w} \in \bigcap_{L \in \mathbb{L}} L $
\end{itemize}

\subsection{Sottospazio Generato}
Sia $ S \subseteq V $, indicheremo con $<S>$ il \textbf{sotto spazio generato da S}.\\
$$ <S> = \bigcap_{L \in \mathbb{L}_s} L$$
In altri termini: è il più piccolo sottospazio rispetto all'intersezione.

\subsubsection{Esempi:}
\begin{itemize}
\item[•] $ <H> = H $ SEMPRE!
\item[•] $ <\{\underline{0}\} = \{\underline{0}\} $
\item[•] $ <V> = V $ 
\item[•] $ <\emptyset> = {0} $ Singleton dell'elemento neutro, poiché appartiene ad ogni elemento.
\end{itemize}

$ S= H \cup K $ con $H,K \underline{<} V$
$$ <H \cup K> = H+K = \{\underline{h}+\underline{k} / \underline{h} \in H, \underline{k} \in K \} $$
Dim:
Come sempre per dimostrare l'uguaglianza dobbiamo dimostare la doppia inclusione:
$$ <H \cup K> \subseteq \textbf{al contrario } H+K = \{\underline{h}+\underline{k} / \underline{h} \in H, \underline{k} \in K \} $$
non ho capito\\

Dimostriamo che sia spazio vettoriale:
\begin{itemize}
\item[Neutro] $$ \underline{0} = \underline{0}^{\text{preso da H}} + \underline{0}^{\text{preso da K}} $$
\item[Stabile $+$] $$ (\underline{h} + \underline{k}) + (\underline{h}^' + \underline{k}^') \in H+K $$
$$ (h+h^') + (k+k^') $$
\item[Stabile $\cdot$] $$ \alpha(\underline{h}+\underline{k}) = \alpha\underline{h} + \alpha \underline{k}  $$
\end{itemize}

\subsubsection{Esempio}
$$ H=\{(0,k) / x \in \mathbb{R}\} \;\;\; K=\{(y,0) / y \in \mathbb{R} \} $$
$$ <H \cup K> = H+K = (0+y, x+0) = \mathbb{R}^2 $$

\subsection{Propietà Sottospazio Generato}
Posto $H,K \underline{<} V$, allora valgono le seguenti propietà:
\begin{itemize}
\item[•] $ H \quad K = H \cap K = \{ \underline{0} \} \;(\text{neutro}) $ Si dicono in somma diretta.
\item[•] $ H + K = V$ allora $H,K$ si dicono supplementari
\item[•] $ H \quad K = V$ allora si dicono complementari (in altri termini devono essere in somma diretta e supplementari).\footnote{È un concetto un po' strano, perché vuol dire somma normale (quindi caso 2), ma ricandoci che l'intersezione da il neutro (quindi caso 1)}
\end{itemize}

\subsubsection{Esempio}
Posti $\{ \underline{0} \}$ e $V$:
\begin{itemize}
\item[Somma diretta]: $ \{ \underline{0} \} \quad V = \{ \underline{0} \} \cap V = \{ \underline{0} \}$
\item[Supplementari]: $  \{ \underline{0} \} + V = V$
\item[Complementare]: Dato che è sia somma diretta che supplementare
\end{itemize}

\subsection{Dipendenza/Indipendenza Lineare}
Sia $V$ uno spazio e vettore e siano $\underline{v}_1, \underline{v}_2, ..., \underline{v} \in V$, sono detti \textbf{linearmente dipendenti} (o legati) $\Leftrightarrow$
$$ \exists \alpha_1, \alpha_2, ..., \alpha_n \neq (0,0,...,0) = \alpha_1\underline{v_1} + ... + \alpha_n+\underline{v}_n = \underline{0}$$
La loro combinazione lineare deve essere il vettore nullo.
Se tali scalari non esistono allora si dice che sono \textbf{linearmente indipendenti} (o liberi), l'unica soluzione valida è quella formata da tutti zero: $0\underline{v}_1+...+0\underline{v}_n = \underline{0}$.
$$ \textbf{Se non sono dipendenti} \Rightarrow \textbf{Sono indipendenti}$$

\subsubsection{Esempio}

\subsection{Propietà dipendenza lineare}
\begin{itemize}
\item[1)] $\underline{0}$ dipende smpre da qualunque sistema
$$ \underline{0} = 0\underline{v}_1+...+0\underline{v}_n $$
\item[2)] $ \forall i \underline{v}_i $ dipende da $ \underline{w}_1,...,\underline{w}_n$ \footnote{Una specie di transitività della dipendenza}
$$  $$
\end{itemize}







