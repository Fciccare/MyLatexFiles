\section{Lezione 12 - 19/04/2023}

\subsection{Teorema di Rouché-Capelli}
Il teorema di Rouché-Capelli anche noto come secondo sistema di compatibilità afferma che un sistema $S$ è compatibile $\Leftrightarrow$ il rango della matrice incompleta è uguale al rango della matrice completa.
$$
\syssubstitute{A{a_{11}}B{a_{21}}C{a_{m1}}D{a_{1n}}E{a_{2n}}F{a_{mn}}X{x_{n}}}
\systeme{
  A x_1 +...+ D X  = c_1,
  B x_1 +...+ E X = c_2,
  C x_1 +...+ F X = c_n
}
\; \text{è compatibile} \Leftrightarrow
v(A) = v(A^{\prime})
$$

\textbf{DIM $\Rightarrow$:}
Per il primo principio compatibilità:
$$ 
\begin{pmatrix}
c_1 \\ \vdots \\ c_n
\end{pmatrix}
=
y_1 \begin{pmatrix}
a_{11} \\ \vdots \\ a_{n1}
\end{pmatrix}
+...+
y_n \begin{pmatrix}
a_{1m} \\ \vdots \\ a_{nm}
\end{pmatrix}
$$
$$ (c_1,...,c_n) \; \text{DIPENDE DALLE COLONNE}$$
DA FINIRE
\textbf{DIM $\Leftarrow$:}
Partiamo da "i due ranghi sono uguali" allora per il teorema degli orlati hanno lo stesso ordine/grado $(L)$.\\
Consideriamo una matrice:
$$
\begin{pmatrix}
a_{11} & \dots & a_{1n} &\aug& c_1 \\
\vdots & \ddots & \vdots &\aug& \vdots \\
a_{n1} & \dots & a_{nm} &\aug& c_n 
\end{pmatrix}
$$
Consideriamo un minore fondamentale della matrice incompleta (base), ma dalla ipotesi lo è anche per la matrice completa.\\
Allora per il teorema degli orlati: le colonne del minore fondamentale formano un sistema di generatori di tutta la matrice allora il sistema è compatibile-

\subsubsection{Esempio Parametrico}
Consideriamo il seguente sistema:
$$ 
\systeme{
x+y+zh = 2,
2x+3y+z = 1,
x+5y+z = 0
}
$$
Abbiamo un fattore \textbf{parametrico} cioé $h$, vogliamo sapere per quali valori di $h$ il sistema è compatibile:\\
Potremmo procedere con la risoluzione a gradini ma ci viene più facile con Rusché-Capelli, costruiamo la matrice:
$$ 
\begin{pmatrix}
1 & 1 & h & \aug & 2 \\
2 & 3 & 1 & \aug & 1 \\
1 & 5 & 1 & \aug & 0
\end{pmatrix}
$$
Comincio a trovare il minore fondamentale della matrice incompleta, come sempre iniziamo dal "basso" quindi consideriamo $1$ nella posizione $3;1$ e andiamo ad orlarlo:
$$
A(2,3;1,2) = 
\begin{pmatrix}
2 & 3 \\
1 & 5
\end{pmatrix}
$$
Dalla definizione di minore fondamentale abbiamo bisogno che il suo determinante sia diverso da zero ($2*5-1*3 \neq 0$) e che tutti i suoi orlati abbiano determinante uguale a 0, in questo caso l'unico orlato possibile (della matrice incompleta) è:
$$
A(1,2,3;1,2,3) = 
\begin{pmatrix}
1 & 1 & h \\
2 & 3 & 1 \\
1 & 5 & 1
\end{pmatrix}
$$
Per calcolare il determinante usiamo $Sorrus$ ci verrà:
$$ 3+1+10h - 3h-2-5 = 7h-3 $$
Essendo che il determinante è anch'esso parametrico abbiamo due casi:
\begin{itemize}
\item[] SE $h=\frac{3}{7}$ allora il $det = 0$ quindi il rango sarà $2$
\item[] SE $h \neq \frac{3}{7}$ allora il $det \neq 0$ quindi il rango sarà $3$
\end{itemize}
(Ovviamente nel caso $h \neq \frac{3}{7}$ il minore fondamentale è la matrice stessa)
Ora dobbiamo trovare il rango della matrice completa, come sempre dobbiamo trovare un minore fondamentale:
$$ 
A(1,2,3;1,2,4) =
\begin{pmatrix}
1 & 1 & 2 \\
2 & 3 & 1 \\
1 & 5 & 0
\end{pmatrix}
$$
Il determinante è diverso da zero quindi è un minore fondamentale, quindi il rango della matrice completa è $3$.\\
Per il teorema di Rusché-Capelli il sistema è compatibile se e solo se il rango delle due matrice combacia ma essendo il rango della matrice incompleta parametrico abbiamo due casi:
\begin{itemize}
\item[]SE $h = \frac{3}{7}$ $ 3 \neq 2 $ INCOMPATIBILE
\item[]SE $h \neq \frac{3}{7}$ $ 3 \neq 3 $ COMPATIBILE (rango massimo)
\end{itemize}
Sappiamo che per $h \neq \frac{3}{7}$ il sistema è compatibile ora bisogna trovare le soluzioni, usiamo la seguente tecnica:
$$ AX=C $$
Sappiamo che $A$ ha determinante diverso da zero allora è invertibile allora sfruttando quello che abbiamo visto a 9.3, sappiamo che le soluzioni $X=A^{-1}C$, allora:
$$ 
A^{-1} =  \frac{1}{7h-3} 
\begin{pmatrix}
A_{11} & A_{21} & A_{31} \\
A_{12} & A_{22} & A_{32} \\
A_{13} & A_{23} & A_{33}
\end{pmatrix}
$$
Ricordiamo che $A_{11}...$ sono i complementi algebrichi.
$$ 
A^{-1} =
\begin{pmatrix}
-2 & 1-h5 & 1-3h \\
1 & 1h & 1*2h \\
7 & 4 & 1
\end{pmatrix}
$$
Quindi ora possiamo esprimere $X$ come:
$$
X = \begin{pmatrix}
X \\ X \\ X
\end{pmatrix}
=
\begin{pmatrix}
-2 & 1-h5 & 1-3h \\
1 & 1h & 1*2h \\
7 & 4 & 1
\end{pmatrix}
*
\begin{pmatrix}
2 \\ 1 \\ 0
\end{pmatrix}
$$

\subsection{Regola di Cramer}
La Regola(Metodo) di Cramer ci dà una mano nel trovare le soluzione di sistema di equazione di \textbf{n equazione, n incognite}.\\
Se ci sono $n$ equazioni ed $n$ incognite sappiamo che la matrice
incompleta è una matrice quadrata, e poiché le righe della matrice quadrata sono indipendenti poiché fanno sempre parte del minore fondamentale, la matrice $A$ ha determinante diverso da zero, questo implica che possiamo invertirla, quindi:
$$ AX=C \Rightarrow X=A^{-1}C $$
Esperiamiamo:
$$ 
\begin{pmatrix}
X_1 \\ \vdots \\ X_n
\end{pmatrix}
= 
\frac{1}{detA = |A|}
\begin{pmatrix}
A_{11} & A_{21} & \dots & A_{n1} \\
A_{12} & \ddots & \dots & \vdots \\
A_{1m} & A_{23} & \dots & A_{nm}
\end{pmatrix}
\begin{pmatrix}
C_1 \\ \vdots \\ C_n
\end{pmatrix}
=
$$
$$ 
=
\begin{pmatrix}
C_1A_{11}+C_2A_{21}+...+C_nA_{n1} \\
\vdots \\
C_1A_{1m}+C_2A_{2m}+...+C_nA_{nm} 
\end{pmatrix}
$$
Infine dalla eguaglianza:
$$ x_1 = \frac{c_1A_{11}+...+c_nA_{n1}}{|A|} $$
%$$ \; \vdots $$
$$ x_n = \frac{c_1A_{1n}+...+c_nA_{nm}}{|A|} $$

\subsubsection{Metodo Semplificato}
Un metodo più semplice prevede di far utilizzo di una matrice ausiliaria andando a sostituire alla $i$-sima colonna i termini noti.\\
Nello specifico se vogliamo calcolarci $x_1$ andremo a sostituire alla prima colonna, la colonna dei termini noti in questo modo:
$$
B_1 = 
\begin{pmatrix}
c_{11} & a_{12} & \dots & a_{1n} \\
c_{21} & \ddots & \vdots & \vdots \\
\vdots & \vdots & \ddots & \vdots \\
c_{n1} & \dots & \dots & a_{nn}
\end{pmatrix}
$$
Ne consegue che a $B_2$ andrà sostituita la seconda colonna e così via...\\
Tornando a $B_1$ se andiamo a calcolare il determinante della prima colonna (che è anche il determinante della matrice stessa) otteremo:
$$ c_1A_{11}+...+c_nA_{n1} $$
che è proprio il numeratore della regola di Cramer, quindi possiamo semplificare nel seguenti modo:
$$ x_1 = \frac{|B_1|}{|A|} $$

\subsubsection{Esempio}
Riprendiamo l'esempio di prima:
$$
(h \neq \frac{3}{7}) \; S =  
\systeme{
x+y+zh = 2,
2x+3y+z = 1,
x+5y+z = 0
}
$$
Dato che ci troviamo nella situazione di $n$ equazioni, $n$ incognite possiamo applicare Cramer, ci aiutiamo usando la matrice ausilaria:
$$ 
B_1 = 
\begin{pmatrix}
2 & 1 & h \\
1 & 3 & 1 \\
0 & 5 & 1 
\end{pmatrix} \; |B_1| = 5h-5
\; \; \; \; \; x_1 = \frac{5h-5}{7h-3} 
$$

$$
x_2 = \frac{h-1}{7h-3} 
$$

$$
x_3 = \frac{10}{7h-3} 
$$









