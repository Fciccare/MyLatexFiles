\section{Lezione 12 - 19/04/2023}

\subsection{Teorema di Rouché-Capelli}
Il teorema di Rouché-Capelli anche noto come secondo sistema di compatibilità afferma che un sistema $S$ è compatibile $\Leftrightarrow$ il rango della matrice incompleta è uguale al rango della matrice completa.
$$
\syssubstitute{A{a_{11}}B{a_{21}}C{a_{m1}}D{a_{1n}}E{a_{2n}}F{a_{mn}}X{x_{n}}}
\systeme{
  A x_1 +...+ D X  = c_1,
  B x_1 +...+ E X = c_2,
  C x_1 +...+ F X = c_n
}
\; \text{è compatibile} \Leftrightarrow
v(A) = v(A^{\prime})
$$

\textbf{DIM $\Rightarrow$:}
Per il primo principio compatibilità:
$$ 
\begin{pmatrix}
c_1 \\ \vdots \\ c_n
\end{pmatrix}
=
y_1 \begin{pmatrix}
a_{11} \\ \vdots \\ a_{n1}
\end{pmatrix}
+...+
y_n \begin{pmatrix}
a_{1m} \\ \vdots \\ a_{nm}
\end{pmatrix}
$$
$$ (c_1,...,c_n) \; \text{DIPENDE DALLE COLONNE}$$
DA FINIRE
\textbf{DIM $\Leftarrow$:}
Partiamo da "i due ranghi sono uguali" allora per il teorema degli orlati hanno lo stesso ordine/grado $(L)$.\\
Consideriamo una matrice:
$$
\begin{pmatrix}
a_{11} & \dots & a_{1n} &\aug& c_1 \\
\vdots & \ddots & \vdots &\aug& \vdots \\
a_{n1} & \dots & a_{nm} &\aug& c_n 
\end{pmatrix}
$$
Consideriamo un minore fondamentale della matrice incompleta (base), ma dalla ipotesi lo è anche per la matrice completa.\\




