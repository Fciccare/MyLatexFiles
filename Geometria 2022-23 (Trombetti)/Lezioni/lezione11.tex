\section{Lezione 11 - 14/04/2023}

\subsection{Operazioni di Riga riguarda il Determinante}
Le operazioni di riga preservano la non nullità del determinante ma non il valore.
\begin{itemize}
\item[$E_1$]: Scambiare due rige fa cambiare il segno
\item[$E_2$]: Bisogna moltiplicare per uno scalare anche il determinante (da rivdere)
\end{itemize}

\subsection{Determinante di una Matrice a Gradini}
Il modo più semplice per riusciure a calcolare il determinante di una matriace è portarla a gradini, poiché il determinante è \textbf{il prodotto della diagonale principale}.\\
$$ 
\begin{pmatrix}
1 & 0 & 0 \\
0 & 2 & 2 \\
0 & 0 & 3
\end{pmatrix} 
\Rightarrow det = 1*2*3 = 6 \Rightarrow Indipendente
$$

\subsection{Invertilità di una Matrice}
Presa una matrice $A \in \mathbb{R}_{n,n}$ esiste una matrice $B \in R_{n,n}$ tale che:
$$ AB = I_n = BA $$
Vale solo se il determinante è diverdo da zero.
$$ |A| \neq 0 \Leftrightarrow \exists inversa B=A^{-1} $$
$A^{-1}$ è unica.
Dim unicità:
$$ B_1=B_1I_N=B_1(AB_2)=(B_1A)B_2=I_nB_2=B_2 $$
Dim $\Rightarrow$:
$$ AB = I_n \Rightarrow det(AB) = det(I_n) = 1 $$
Dim $\Leftarrow$:
TODO:FINIRE

\subsection{Minore di una Matrice (sottomatrice quadrata)}
Indichiamo il minore come:
$$ A_{(i_1,...,i_h; j_1,...,j_n)} $$
\begin{itemize}
\item[]$i_1,...,i_h$ indica le righe
\item[]$j_1,...,j_n$ indica le colonne
\end{itemize}
Consideriamo la seguente matrice:
$$
\begin{pmatrix}
2 & 3 & 0 \\
0 & 1 & 2 \\
1 & 0 & 0
\end{pmatrix}
$$
$$ A_{(2,3;2,3)} = \begin{pmatrix}
1 & 2\\
0 & 0
\end{pmatrix} $$
I minori valgono anche sulle matrice rettangolari ma i minori rimangono sottomatrici quadrate.\\

\subsection{Grado Massimo}
Un minore si dice di \textbf{grado massimo} se il suo grado coincide con $min\{n,m\}$.\\
Una matrice non quadrata ha sempre più di un minore di ordine massimo, mentre una matrice quadrata ha un solo minore di ordine massimo ed è la matrice stessa.
\subsubsection{Esempio}
$$
\begin{pmatrix}
1 & 1 & 1 & 1 \\
2 & 2 & 2 & 2 \\
0 & 0 & 1 & 1 
\end{pmatrix}
$$
Consideriamo:
$$ A_{(2,3;2,4)} = \begin{pmatrix}
2 & 2 \\ 0 & 1
\end{pmatrix}$$
Questo non di grado massimo poichè il grado massimo è $min\{3,4\} = 3$

\subsection{Orlato}
Se abbiamo un minore che non è di grado massimo, possiamo orlarlo aggiungendo una riga e una colonna.\\
Un orlato rimane un minore, e si più orlare un orlato.\\
Riprendiamo l'esempio di sopra, eravamo rimasti che $A_{(2,3;2,4)}$ non fosse di grado massimo, andiamo ad orlarlo:
$$ A_{(1,2,3;2,3,4)} = \begin{pmatrix}
1 & 1 & 1 \\
2 & 2 & 2 \\
0 & 1 & 1 \\
\end{pmatrix} $$
Abbiamo raggiunto una matrice di grado massimo orlando cioè abbiamo aggiunto la $1$ riga e la $3$ colonna.

\subsection{Minore Fondamentale}
Si definisce \textbf{minore fondamentale} un minore che rispetta queste propietà:
\begin{itemize}
\item[1)] $det \neq 0$
\item[2)] Tutti i suoi orlati hanno $det = 0$  
\end{itemize}
\textbf{ESISTONO SEMPRE I MINORI FONDAMENTALI}\\
Non ci sono sempre minori fondamentali, ad esempio la matrice nulla non ha minori fondamentali, ma, se la matrice non è nulla allora esiste sempre almeno un minore fondamentale. Il minore fondamentale non è necessariamente unico.\\
Se un minore di ordine massimo ha determinante diverso da zero allora esso è un minore fondamentale

\subsubsection{Algoritmo per trovare Minore Fondamentale}
Un modo semplice per trovare un \textbf{minore fonmentale} è cominciare sempre da un minore molto piccolo e poi andare ad orlarlo fino a raggiugere il grado massimo.\\
Cominciamo con prendere una matrice:
$$ \begin{pmatrix}
1 & 1 & $\hlcyan{1}$ & 0 \\
2 & 3 & 4 & 0 \\
1 & 1 & 1 & 0 
\end{pmatrix} $$
Andiamo a prendere un minore "piccolo" con determinante diverso da zero:
$$ A_{(1;3)} = (1) \; \text{con det} \neq 0 $$
Abbiamo escludo gli "zeri" poiché il loro determinante è zero.\\
Procediamo con nostro "algoritmo" andando ad orlarlo:
$$ A_{(1;3)} \rightarrow A_{(1,2;3,4)} = \begin{pmatrix} 1 & 0 \\ 4 & 0 \end{pmatrix} det = 0 $$
Avendo trovato $det = 0 = (1*0)-(0*4)$ dobbiamo scegliere un altro orlato:
$$ A_{(1;3)} \rightarrow A_{(1,2;2,3)} = \begin{pmatrix} 1 & 1 \\ 3 & 4 \end{pmatrix} det = 1 $$
Abbiamo raggiunto un buon candidato ora dobbiamo vericare che tutti i suoi orlati abbiano $det = 0$, in questo caso il suo unico orlato è:
$$ A_{(1,2;3,4)} \rightarrow A_{(1,2,3;1,2,3)} = \begin{pmatrix}
 1 & 1 & 1 \\
 1 & 3 & 4 \\
 1 & 1 & 1
\end{pmatrix} \; \text{ unico orlato con }  det=0$$
Quindi in definitiva:
$$ A_{(1,2;2,3)} = \begin{pmatrix} 1 & 1 \\ 3 & 4 \end{pmatrix} $$ 
$$ \textbf{MINORE FONDAMENTALE}$$

\subsection{Teorema degli Orlati (NO DIM)}
Sia $A \in \mathbb{R}_{n,m} $ matrice rettangolare e $A(i_1,...,i_h; j_1,...,j_n)$ minore fondamentale allora $\Rightarrow$:
$$ \underline{a}_{1,1},...,\underline{a}_{i,n}$$
Sono una base dello spazio vettoriale generato dalle \textbf{righe}.\\
(Esiste equivalente per colonne)
Conseguanza di ciò:
$$ \textbf{dim Righe Generate = dim Colonne Generate} $$

\subsubsection{Rango}
$$ \textbf{Rango = dim(Minore Fondamentale)} $$
Da questo sappiamo che i pivot di una matrice a gradini corrisponde alla dimensione, quindi al \textbf{rango}.

\subsubsection{Corollari derivati}
\begin{itemize}
\item[•] Il rango di riga è sempre uguale al rango di colonna, inoltre il rango di $A$ è uguale al numero di pivot di una matrice a gradini equivalente per righe.
\item[•] Tutti i minori fondamentali hanno lo stesso grado
\item[•] Il determinante di una matrice quadrata è diversa da zero se e solo se le righe (o colonne) sono indipedenti
\item[•] $det A = 0 \Leftrightarrow \; \text{righe (o colonne) dipendenti}$
\end{itemize}

\subsection{Criteri di compatibilità sistemi di equazioni lineare}
Consideriamo un sistema lineare:
$$
\syssubstitute{A{a_{11}}B{a_{21}}C{a_{m1}}D{a_{1n}}E{a_{2n}}F{a_{mn}}X{x_{n}}}
\systeme{
  A x_1 +...+ D X  = c_1,
  B x_1 +...+ E X = c_2,
  C x_1 +...+ F X = c_n
}
$$
Possiamo scriverlo in forma matrice nel seguenti modo $AX=C$:
$$ A = \begin{pmatrix} a_{11} & \dots & a_{1n} \\ \vdots & \ddots & \vdots \\ a_{n1} & \dots & a_{n1} \end{pmatrix} \; X = \begin{pmatrix} x_1 \\ \vdots \\ x_n \end{pmatrix} \; C =  \begin{pmatrix} c_1 \\ \vdots \\ c_n \end{pmatrix} $$
Possiamo esprimere per $C$ e spezzarla:
$$ 
C = x_1 \begin{pmatrix} a_{11} \\ \vdots \\ a_{n1} \end{pmatrix} +...+ x_n \begin{pmatrix} a_{1m} \\ \vdots \\ a_{nm} \end{pmatrix}
$$
Quindi se c’è una soluzione la colonna dei termini noti quindi se c’è una soluzione la colonna dei termini noti.\\
\subsubsection{Primo Criterio di Compatibilità}
Un sistema S  è compatibile $\Leftrightarrow$ la colonna dei termini noti è combinazione lineare della matrice incompleta (è soluzione).






