\section{Lezione 11 - 14/04/2023}

\subsection{Operazioni di Riga riguarda il Determinante}
Le operazioni di riga preservano la non nullità del determinante ma non il valore.
\begin{itemize}
\item[$E_1$]: Scambiare due rige fa cambiare il segno
\item[$E_2$]: Bisogna moltiplicare per uno scalare anche il determinante (da rivdere)
\end{itemize}

\subsection{Determinante di una Matrice a Gradini}
Il modo più semplice per riusciure a calcolare il determinante di una matriace è portarla a gradini, poiché il determinante è \textbf{il prodotto della diagonale principale}.\\
$$ 
\begin{pmatrix}
1 & 0 & 0 \\
0 & 2 & 2 \\
0 & 0 & 3
\end{pmatrix} 
\Rightarrow det = 1*2*3 = 6 \Rightarrow Indipendente
$$

\subsection{Inveribilità}
Presa una matrice $A \in \mathbb{R}_{n,n}$ esiste una matrice $B \in R_{n,n}$ tale che:
$$ AB = I_n = BA $$
Vale solo se il determinante è diverdo da zero.
$$ |A| \neq 0 \Leftrightarrow \exists inversa B=A^{-1} $$
$A^{-1}$ è unica.
Dim unicità:
$$ B_1=B_1I_N=B_1(AB_2)=(B_1A)B_2=I_nB_2=B_2 $$
Dim $\Rightarrow$:
$$ AB = I_n \Rightarrow det(AB) = det(I_n) = 1 $$
Dim $\Leftarrow$:
TODO:FINIRE

\subsection{Minore di una Matrice (sottomatrice quadrata)}
Indichiamo il minore come:
$$ A_{(i_1,...,i_h; j_1,...,j_n)} $$
\begin{itemize}
\item[]$i_1,...,i_h$ indica le righe
\item[]$j_1,...,j_n$ indica le colonne
\end{itemize}
Consideriamo la seguente matrice:
$$
\begin{pmatrix}
2 & 3 & 0 \\
0 & 1 & 2 \\
1 & 0 & 0
\end{pmatrix}
$$
$$ A_{(2,3;2,3)} = \begin{pmatrix}
1 & 2\\
0 & 0
\end{pmatrix} $$
I minori valgono anche sulle matrice rettangolari ma i minori rimangono sottomatrici quadrate.\\

\subsection{Grado Massimo}


\subsection{Orlato}
Se abbiamo un minore che non è di grado massimo, possiamo orlarlo aggiungendo una riga e una colonna.\\
Un orlato rimane un minore, e si più orlare un orlato.\\
Riprendiamo l'esempio di sopra, eravamo rimasti che $A_{(2,3;2,4)}$ non fosse di grado massimo, andiamo ad orlarlo:
$$ A_{(1,2,3;2,3,4)} = \begin{pmatrix}
1 & 1 & 1 \\
2 & 2 & 2 \\
0 & 1 & 1 \\
\end{pmatrix} $$
Abbiamo raggiunto una matrice di grado massimo orlando cioè abbiamo aggiunto la $1$ riga e la $3$ colonna.\\

\subsection{Minore Fondamentale}

