\section{Lezione 17 - 10/05/2023}

\subsection{Criteri di Diagonalizzazione}
Sia $f: V_n \rightarrow V_n$ un endomorfismo è diagonalizzabile $\Leftrightarrow$ $V_n$ ammette una base datta da autovettori.\\
\textbf{DIM $\Rightarrow$:}\\
Supponiamo esista un riferimento $\mathtt{R}$ e consideriamo la matrice associata:
$$ 
\exists \mathtt{R}=(\underline{e}_1,\underline{e}_2,...,\underline{e}_n): M_{\mathtt{R}}(f) = \begin{pmatrix}
a_1 & 0 & 0 \\
0 & \ddots & 0 \\
0 & 0 & a_n
\end{pmatrix}
$$
Andiamo fare l'immagini dei valori del riferimento come combinazione lineare rispetto le colonne
$$ 
f(\underline{e}_1) = a_1\underline{e}_1 + 0\underline{e}_2 + ...
f(\underline{e}_2) = a_2\underline{e}_2
...
f(\underline{e}_n) = a_n\underline{e}_n
$$
Quindi i valori del riferimento sono tutti \textbf{autovettori}.

\textbf{DIM $\Leftarrow$:}
Supponiamo quindi di avere un riferimento di autovettori
TODO: RIASCOLARE AUDIO


\subsection{Esercizio: Trovare gli autovalori}
Consideriamo il seguendo endomorfismo:
$$ f:\mathbb{R}^3 \rightarrow \mathbb{R}^3  $$
$$ (x,y,z) \rightarrow (-y,x,z) $$
Prendiamo un riferimento in questo caso quello canonico/naturale:
$$ \mathtt{R} = ((1,0,0),(0,1,0),(0,0,1)) $$
Costruiamoci la matrice associate delle immagini per colonna
$$ 
M_{\mathtt{R}} = \begin{pmatrix}
0 & -1 & 0 \\ 1 & 0 & 0 \\ 0 & 0 & 1 
\end{pmatrix}
$$
Andiamo a sotrarre la matrice diagonale con una variabile $t$:
$$ 
\begin{pmatrix}
-t & -1 & 0 \\ 1 & 0-t & 0 \\ 0 & 0 & 1-t 
\end{pmatrix}
$$
E andiamo a calcolarci il determinante 
$$ 
det (\begin{pmatrix}
-t & -1 & 0 \\ 1 & 0-t & 0 \\ 0 & 0 & 1-t 
\end{pmatrix}) 
= (1-t)(t^2+1) = 0
$$
Quello che ci siamo ricavati è il \textbf{polinomio caratteristico}, l'unica sua soluzione è:
$$ t = 1 \;\;\; \textbf{unico AUTOVALORES}$$

\subsubsection{Trovare Autovettore}
Avendo trovato che l'unico autovare è $t=1$ andiamo a sostiturlo nella matrice associata:
$$
\begin{pmatrix}
-1 & -1 & 0 \\ 1 & -1 & 0 \\ 0 & 0 & 0 
\end{pmatrix}
$$
Adesso andiamo a considerare il sistema di equazioni omogeneo:
$$ 
\begin{pmatrix}
-1 & -1 & 0 \\ 1 & -1 & 0 \\ 0 & 0 & 0 
\end{pmatrix}
\begin{pmatrix}
x \\ y \\ z
\end{pmatrix}
$$
$$
\systeme{-x-y=0, x-y=0, 0=0}
$$
Il sistema è già risolto e quindi l'insieme delle soluzioni è:
$$ 
S = \{(0,0,z)/ z \in \mathbb{R} \} = <(0,0,1)>
$$

\subsection{Autospazio relativo ad autovalore h}
Consideriamo il seguente endomorfismo $f: V_n \rightarrow V_n$ e $h$ un autovalore, definisco $V(h)$ lo \textbf{spazio relativo all'autovalore h}:
$$ V(h) = \{\underline{v}/f(\underline{v}) = h\underline{v}\} = 
\{\{\text{autovettori di autovalore h} \} \cup \{\underline{0}\} \} 
$$
È un sottospazio vettoriale (dimostazione da fare a casa :()
\subsubsection{Proprietà}
Elenchiamo le seguenti propietà:
\begin{itemize}
\item[•] $dim V(h) \ge 1 $ poiché c'è almeno un autovettore
\item[•] $V(h)$ isomorfo a $\{X/(A-\underbrace{hI_n}_{M_{\mathtt{R}}(f)})X=0 \} = \overline{S} $ mediante cordinazione associata
\item[•] Se $h \neq k$ allora $V(h) \cap V(k) = \{\underline{0}\}$\\
\textbf{DIM:}\\
Consideriamo $\underline{v} \neq \underline{0} \in  V(h) \cap V(k)$, allora può essere autovettore di autovalore di $h,k$, ma poiché un autovettore può avere un unico e solo autovalore si arriverebbe all’assurdo che $h = k$ quindi $\underline{v} = \underline{0}$ quindi l'intersezioni deve essere vuota.
\item[•] Se $\underline{v}_1,...,\underline{v}_n$ sono autovattore di autovalori $h_1,...,h_n$ autovalori \textbf{distinti}, allora sono \textbf{indipendenti}.\\
\textbf{DIM:}\\
Poniamo per assurdo che non siano indipendenti, allora questo vuol dire che:
$$ \underline{v}_1 = a_2\underline{v}_2+...+a_n\underline{e}_n $$
$$ f(\underline{v}_1) = h_1\underline{v}_1 $$
Sviluppiamo mantendendo l'uguaglianza:
$$ a_2f(\underline{v}_2)+...+.a_nf(\underline{v}_n) = h_1a_2\underline{v}_2+...+h_1a_n\underline{v}_n $$
$$ a_2h_2\underline{v}_2+...+a_nh_2\underline{v}_n = h_1a_2\underline{v}_2+...+h_1a_n\underline{v}_n $$
$$ a_ih_1 = a_ih_i $$ 
$$ h_1 = h_i$$
Siamo arrivando ad uno assurdo poiché distinti

\end{itemize}

\subsection{Corolarrio}
Consideriamo l'endomorfismo $f: V_n \rightarrow V_n$ allora:
$$ f \; \text{ammette n autovalori distinti} \rightarrow f \; \text{diagonalizzabile}  $$
\textbf{DIM:}\\
$$ \underline{v}_1,...,\underline{v}_n $$
Essendo distinti per quanto visto prima allora sono indipendenti, ma sempre per quanto osservato sono base di $V_n$, quindi è \textbf{diagonalizzabile}\\
\paragraph{Esempio}
Consideriamo $f: \mathbb{R}^3 \rightarrow \mathbb{R}^3$ quindi $n=3$
$$ (x,y,z) \rightarrow \begin{pmatrix}
2 & 0 & 0 \\ 1 & 3 & 0 \\ 0 & 0 & 1 
\end{pmatrix}
\begin{pmatrix}
x \\ y \\ z
\end{pmatrix} $$
Andiamo a calcolare il terminante andando a sottrarre alla diagonale $t$:
$$
det \begin{pmatrix}
2-t & 0 & 0 \\
1 & 3-t & 0 \\
0 & 0 & 1-t
\end{pmatrix}
= (1-t)(3-t)(2-t)
$$
Quindi le soluzioni del polinomio caratteristico sono $t=1,2,3$ essendo $3=n$ e distinti allora è \textbf{diagonalazzabile}

\subsection{Molteplicità Algebrica}
Considerato un generico polinomio:
$$ a_nx^n + a_{n-1}x^{n-1}+...+a_1x+a_0 = 0 $$
Per il teorema fondamentale dell'algebra:
$$ (x-c_1)...(x-c_n) \in \mathbb{C} $$
Possiano riscriverlo come:
$$ (x-c_1)^{n_1}...(x-c_n)^{n_m} $$
Quindi $n_i$ sarà la nostra molteplicità algebrica

\subsection{Relazione Moltiplicità Algebrica e Geometria}
Oltre la molteplicità algebrica che indicheremo con $m_a(h)$, defininiamo la \textbf{molteplicità Geometrica}, come la dimensione di uno \textbf{autospazio relativo ad h}:
$$ dimV(h) = m_g(h) $$
Inoltre fissiamo una relazione tra molteplicità algebrica e geometriaìca:
$$ 1 \le m_g(h) \le m_a(h) $$
\textbf{DIM:}\\
Prendiamo una base di $V(h)$ e completiamola a $V_n$s
$$ \underline{v}_1,...,\underline{v}_e \; \text{base di V(h)}$$
$$ \underline{v}_1,...,\underline{v}_e,...,\underline{v}_n \; \text{estesa a } V_n $$
Andiamo a prendere un riferimento:
$$\mathtt{R}=(\underline{v}_1,...,\underline{v}_e,...,\underline{v}_n)$$
E calcoliamo la matrice associata:
$$ M_{\mathtt{R}}(f) = 
\begin{pmatrix}
h & 0 & 0 & * & * \\
0 & \ddots & \vdots & * & * \\
0 & 0 & h_e & * & * \\
\vdots & \vdots & \vdots & * & * \\
0 & 0 & 0 & * & * 
\end{pmatrix}
$$
Consideriamo solo la "sottomatrice" formata dala base fino a $\underline{v}_e$
$$ f(\underline{v}_1) = h\underline{v}_1 $$
Andiamo a calcolare il determinante:
$$ det M_{\mathtt{R}}(f) = 
\begin{pmatrix}
h-t & 0 & 0\\
0 & h-t & 0 \\
0 & 0 & h-t \\
\end{pmatrix} = \underbrace{(h-t)(h-t)...(h-t)}_{\text{l-volte}}  - detB $$
(Con $det B$ indichiamo la matrice che siamo andati ad escludere)
$$ (h-t)^{l} * det B $$
Quindi $l$ sarà la nostra molteplicità algebrica, che sarà:
$$ m_g(h) \le l \le m_a(h) $$

\paragraph{PROPOSIZIONE:} Se la molteplicità algebrica è $1$ anche quella geometria lo è.
$$ m_a(h)=1 \Rightarrow m_g(h) = 1 $$

\subsubsection{Esempio}
Consideriamo l'endomorfismo $f: \mathbb{R}^3 \rightarrow \mathbb{R}^3$
$$ (x,y,z) \rightarrow \begin{pmatrix}
1 & -1 & 2 \\
-1 & 1 & 1 \\
0 & 0 & 2
\end{pmatrix}
\begin{pmatrix}
x \\ y \\ z
\end{pmatrix} $$
Avendo già la matrice associata andiamo a calcolarci il determinante:
$$ 
det 
\begin{pmatrix}
1-t & -1 & 2 \\
-1 & 1-t & 1 \\
0 & 0 & 2-t
\end{pmatrix}
= (2-t)*[(1-t*1-t)+(-1*-1)] = (2-t)*(1-t)^2+1 = -(2-t)^2-t
$$
Quindi abbiamo i seguenti autovalori/radici:
$$t=2 \rightarrow 2 \; \text{(molteplicità algebrica)} $$
$$t=0 \rightarrow 1 \; \text{(molteplicità algebrica e geometria)} $$
Dato che non sappiamo il valore geometrico $t=2$ allora andiamo a calcolarlo:
$$ \begin{pmatrix}
-1 & -1 & 2 \\
-1 & -1 & 1 \\
0 & 0 & 0 \\
\end{pmatrix}
\begin{pmatrix}
x \\ y \\ z
\end{pmatrix} = \begin{pmatrix}
0 \\ 0 \\ 0
\end{pmatrix} $$
Adesso possiamo andare a scrivere il sistema omogeneo:
$$ 
\systeme{-x-y-2z=0, -x-y+z=0}
$$
Quindi $z=0 \rightarrow x=-y$, quindi il sistema delle soluzioni è:
$$ \overline{S}=\{(-y,y,0)/ y \in \mathbb{R} \} = <(-1,1,0)> $$
La dimensione del sottospazio è $dim = 1$\\
\textbf{ATTENZIONE:} Credo ci siano degli errori, considerarlo solo come metodo di risoluzione e non per i valori.





