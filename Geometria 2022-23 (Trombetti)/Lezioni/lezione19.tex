\section{Lezione 19 - 17/05/2023 (Geometria Euclidea)}

\subsection{Spazio Vettoriale dei Vettori Geoetrici Liberi}
Gli spazi vettoriali dei vettori geometrici liberi nel piano(x,y) oppure nello spazio(x,y,z) sono classi di equipollenza in cui figurano
tutti i vettori equipollenti ad un dato vettore geometrico.\\
$$ \overline{AB} \; \text{CLASSI DI EQUIPOLLENZA} $$

\subsection{Paralleli}
Presi due vettori $ \underline{0} \neq \underline{a}, \underline{b} \in V_{\pi}$ allora si dicono paralleli se e solo se:
$$ \underline{a} \; \text{È PARALLELO} \; \underline{b} \Leftrightarrow \vec{AB} // \vec{CD} $$
Con la notazione $\vec{AB}$ indichiamo i vettori rappresentati.\\
\paragraph{Paralleli $\Rightarrow$ Dipendenti}
Preso un punto origine $O$ allora:
$$ \equalto{\underline{a}}{\overline{OA}} // \equalto{\underline{B}}{\overline{OB}} \Leftrightarrow \vec{OA} // \vec{OB} \Leftrightarrow \vec{OA}, \vec{OB} \; \text{giaciano sulla stessa retta} \Leftrightarrow$$
 
$$\Leftrightarrow \exists \alpha \neq \underline{0} \; \vec{OA} = \alpha \vec{OB} \Leftrightarrow \underline{0}=\alpha\underline{b} \Leftrightarrow \underline{a}, \underline{b} \; \text{SONO DIPENDENTI} $$
Il vettore degenere $\vec{OO}$ è sempre parallelo.\\

\subsection{Angolo}
Presi $\underline{a}, \underline{v} \in V_{\pi}$ definiamo l'angolo compreso nel seguente modo:
$$ \equalto{\underline{a}}{\overline{OA}} \wedge \equalto{\underline{b}}{\overline{OB}} = \; \text{misura in radianti dell'angolo convesso ($\le $180) formato dai due vettori} $$
TODO:INSERIRE RAPPRESENTAZIONE GRAFICA

\subsection{Prodotto Scalare Standard tra due Vettori}
Definiamo il prodotto scalare standard nel seguente modo:
$$ \underline{a} \cdot \underline{b} = |\underline{a}||\underline{b}|cos(\underline{a} \wedge \underline{b}) $$
È bilineare, simmetrice definita positiva:
\begin{itemize}

\item[]\textbf{Bilineare:}\\
$$ \underline{a} \cdot (h\underline{b}+k\underline{c}) = h(\underline{a} \cdot \underline{b}) + k (\underline{a} \cdot \underline{c}) $$

\item[]\textbf{Simmetrica:}\\
$$ \underline{a} \cdot \underline{b} = \underline{b} \cdot \underline{a} $$

\item[]\textbf{Positiva:}\\
$$ \underline{a} \cdot \underline{a} = \underline{0} \Leftrightarrow \underline{a} = \underline{0} $$
DIM $\Leftarrow$: BANALE\\
DIM $\Rightarrow$: Angolo=$0^{\circ}=(cos 0^{\circ}) = 1 \Rightarrow |a^2| = 0 \Rightarrow |\underline{a}| = \underline{0}$

\end{itemize}
$$\underline{a} \cdot \underline{a} = |\underline{a}^2| \Rightarrow |\underline{a}| = \sqrt{\underline{a} \cdot \underline{a}}$$

\subsection{Perpendicolari/Ortogonali}
$$ \underline{a} \bot \underline{b} \Leftrightarrow \vec{AB} \bot \vec{CB} $$
Prendiamo un punto $0$ chiamato origine ed esprimiamo i vettori, sappiamo che un vettori perpendicolari formano un angolo di $90^{\circ}$, quindi:
$$ \underline{a} \bot \underline{b} \Leftrightarrow \vec{OA} \bot \vec{OB} \Leftrightarrow \vec{OA} \wedge \vec{OB} = \frac{\pi}{2} \Leftrightarrow cos(\vec{OA} \wedge \vec{OB}) = 0 \Leftrightarrow \underline{a} \cdot \underline{b} = 0 $$

\subsection{Riferimento Ortonormale}
Con ortonormale andiamo ad indicare un vettore che è sia un \textbf{versore} che \textbf{ortogonale}(perpendicolare).\\
\textbf{Un Versore} è un vettore unitario cioè: $|\underline{v}| = 1$ \\
Preso un riferimento nel piano $(\underline{e}_x, \underline{e}_y)$ i vettori devono essere ortogonali quindi:
$$ \underline{e}_x \cdot \underline{e}_y = 0 \;\;\;\; |\underline{e}_x|=1 \;\;\;\; |\underline{e}_y|=1 $$ 
Invece preso un riferimo nello spazio $(\underline{e}_x,\underline{e}_y,\underline{e}_z)$ i vettori devono essere ortoganali quindi:
$$ \underline{e}_x \cdot \underline{e}_y = 0 \;\;\;\; \underline{e}_x \cdot \underline{e}_z = 0 \;\;\;\; \underline{e}_z \cdot \underline{e}_y = 0 $$
TODO: AGGIUNGERE BOH NON HO CAPITO CHE CI STA SCRITTO

\subsection{Riduzione angolo Spazio Vettoriale}
BOH

\subsection{Riferimento Cartesiano Ortogonale Goniometrico}
Fissata una origine $O$ possiamo definire il r.c.o.g come la coppia costituita dall'origine e il riferimento ortonormale del piano/spazio:
$$ (0,(\underline{e}_x,\underline{e}_y)) $$
$$ \underline{e}_x = OX = (1,0) \;\;\;\;\; \underline{e}_y = OY = (0,1) $$
TODO:INSERIRE GRAFICO (punti unitare ecc...)

\paragraph{Versore della Retta:} parallelo alla retta

\subsection{Proposizione}
$$ A(x_1,y_1), B(x_2,y_2) \Leftarrow \overline{AB}(x_2-x_1, y_2-y-1) $$
\textbf{DIM:}
$$ \overline{OA}(x_1,y_1) \;\;\;\; \overline{OB}(x_2,y_2) $$
$$ \overline{OA}+\overline{OB} = \overline{OB}$$
$$ \overline{AB} = \overline{OB} - \overline{OA} $$
$$ (x,y) = (x_2,y_2)-(x_1,y_1) = (x_2-x_1, y_2-y_1) $$

\subsection{Cambio di Riferimento}
TODO:INSERIRE

\subsection{Prodotto Vettoriale (Spazio)}
Presi due riferimenti $\underline{V}$ e $\underline{W}$ definiti nel segunte modo:
$$ \underline{V}(v_x,v_y,v_z) \;\;\;\;\; \underline{W}(w_x,w_y,w_z) $$
Allora il prodotto cartesiano sarà uguale:
$$ \underline{V}x\underline{W} = \begin{pmatrix}
\underline{e}_x & \underline{e}_y & \underline{e}_z \\
v_x & v_y & v_z \\
w_x & w_y & w_z
\end{pmatrix} =
(v_yw_z - v_zw_y)\underline{e}_x - (v_xw_z - v_zw_x)\underline{e}_y + (v_xw_y - v_yw_x)\underline{e}_z $$
(Il determinante sviluppato per la prima riga)\\
Esempio: $\underline{V}(0,1,2)x\underline{W}(1,2,0) =$
$$ 
\begin{pmatrix}
\underline{e}_x & \underline{e}_y & \underline{e}_z \\
0 & 1 & 2 \\
1 & 2 & 0 
\end{pmatrix}
= -3\underline{e}_x-2\underline{e}_y+1\underline{e}_z
$$
\paragraph{Proposizione:} $VxW = \underline{0}$ se parallelo oppure dipendenti
DIM:\\
\paragraph{Proposizione:} $(VxW) \bot V=0$
Dim:\\
\paragraph{Proposizione:} 
$$ \underline{e}_x x \underline{e}_y = \underline{e}_z $$
$$ \underline{e}_z x \underline{e}_y = \underline{e}_x $$
$$ \underline{e}_z x \underline{e}_x = \underline{e}_y $$




