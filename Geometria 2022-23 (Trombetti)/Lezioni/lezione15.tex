\section{Lezione 15 - 03/05/2023}

\subsection{Teorema Applicazioni Lineari}
Consideriamo un applicazione lineare $f: V \rightarrow W$ essa si dice \textbf{nota} quando sono noti i corrispondenti dei vettori di una base (se conosco come agisce la funzione sui vettori della base allora possa risalire alla funzione).

\subsubsection{Esempio}
Consideriamo $f: \mathbb{R}^2 \rightarrow \mathbb{R}^3$, prendiamo uno base per $\mathbb{R}^2$ per semplicità scegliamo quella canonica $<(1,0),(0,1)>$:
$$ f(1,0) = (2,0,0) \;\;\;\;\;\;\; f(0,1) = (1,0,0) $$
Sapendo una base per $\mathbb{R}^2$ e note l'immagini possiamo risalire alla funzione:
$$ (x,y)=x(1,0)+y(0,1) $$
$$ f(x,y)=x(2,0,0)+y(1,0,0) = (2x+y,0,0)$$

\subsection{Matrice di Passaggio Applicazioni Lineari}
Consideriamo un applicazione lineare $f: V \rightarrow W$ e prendiamo un riferimento nel dominio e codominio, andiamo a scrivere per ogni vettore la sua immagine tramite combinazione lineare:
$$ \mathtt{R}=(\underline{e}_1,...,\underline{e}_n) \;\;\; \text{rif. di V} $$
$$ \mathtt{R}^{\prime}=(\underline{e}_1^{\prime},...,\underline{e}_n^{\prime}) \;\;\; \text{rif. di W} $$

$$ f(\underline{e}_1) = a_{1,1}\underline{e}^{\prime}_1+a_{2,1}\underline{e}^{\prime}_2+...+a_{m,1}\underline{e}^{\prime}_n... $$
$$ ... f(\underline{e}_n) = a_{1,n}\underline{e}^{\prime}_1+a_{2,n}\underline{e}^{\prime}_2+...+a_{m,n}\underline{e}^{\prime}_n $$

Possiamo notare come gli indici di $a$ siano espressi come colonna, quindi possiamo costruire la \textbf{matrice associata ad $f$ nei rif di $\mathtt{R}$ e $\mathtt{R}^{\prime}$}:
$$
\begin{pmatrix}
a_{1,1} & a_{1,2} & \dots & a_{1,n} \\
a_{2.1} & \ddots & \ddots & \vdots \\
\vdots & \ddots & \ddots & \vdots \\
a_{m,1} & \dots & \dots & a_{m,n}
\end{pmatrix}
= M_{\mathtt{R},\mathtt{R}^{\prime}}(f)
$$

Adesso possiamo considerare la funzione matrice $F_A: \mathbb{R}^n \rightarrow \mathbb{R}^m$:

$$ 
(x_1,...,x_n) \rightarrow
\begin{pmatrix}
a_{1,1} & a_{1,2} & \dots & a_{1,n} \\
a_{2.1} & \ddots & \ddots & \vdots \\
\vdots & \ddots & \ddots & \vdots \\
a_{m,1} & \dots & \dots & a_{m,n}
\end{pmatrix}
\begin{pmatrix}
x_1 \\ \vdots \\ x_n \\
\end{pmatrix}
$$
TODO: Aggiungere spiegazione qua:
$$ \underline{v} \in V, \; \underline{v}=x_1\underline{e}_1+...+x_n\underline{e}_n $$
Adesso possiamo passare dalle componenti in $\mathtt{R}$ di $\underline{v}$ alle componenti di $f(\underline{v})$ in $\mathtt{R}^{\prime}$

$$ 
(x_1,...,x_n) \rightarrow F_A(x_1,...,x_n)
$$
Andiamo a dimostrare:
$$f(\underline{v}= x_1f(\underline{e}_1)+...+x_nf(\underline{e}_n) $$
$$ x_1(a_{1,1}\underline{e}_1^{\prime}+...+a_{n,1}\underline{e}_n^{\prime}) + ... + x_n(a_{1,n}\underline{e}_1^{\prime}+...+a_{m,n}\underline{e}_n^{\prime}) $$
$$ (x_1a_{1,1}+...+x_na_{1,n})\underline{e}_1^{\prime} +...+ (x_1a_{n,1}+...+x_na_{m,n})\underline{e}_n^{\prime} \; \; \; \text{per unicità di scrittura} $$
TODO: CONTINUARE

\subsection{Esempio}
Consideriamo la seguente funzione lineare: $f: \mathbb{R}_2[x] \rightarrow M_2(\mathbb{R})$, andiamo a prendere una base per $\mathbb{R}_2[x]$ per semplicità quella canonica $<1,x,x^2>$.
$$ 1 \rightarrow \begin{pmatrix}0 & 0 \\ 0 & 1 \end{pmatrix} $$
$$ x \rightarrow \begin{pmatrix}0 & 0 \\ 0 & 0 \end{pmatrix} $$
$$ x^2 \rightarrow \begin{pmatrix}1 & 0 \\ 0 & 1 \end{pmatrix} $$
$$ f(ax^2+bx+c) = a\begin{pmatrix}1 & 0 \\ 0 & 1 \end{pmatrix} +b \begin{pmatrix}0 & 0 \\ 0 & 0 \end{pmatrix} +c \begin{pmatrix}0 & 0 \\ 0 & 1 \end{pmatrix} = \begin{pmatrix} a & 0 \\ 0 & a+c \end{pmatrix}  $$
Calcoliamo i valori per la matrice di passaggio:
$$ 
\mathtt{R}=(x^2,x,1) \;\;\;\;\; \mathtt{R}^{\prime}=(
\begin{pmatrix}
1 & 0 \\ 0 & 0
\end{pmatrix},
\begin{pmatrix}
0 & 1 \\ 0 & 0
\end{pmatrix},
\begin{pmatrix}
0 & 0 \\ 1 & 0
\end{pmatrix},
\begin{pmatrix}
0 & 0 \\ 0 & 1
\end{pmatrix}) $$

$$ f(x^2) = \begin{pmatrix}1 & 0 \\ 0 & 1 \end{pmatrix} = 1*\begin{pmatrix}
1 & 0 \\ 0 & 0
\end{pmatrix}
+0*
\begin{pmatrix}
0 & 1 \\ 0 & 0
\end{pmatrix}
+0*
\begin{pmatrix}
0 & 0 \\ 1 & 0
\end{pmatrix}
+1*
\begin{pmatrix}
0 & 0 \\ 0 & 1
\end{pmatrix} $$

$$ f(x) = \begin{pmatrix}0 & 0 \\ 0 & 0 \end{pmatrix} = 0*\begin{pmatrix}
1 & 0 \\ 0 & 0
\end{pmatrix}
+0*
\begin{pmatrix}
0 & 1 \\ 0 & 0
\end{pmatrix}
+0*
\begin{pmatrix}
0 & 0 \\ 1 & 0
\end{pmatrix}
+0*
\begin{pmatrix}
0 & 0 \\ 0 & 1
\end{pmatrix} $$

$$ f(1) = \begin{pmatrix}0 & 0 \\ 0 & 1 \end{pmatrix} = 0*\begin{pmatrix}
1 & 0 \\ 0 & 0
\end{pmatrix}
+0*
\begin{pmatrix}
0 & 1 \\ 0 & 0
\end{pmatrix}
+0*
\begin{pmatrix}
0 & 0 \\ 1 & 0
\end{pmatrix}
+1*
\begin{pmatrix}
0 & 0 \\ 0 & 1
\end{pmatrix} $$
Adesso poossiamo costruire la matrice di passaggio:
$$
A = M_{\mathtt{R},\mathtt{R}^{\prime}}(f) = 
\begin{pmatrix}
1 & 0 & 0 \\
0 & 0 & 0 \\
0 & 0 & 0 \\
0 & 0 & 1 
\end{pmatrix}
$$
Ora possiamo considerare la fuzione matrice:
$$ F_A: \mathbb{R}^3 \rightarrow \mathbb{R}^4 $$
$$ (x,y,z) \rightarrow A \begin{pmatrix}x \\ y \\ z \end{pmatrix} = (x,0,0,x+z)$$
Adesso grazie alla funzione matrice possiamo passare dal dominio al codominio e viceversa in modo facile:
$$ \underbrace{x^2+x+1}_{(1,1,1)} \rightarrow \underbrace{\begin{pmatrix}1 & 0 \\ 0 & 2\end{pmatrix}}_{(1,0,0,2)}  $$ 

\subsection{Matrice di Passaggio Invertibile}
La matrice di passaggio da $\mathtt{R}^{\prime}$ ad $\mathtt{R}$ è invertibile (la sua inversa risulta essere proprio la matrice di passaggio da $\mathtt{R}^{\prime}$ ad $\mathtt{R}$).




