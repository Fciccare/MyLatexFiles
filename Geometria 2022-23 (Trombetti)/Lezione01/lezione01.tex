\section{Lezione 01 - XX/03/2023}

\subsection{Definizioni di base}

\subsubsection{Prodotto Cartesiano}

\begin{equation*}
S,T \neq \emptyset $$ $$ SxT = \{(s,t)/ s \in S, t \in T\}
\end{equation*}

\begin{equation*}
S^2 = SxS = \{(s,t)/ s \in S, t \in T\}
\end{equation*}


\subsubsection{Coppie}
La definizione di coppia è la seguente:
\begin{equation*}
(s,t) = \{\{s,t\}, \{s\}\}
\end{equation*}

Negli insiemi l'ordine non conta $ \{s,t\} = \{t,s\}$, invece nelle coppie è rilevante, infatti due coppie sono uguali se e solo  sono ordinatamente uguali:
$$ (s,t) = (s^1, t^1) \leftrightarrow s=s^1\:,\: t=t^1 $$

Andiamo a dimostrare questa affermazione: 

\begin{itemize}
\item DIM $\leftarrow$: BANALE
\item DIM $\rightarrow$:
	\begin{itemize}
		\item[a] SE $s=t$
		\item[b] SE $s \neq t$
	\end{itemize}
\end{itemize}



