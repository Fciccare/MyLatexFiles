\documentclass[12pt,a4paper]{article}
\usepackage[a4paper, total={6in, 9.5in}]{geometry}
\usepackage[utf8]{inputenc}
\usepackage[italian]{babel}
\usepackage{eucal}
\usepackage{cancel}
\usepackage{mathtools}
\usepackage{amsfonts}
\usepackage{systeme}
\usepackage{blindtext}
\usepackage{hyperref}
\usepackage{stackengine}
\usepackage{soul}
\usepackage{color}
\newcommand{\mathcolorbox}[2]{\colorbox{#1}{$\displaystyle #2$}}
\DeclareRobustCommand{\hlcyan}[1]{{\sethlcolor{cyan}\hl{#1}}}
\DeclareRobustCommand{\hlblue}[1]{{\sethlcolor{blue}\hl{#1}}}
\newcommand{\verteq}{\rotatebox{90}{$\,=$}}
\newcommand{\equalto}[2]{\underset{\scriptstyle\overset{\mkern4mu\verteq}{#2}}{#1}}
\newcommand\aug{\fboxsep=-\fboxrule\!\!\!\fbox{\strut}\!\!\!}
\hypersetup{linktoc=all}

\title{Guida Esercizi Geometria 2022-23 (Trombetti)}
\author{}
\date{}

\begin{document}
\maketitle
\tableofcontents

\newpage

\section{Spazi Vettoriali}

\subsection{Intersezione Spazi Vettoriali}
Presi due sottospazi $H$ e $K$, un generico vettore $\underline{v}$ appartiene alla loro intersezione se e solo se si può esprimere CONTEMPORANEAMENE come combinazione lineare di $H$ e $k$.\\
Esempio:\\
$$ H=<(1,2,3),(1,2,4)> \;\;\;\;\; K=<(0,0,1),(1,3,2)> $$
Possiamo esprimere $\underline{v}$ nei seguenti modi:
$$ \underline{v}=\alpha_1(1,2,3)+\beta_1(1,2,4) \;\;\;\;\; \underline{v}=\alpha_2(0,0,1)+\beta_2(1,3,2)$$
Come abbiamo detto prima queste due condizioni devo verificarsi CONTEMPORAMENTE quindi imponiamo l'uguaglianza:
$$ 
\systeme{1\alpha_1+1\beta_1 = 0\alpha_2+1\beta_1, 2\alpha_1+2\beta_1=0\alpha_2+3\beta_2, 3\alpha_1+4\beta_1 = 1\alpha_2+2\beta_2}
\Rightarrow
\systeme{\alpha_1+\beta_1 = \beta_1, 2\alpha_1+2\beta_1=3\beta_2, 3\alpha_1+4\beta_1 = \alpha_2+2\beta_2}
$$
Andando a risolvere il sistema (con il metodo che più ci aggrada) verrà:
$$
\systeme{\alpha_1=-\beta_1, \beta_1=\beta_1, \alpha_2=\beta_1, \beta_2=0}
$$
Possiamo notare come il sistema dipenda da $\beta_1$.\\
L'intersezione sarà:
$$ \underline{v}=\{(0,0,\beta_1)\} $$
Questo perché andando a sostuire i valori precedentemente trovati ci viene:
$$ \underline{v}=\beta_1(0,0,1)+\cancel{0(1,3,2)} \;\;\;\;\; \underline{v}=-\beta_1(1,2,3)+\beta_1(1,2,4) $$
Andando a sviluppare entrambe le espressioni:
$$ \underline{v}=\beta_1(0,0,1)+\cancel{0(1,3,2)} = (0,0,\beta_1)$$
$$\underline{v}=-\beta_1(1,2,3)+\beta_1(1,2,4) = (-1\beta_1,-2\beta_1,-3\beta_1)+(\beta_1,2\beta_1,4\beta_1) = (0,0,\beta_1) $$
Scegliendo la soluzione più semplice cioé $\beta_1=1$ avremo:
$$ H \cap K = (0,0,1) \;\;\; dim=1 $$

\end{document}
