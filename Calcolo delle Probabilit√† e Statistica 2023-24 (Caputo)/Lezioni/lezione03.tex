\section{Lezione 03 - 13/03/2023}

\subsection{Permutazioni}
Ogni n-disposizione semplice è detta essere una permutazione degli n elementi di S\\
(Possiamo considerare le permutazioni un caso speciale delle 
\hyperref[sec:disposizioni]{disposizioni semplici}, cioè avviene quando $n=k$)
$$ (se \: k=n) \: \: \: P_n = D_{n,n} = \frac{n!}{\equalto{(n-n)!}{0!=1}} = n!  $$

\subsection{Permutazioni con Ripezioni}
Sia $n=k_1+k_2+...+k_r$, Una n-selezione di S avente $k_1$ elementi uguali al primo elemento di S, $k_2$ elementi uguali al secondo elemento di S e cosi via fino a $k_r$ è detta una $(k_1,k_2,...,k_r)$-permutazioni con ripetizioni.\\
Il numbero di tutte le $(k_1,k_2,...,k_r)$-permutazioni con ripetizioni di S è dato da:
$$ P^{(r)}_n = \frac{n!}{k_1! * k_2! * ... * k_r!} = \binom{n}{k_1,...,k_r} \: \: (k_1+...+k_r)=r$$

\subsection{Esempi Permutazioni}
$$ S=\{A,I,O,S\} \: \#S=4 \: k=n=4  $$
Possiamo formare varie parole: OASI, SAIO, SOIA..., possiamo calcolarle:
$$ P_4 = 4! = 24  $$

Poniamo caso che vogliamo sapere le possibili combinazioni di $ARCANE$, possiamo notare che la $A$ si ripete 2 volte, per calcore dobbiamo usare:
$$ \frac{6!}{2!} = 360 $$
il $2!$ si riferisce a quante volte appare la lettera $A$.

\subsection{Combinazioni Semplici}
Sia $k<=n$, una k-combinazione semplice di S si ottine indentificando tutte le k-disposizioni semplici di S avente i medesimi elementi posti in differente ordine (in altri termini l'ordine di presentazione degli elementi è ininfluente).\\
Il numero di tutte le k-combinazioni semplici è dato da:
$$ C_{n,k} = \binom{n}{k} \: \: \: (con\:k<=n)$$

\subsection{Combinazioni con Ripetizioni}
Una k-combinazione con ripetizione di S si ottiene identificando tutte le k-disposizioni con ripetizioni di S aventi i medesii elementi posti in un differente ordine (in altri termini è ammessa la ripetizioni di qualche elemento di S e l'ordine è ininfluente).\\
Il numero di tutte le k-combinazioni con ripetizioni di S è dato da:
$$ C_{n,k}^{(r)} = \binom{n+k-1}{k} $$

\subsection{Esempi}
\blindtext

