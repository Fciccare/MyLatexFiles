\section{Lezione 05 - 16-03-2023}

\subsection{Definizioni simboli Insiemestici ed Eventi}

\resizebox{\columnwidth}{!}{
\begin{tabular}{ |c|c|c| } 
 \hline
 Begin & Algebra degli Insiemi & Logica degli Eventi \\ 
 $\Omega$ & Insieme universo & Spazio Campione \\ 
 $A \in F$ & Insieme & Evento \\
 $A^C$ & Complementare di A & Si verifica quando non si verifica A \\
 $A \cup B$ & Unione di A e B & OR degli eventi, deve verificarli almeno uno tra A e B \\
 $ A \cap B $ & Intersezione tra A e B & AND degli eventi, devono verificarsi entrambi \\
 $ \bigcup_{k=1}^n A_k $ & Unione finita & n verifica almeno una tra $A_1, A_2,...,A_n$ \\
 $ \bigcup_{k=1}^{\infty} A_k $ & Unione numerabile & "" \\
 $ \bigcap_{k=1}^n A_k$ & Intersezione finita & Si verifica se tutti gli eventi $A_1,...,A_n$ si verificano \\
 $ \bigcap_{k=1}^{\infty} A_k$ & Unione numerabile & ""  \\
 $ \emptyset $ & Insieme Vuoto & Evento Impossibile \\
 $ A \cap B = \emptyset $ & A e B sono disgiunti & Eventi Incompatibili \\
 $ A \subset B $ & A contenuto in B & Il verificare di finita di due elementiA implica il verificare di B \\
 $ \uplus_k A_k = \Omega $ & Ricoprimento disgiunto (partizione) & $A_1, A_2,...,A_n$ eventi neccessari\\
 \hline
\end{tabular}
}

\subsection{Unione Finita/Numerabile}
Unione finita cioè che preso $n$ finito di insiemi la loro unione è chiusa:
$$ \bigcup_{i=n}^n A_i \rightarrow \;\;\; \text{ciclo finito con inizio e fine}$$
Unione numerabile è un unione di un numero non finito di insiemi ma che si possono contare:
$$ \bigcup_{n=1}^{\infty} A_n \rightarrow  \;\;\; \text{ciclo infinito di unione} $$ 
Tutto questo vale anche per l'intersezione tramite \textbf{de morgan}.

\subsection{Algebra e Sigma Algebra}
Preso un $\Omega$ spazio campione e un $a$ (a tondo), classe non vuota di sottinsiemi di $\Omega$ allora:
$$ a \: \text{è un algebra} \Leftrightarrow $$
$$ i) A \in a \Rightarrow A^C \in a \:\:\: \text{(a è chiusa rispetto il complemento)} $$
$$ ii) A_1, A_2 \in a \Rightarrow A_1 \cup A_2 \in a \:\:\: \text{(a è chiusa rispetto l'unione finita di due elementi)} $$
C'è un anche una sua variante chiamanta Sigma(numerabile) Algebra definita così:\\
\begin{center}
$a$ è una $\sigma$-algebra $\Leftrightarrow $
\end{center}
$$ i)uguale $$
$$ ii) n \in N, A_n \in a \Rightarrow \bigcup_{n=1}^{\infty} A_n \in a \:\:\: \text{(a è chiusa rispetto l'unione numerabile)}$$
Riassumendo:\\
"Un'algebra è chiusa rispetto all'unione/intersezione di due suoi elementi e rispetto al complemento."\\
"Una $\sigma$-algebra è chiusa rispetto all'unione/intersezione numerabile di suoi elementi e rispetto al complemento."
\subsubsection{Osservazioni}
Posto $a=\{\{2,3\}, \{6\}, \{4,5\}\}$, osserviamo i seguenti esempi:
\begin{center}
$ \{4,5\} \subseteq a $ SBAGLIATO\\
$ \{4,5\} \in a $ CORRETTO\\
$ \{\{4,5\}\} \subseteq a $ CORRETTO
\end{center}

\subsubsection{Casi Particolari}
Poniamo $A \subseteq \Omega$, si definisce \textbf{algebra(sigma) banale}, $a$ posto come:
$$ a = \{\emptyset, \Omega \}$$
È l'unica algebra a due elementi, ovviamente entrambe le propietà sono banalmente dimostrate poiché:
$$ \Omega^C = \emptyset $$
$$ \Omega \cup \emptyset = \Omega \in a $$
Gli elementi $\emptyset$ e $\Omega$ sono neccessari per essere un \textbf{algebra}.
Poniamo caso di un $a=\{A, A^c\}$ questa non è un algebra poiché $A \cup A^c = \Omega \not \in a$, se aggiunssimo solo $\Omega$ non sarebbe rispettata la prima condizione poiché $ \Omega^c = \emptyset \not \in a$.\\
Ricapitolando:
$$ a=\{A, A^C\} \:\:\: \textbf{non è algebra} \:\:\:\:\: a=\{A,A^C,\emptyset,\Omega\} \:\:\:\:\:\textbf{è algebra (sigma)}$$
Per contrapposizione la (sigma) algebra più grande è $P(\Omega)$, tutte le altre algebra(sigma) sono sottoinsiemi di $P(\Omega)$

\subsection{Propietà (conseguenze)}
\begin{enumerate}
\item $a$ è una algebra (sigma) $\Rightarrow \: \emptyset,\Omega \in \: a$ (come abbiamo osservato prima)\\Tutti gli elementi dell'algebra banale devono essere presenti in ogni algebra(sigma).
\item L'unione finita di elementi di un algebra (sigma) appartiene comunque ad $a$\\ Per $ii)$ abbiamo visto come l'unione si applica per due elementi, ma essendo $\cup$ associativa nel caso di $n-elementi$ basta operarli a due a due e quindi portare questa propietà fino a n elementi.
\item $ Sigma\:algebra \Rightarrow Algebra \:\:\text{MA}\:\: Sigma\: algebra \not \Leftarrow Algebra $\\
Questo poiché un unione finita da 0 a $+\infty$ non appartiene a tutte le algebra, cose che invece accade con le sigma algebra.
\end{enumerate}







