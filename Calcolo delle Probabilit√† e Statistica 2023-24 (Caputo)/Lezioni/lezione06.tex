\section{Lezionie 06 - 20/03/2023}

\subsection{SigmaAlgebra Generata}
Sia $C$ una classe su $\Omega$, esiste una $\sigma$-algebra $F$ che contiene $C$ ed è contenuta in tutte le $\sigma$-algebra che contengono $C$.\\
Tale minima $\sigma$-algebra contenente $C$ si dice \textbf{GENERATA DA $C$}.\\
In primo luogo esiste, per ogni $C$, una $\sigma$-algebra che la contiene e l'insieme delle parti.
Dopo di ciò:
$$ F = \bigcap_{i \in I} F_i  \text{ con } i \in I $$

\subsubsection{Dimostrazione?}
Sia $ \mathcal{F}_i $ una sigma algebra su $\Omega$, $\forall_i \in \mathcal{I}$\\
Poniamo $\mathcal{F} = \cap_{i \in \mathcal{I}} \mathcal{F}_i$ è una classe di sottoinsiemi di $\Omega$ e in particolare:
$$ \mathcal{F} \text{ è una sigma algebra di } \Omega $$
Andiamo a verificare sia effettivamente una sigma algebra tramite i tre punti:
\begin{itemize}
\item Non Vuota\\
$$ \Omega \in \mathcal{F}_i \;\;\; \forall_i \in \mathcal{I} \Rightarrow \Omega \in \mathcal{F} = \cap_{i \in \mathcal{I}} \mathcal{F}_i $$
\begin{center}
Quindi $\mathcal{F}$ \textbf{è una classe non vuota.}
\end{center}

\item i)\\
$$ A \in \mathcal{F} \Rightarrow A \in \cap_{i \in \mathcal{I}} \mathcal{F}_i \Rightarrow  $$

\item ii)\\

\end{itemize}


\subsubsection{Esempio (Buonocore)}
Poniamo di lanciare due dadi onesti, assumiamo i possibili risultati:
$$ A = \{2,4,6\} \:\:\:\:\:\: B=\{5,6\}$$
Considerando la famiglia $G$ in questo modo:
$$ G = \{A,B\}$$
Possiamo considerare la $\sigma$-algebra generata da una famiglia:
$$ \sigma(G)$$
Per trovarci gli atomi dobbiamo andare a intersercare tutte le possibili combinazioni tra $A$ e $B$:\\
$$ A \cap B = \{6\}$$
$$ A \cap B^C = \{2,4\}$$
$$ A^C \cap B = \{5\}$$
$$ A^C \cap B^C = \{1,3\}$$
Abbiamo trovato \textbf{4 atomi}, per ottenere tutto l'insieme dobbiamo andare a intersecare gli atomi a due a due:
$$ \sigma(G) = \{ \{6\},  \{5\}, \{2,4\},  \{1,3\}, \{5,6\}, \{1,3,5\}, \{2,4,5\}, \{1,3,6\}, \{2,4,6\}, $$ 
$$ \{1,2,3,4\}, \{1,3,5,6\}, \{2,4,5,6\}, \{1,2,3,4,6\}, \{1,2,3,4,5\}, \Omega, \emptyset \}\}  $$
\begin{center}
$ \emptyset,\Omega \in \sigma(G)$ per come abbiamo dimostrato un paio di lezioni fa.
\end{center}

\subsection{Probabilità di Laplace (Classica)}
Sia $\Omega$ è finito, ed un evento appartente a una famiglia di eventi $E \in F$, allora la probabilità dell'evento $E$ si può rappresentare nel seguente modo:
$$ P_c(E) = \frac{\#E}{\#\Omega} $$
Questa è un ottima definizione ma solo se c'è simmetria.
Considerando il caso di due eventi disgiunti cioé che non possono mai verifarsi insieme, in questo caso con l'unione disgiunta cioè che deve verifacarsi o uno o l'altro:
$$ \mathbb{P}(A_1 \cupdot A_2) = \frac{N_{A_1} \cupdot A_2}{N} = \frac{N_{A_1}+N_{A_2}}{N} = \mathbb{P}(A_1)+\mathbb{P}(A_2)$$

\subsection{Probabilità Frequentista (Statistica)}
Se un esperimento aleatorio $E$ si ripete un numero indefinito di volte, possiamo considerare il rapporto:
$$ n \in N \:\:\: P_f(E) = \frac{n_E}{n}$$
$n$ è il numero delle ripetizioni di $E$\\
$n_E$ è il numero delle prove tra le quali $E_n$ si è presentato.\\
Questo definizione però è molto approsimativa, quella più corretta e precisa è:\\
 \[ P_f(E) = \lim_{n\to\infty} \frac{n_E}{n} >= 0 \]
Considerando il caso di due eventi disgiunti cioé che non possono mai verifarsi insieme, in questo caso con l'unione disgiunta cioè che deve verifacarsi o uno o l'altro:
$$ \mathbb{P}(A_1 \cupdot A_2) = \frac{N_{A_1} \cupdot A_2}{N} = \frac{N_{A_1}+N_{A_2}}{N} = \mathbb{P}(A_1)+\mathbb{P}(A_2)$$






