\section{Lezione 10 - 29/03/2023}

\subsection{Variabile Aeleatoria}

\subsection{Lancio moneta un numero indifinità di volte}
Poniamo uno spazio campione: 
$$\Omega_2 = \{T,C\} x \{T,C\} = \{T,C\}^2 = \{TC,TT,CC,CT\}$$
Queste sono $2$-selezioni, possiamo estendere questo caso fino ad $n$:
$$\Omega_n = \{T,C\}^n = \{ w = w_1w_2...w_n, i=1,2,...,n, w_i \in \{T,C\} \}$$
Nel caso nostro dobbiamo considerare l'infinito:
$$ \Omega = \{T,C\}^{\infty} $$
Andiamo a considerare le varie possibilità:
$$ \mathbb{P}(\{TTT...T...\}) = 0 $$
$$ \mathbb{P}(\{T\}) = \frac{1}{2} $$
$$ \mathbb{P}(\{TT\}) = \frac{1}{4} $$
$$ \mathbb{P}(\{TTT\}) = \frac{1}{8} $$
$$ \mathbb{P}(\{TTT...T^n\}) = \frac{1}{2^n} $$
Se consideriamo l'ultimo caso all'aumentare di $n$ quindi delle teste il valore si avvicina sempre di più a $0$, quindi il caso di tutte teste possiamo dire è praticamente zero.\\
Quindi ci verrebbe da dire che la probababilità di tutte teste sia il limite tendente a zero, \textbf{ma questo non è vero!}.
$$ \mathbb{P}(\{TTT...T...\}) = \displaystyle\lim_{n\ \rightarrow\ \infty}\mathbb{P}\{TTT...T\} = \displaystyle\lim_{n\ \rightarrow\ \infty} \frac{1}{2^n} = 0 $$
$$ \mathbb{P(\displaystyle\lim_{n\ \rightarrow\ \infty}\mathbb{P}\{TTT...T\})} $$
Possiamo notare col il limite sia all'interno della probabilità quindi non possiamo affermare che vale quell'uguaglianza, però è intuitivamente corretto.\\
Poniamo una variabile aleatoria che assuma $2$ valori in base al caso:
\begin{equation*}
X_1 = 
\begin{cases}
0 \: \: \: \text{Se esce $C=T^c$ al primo lancio} \\
1 \: \: \: \text{Se esce $T$ al primo lancio}  
\end{cases}
\end{equation*}
Poniamo di lanciare due volte la moneta e una volta esce testa e l'altra volta esce croce:
$$ \mathbb{P}(X_1=1) = \textit{p} $$
$$ \mathbb{P}(X_1=0) = 1-\textit{p} \text{complemento}$$
Con $\textit{p}$ indichiamo la probabilità che esca $T$ ed è compresa tra 0,1 esclusi $\textit{p} \in [0,1]$.\\
La possibilità esce testa o che esca croce è la stessa quindi possiamo dire che sono \textbf{somiglianti}.\\
Poniamo il caso:
\begin{itemize}
\item[$n=2$] $S_2=X_1+X_2$ (somma aeleatoria) abbiamo 3 possibili output:
\item CC = 0 = (1-\textit{p})(1-\textit{p})
\item CT,TC = 1 = 2\textit{p}(1-\textit{p})
\item TT = 2 = \textit{p}\textit{p}
\item[$n=3$]
\end{itemize}





