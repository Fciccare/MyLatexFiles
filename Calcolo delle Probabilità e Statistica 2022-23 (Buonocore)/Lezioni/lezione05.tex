\section{Lezione 05 - 16-03-2023}

\subsection{Definizioni simboli Insiemestici ed Eventi}

\resizebox{\columnwidth}{!}{
\begin{tabular}{ |c|c|c| } 
 \hline
 Begin & Algebra degli Insiemi & Logica degli Eventi \\ 
 $\Omega$ & Insieme universo & Spazio Campione \\ 
 $A$ & Insieme & Evento \\
 $A^C$ & Complementare di A & Negato di A \\
 $A \cup B$ & Unione di A e B & OR degli eventi, deve verificarli almeno uno tra A e B \\
 $ A \cap B $ & Intersezione tra A e B & AND degli eventi, devono verificarsi entrambi \\
 $ \bigcup_{k=1}^n A_k $ & Unione finita & n verifica almeno una tra $A_1, A_2,...,A_n$ \\
 $ \bigcup_{k=1}^{\infty} A_k $ & Unione numerabile & "" \\
 $ \bigcap_{k=1}^n A_k$ & Intersezione finita & Si verifica se tutti gli eventi $A_1,...,A_n$ si verificano \\
 $ \bigcap_{k=1}^{\infty} A_k$ & Unione numerabile & ""  \\
 $ \emptyset $ & Insieme Vuoto & Evento Impossibile \\
 $ A \cap B = \emptyset $ & A e B sono disgiunti & Eventi Incompatibili \\
 $ A \subset B $ & A contenuto in B & Il verificare di A implica il verificare di B \\
 $ \uplus_k A_k = \Omega $ & Ricoprimento disgiunto (partizione) & $A_1, A_2,...,A_n$ eventi neccessari\\
 \hline
\end{tabular}
}


\subsection{Esempio Lancio Moneta 1}
Poniamo caso che vogliamo descrivere l'evento che al terzo lancio di una moneta esca Testa, per prima cosa scegliamo un spazio campione:
$$ \Omega = {\{T,C\}}^{N} $$
Una moneta ha solo due casi, testa oppure croce, ora descriviamo che testa esca al terzo lancio:
$$ T_3 = \{ (w_1,w_2,...) \in \Omega: w_3 = "T" \} $$ 
abbiamo descritto questo eveno tramite propietà degli insiemi, nel caso volessimo esprimere lo stesso concetto ma per le croci ci basta fare il comlemento:
$$ T_3^C = C_3 = \{ (w_1,w_2,...) \in \Omega: w_3 = "C" \}  $$

\subsection{Esempio Lancio Moneta 2}
Poniamo invecere di voler complicare le cose, voglia esprimere l'evento che escano due Testa prima di due Croci, chiamiamo questo evento $A$, questo evento ha infinite possibilità, facciamo alcuni esempi:
$$ A_2 = T_1, T_2, \Omega $$
$$ A_3 = C_1, T_2, T_3, \Omega $$
$$ A_4 = T_1, C_2, T_3, T_4, \Omega $$
$$ A_5 = C_1, T_2, C_3, T_4, T_5 \Omega $$
Possiamo fare alcune osservazioni, $A_2, A_3$ sono incompatibili, non possono verificarci contemporaneamente, invece $A_5$ è incompatibile con $A_2,A_3,A_4$ .
Possiamo esprimere il verificarsi dell'evento $A$ in vari modi:\\
$$ A = A_2 \cup A_3 $$
$$ A = A_2 \cup A_3 \cup A_4 $$
$$ A = A_2 \uplus A_3  \uplus A_4 $$
$$ A = A_2 \uplus A_3 \uplus A_4 \uplus A_5 $$
Possiamo esprimere questo evento $A$ tramite \textbf{Unione Numberabile}:
\begin{equation*}
A_n =
\begin{cases}
C_1,T_2,...,C_{n-2},T_{n-1},T_n  \:\:\:\:\:\text{n dispari inizia con una croce}\\
T_1,C_2,...,C_{n-2},T_{n-1},T_n  \:\:\:\:\:\text{n pari inizia con una testa}
\end{cases}
\Rightarrow A = \bigcup_{n=2}^{\infty} \:\:\:\:\: \text{è un evento}
\end{equation*}

\subsection{Classi/Famiglie}
Quando gli elementi di un insieme $a$ sono a loro volta degli insiemi si usa per $a$ la parola \textbf{classe}.
$$ a = \{ \{2,3\},\{2\},\{5,6\} \}$$
In particolare se $\Omega$ è un insieme, la classe di tutti i sottinisiemi di $\Omega$ si dice l'insieme delle parti di $\Omega$ e si indica con $P(\Omega)$.\\
Se $\Omega$ è un insieme e $a$ è una classe di sottinsimi di $\Omega$ tale che l'unione di essi ha come risultato $\Omega$ allora $a$ è detta essere un \textbf{ricoprimento} di $\Omega$.\\
Un ricoprimento $a$ di $\Omega$ è detto essere una \textbf{partizione} di $\Omega$ se i suoi elementi a due a due disgiunti.\\
Esempio:
 $$ \Omega= \{1,2,3,4,5,6,7,8,9\} $$
 $$ a = \{ \{1,3,5\},\{2,6\},\{4,7\},\{7,8,9\} \} \:\:\: \text{è un ricomprimento ma non una partizione}$$
 $$ a = \{ \{1,3,5\},\{2,4,6,8\},\{7,9\} \} \:\:\: \text{è partizione poiché  tutti gli insiemi sono disgiunti}$$

\subsection{Algebra e Sigma Algebra}
Preso un $\Omega$ spazio campione e un $a$ (a tondo), classe non vuota di sottinsiemi di $\Omega$ allora:
$$ a \: \text{è un algebra} \Leftrightarrow $$
$$ i) A \in a \Rightarrow A^C \in a \:\:\: \text{(a è chiusa rispetto il complemento)} $$
$$ ii) A_1, A_2 \in a \Rightarrow A_1 \cup A_2 \in a \:\:\: \text{(a è chiusa rispetto l'unione di due elementi)} $$
C'è un anche una sua variante chiamanta Sigma(numerabile) Algebra definita così:\\
\begin{center}
$a$ è una $\sigma$-algebra $\Leftrightarrow $
\end{center}
$$ i)uguale $$
$$ ii) n \in N, A_n \in a \Rightarrow \bigcup_{n=1}^{\infty} A_n \in a$$
Riassumendo:\\
"Un'algebra è chiusa rispetto all'unione di due suoi elementi e rispetto al complemento."\\
"Una $\sigma$-algebra è chiusa rispetto all'unione numerabile di suoi elementi e rispetto al complemento."
\subsubsection{Osservazioni}
Posto $a=\{\{2,3\}, \{6\}, \{4,5\}\}$, osserviamo i seguenti esempi:
\begin{center}
$ \{4,5\} \subseteq a $ SBAGLIATO\\
$ \{4,5\} \in a $ CORRETTO\\
$ \{\{4,5\}\} \subseteq a $ CORRETTO
\end{center}

\subsubsection{Casi Particolari}
Poniamo $A \subseteq \Omega$, si definisce \textbf{algebra(sigma) banale}, $a$ posto come:
$$ a = \{\emptyset, \Omega \}$$
È l'unica algebra a due elementi, ovviamente entrambe le propietà sono banalmente dimostrate poiché:
$$ \Omega^C = \emptyset $$
$$ \Omega \cup \emptyset = \Omega \in a $$
Gli elementi $\emptyset$ e $\Omega$ sono neccessari per essere un \textbf{algebra}.
Poniamo caso di un $a=\{A, A^c\}$ questa non è un algebra poiché $A \cup A^c = \Omega \not \in a$, se aggiunssimo solo $\Omega$ non sarebbe rispettata la prima condizione poiché $ \Omega^c = \emptyset \not \in a$.\\
Ricapitolando:
$$ a=\{A, A^C\} \:\:\: \textbf{non è algebra} \:\:\:\:\: a=\{A,A^C,\emptyset,\Omega\} \:\:\:\:\:\textbf{è algebra (sigma)}$$
Per contrapposizione la (sigma) algebra più grande è $P(\Omega)$, tutte le altre algebra(sigma) sono sottoinsiemi di $P(\Omega)$

\subsection{Propietà (conseguenze)}
\begin{enumerate}
\item $a$ è una algebra (sigma) $\Rightarrow \: \emptyset,\Omega \in \: a$ (come abbiamo osservato prima)\\Tutti gli elementi dell'algebra banale devono essere presenti in ogni algebra(sigma).
\item L'unione finita di elementi di un algebra (sigma) appartiene comunque ad $a$\\ Per $ii)$ abbiamo visto come l'unione si applica per due elementi, ma essendo $\cup$ associativa nel caso di $n-elementi$ basta operarli a due a due e quindi portare questa propietà fino a n elementi.
\item $ Sigma\:algebra \Rightarrow Algebra \:\:MA\:\: Sigma\: algebra \not \Leftarrow Algebra $\\
Questo poiché un unione finita da 0 a $+\infty$ non appartiene a tutte le algebra, cose che invece accade con le sigma algebra.
\end{enumerate}







