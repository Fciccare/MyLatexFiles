\section{Lezione 08 - 23/03/2023}

\subsection{Recap}
Sia $\phi$ e la terna: $(\Omega, F, P)$ spazio di probabilità:\\
Assiomi:
\begin{itemize}
\item[a)] $F$ non è vuota
\item[b)] $F$ è stabile rispetto al complemento
\item[c)] $F$ è stabile rispetto a $\cup_{n \in N}()$
\item[d)] $P$ è non negativa e finita
\item[e)] $P(\Omega) = 1$
\item[f)] è qualcosa additiva
\end{itemize}
Teoremi:
\begin{itemize}
\item[1)] $P(\emptyset) = 0$
\item[2)] $P$ è finita additiva
\item[3)] $P[(A)^C] = 1-P(A)$
\item[4)] $P(A) <= 1$
\item[5)] $P$ è monotona
\end{itemize}

\subsection{Spazi di Probabilità Finiti}
$i)$ Posto $\Omega = \{a_1, a_2,...\}, F = \mathbb{P}(\Omega)$, posto $ i=1,2,...,n$ avremo:
$$  \mathbb{P}(\{a_i\}) = \textit{p}_i \;\;\; \textbf{"probabilità di \{$a_i$\}"} $$
Bisognerà avere:
\begin{itemize}
\item[1)] $ \textit{p}_i \underline{>} 0, i=1,2,...n $
\item[2)] $ \textit{p}_1 +   \textit{p}_2 +  ... +  \textit{p}_n  = 1 $
\end{itemize}
La probabilità di un qualsiasi evento $A$ sarà dunque definita come la somma delle probabilità degli elementi di $A$.

\subsection{Spazi Equiprobabili finiti}
$ii)$ Posto $\Omega = \{a_1, a_2,...\}, F = P(\Omega)$, basta porre:
$$ \textit{p}_i = \frac{1}{n} \;\;\; \text{ per } i=1,2,...,n $$
Per ogni evento $A$ la probabilità sarà definita:
$$ \mathbb{P}(A) = \frac{\#A}{n} $$

\subsection{Spazi di Probabilità Numerabili}
$iii)$ $\Omega = \{a_1, a_2,...\}, F = P(\Omega)$
sia $n \in N, p_n:= P({a_n})$
\begin{itemize}
\item[1)] $n \in N, p_n >= 0$
\item[2)] $ \sum_{n=1}^{\infty} p_n = 1$
\end{itemize}
boh: $ \sum_{n=0}^{\infty} h^n = \frac{1}{1-h}$

\subsection{Spazio di Probabilità infiniti non Numerabili (da rivedere)}
$iv)$ $\#\Omega > \#N$ \\
$$ \Omega = (0,1)$$
$$ F = \{ (a,b): a<b ^ a,b \in 0,1 \} $$
Questa cosa si dice \textbf{sigma algebra di barel dell'intervallo 0,1}.\\
$$ P[(a,b)] = \frac{1}{b-a} = b-a $$

\subsection{Esempi Spazi di Probabilità Numberabili}
\subsubsection{Esempio 01}
Si ponga lo scalare $\lambda > 0 $  e $ n \in \mathbb{N}_0 $ definiamo: 
$$ \textit{p}_n = e^{-\lambda} \frac{\lambda^n}{n!} $$
Per verificare che sia una spazio di probabilità \textbf{numerabile}, dobbiamo verificare la $1)$ e la $2)$.
\begin{itemize}
\item[1)] È banalmente verificata poiché sono tutte quantità positive:
$$  e^{- \lambda} > 0, \;\;\; n! > 0, \;\;\; \lambda^n > 0 $$
\item[2)] Basta verificare che la serie di termine generale $\textit{p}_n$ si somma a 1:
$$ \sum_{n=0}^{\infty} \textit{p}_n = \sum_{n=0}^{\infty} e^{-\lambda} \frac{\lambda^n}{n!} = e^{-\lambda} \sum_{n=0}^{\infty} \frac{\lambda^n}{n!} = e^{-\lambda} * e^{\lambda} = 1 $$
L'ultimo passaggio si ottiene ricordando lo sviluppo in serie di Mc Laurin della funzione $e^x$.
\end{itemize}
\subsubsection{Esempio 02}
Si ponga, con $\textit{p} \in (0,1)$, $n \in \mathbb{N}$, $\textit{p}^{\prime}_n = \textit{p} (1-\textit{p}){n-1}$.\\
Procedendo come nell'esempio precedente, basta verificare che la serie di termini generale $\textit{p}^{\prime}_n$ si somma a 1:
$$  \sum_{n=1}^{\infty} \textit{p}^{\prime}_n = \sum_{n=1}^{\infty} \textit{p} (1- \textit{p})^{n-1} = \textit{p} \sum_{n=1}^{\infty} (1-\textit{p})^{n-1} = \textit{p} \sum_{m=0}^{\infty}  (1-\textit{p})^m = \textit{p} \frac{1}{1-(1-\textit{p})} = \textit{p} \frac{1}{\textit{p}} = 1$$
L'ultimo passaggio si ottiene ricordando la serie geometrica di ragione $h \in (-1,1)$ si somma a $\frac{1}{1-h}$

\subsection{Eventi non Incompatibili}
\subsubsection{Teorema 06}
$ \forall A \in F, \forall B \in F $ si ha:
$$ P(A \cup B) = P(A) + P(B) - P(A \cap B)$$
(se $A$ e $B$ sono disgiunti allora abbiamo la finita additività).\\
Dim: 
$$ A \cup B = A \cup (B \cap A^C) $$
Inoltre dalla relazione:
$$ B=B \cap \Omega = B \cap (A \uplus A^C) = (B \cap A) \uplus (B \cap A^C)$$
si ricava, per la finita additività di $P$:
$$ P(B) = P(B \cap A) + P(B \cap A^C)$$
$$ P(B \cap A^C) = P(B) - P(A \cap B)$$
In definitiva:
$$ P(A \cup B) = P(A) + P(B \cap A^C) = P(A) + P(B) - P(A \cap B)$$
Osservazione:\\
Se $A \cap B = \emptyset$ allora $P(A \cap B) = P(\emptyset) = 0$ e il risultato coincide con la finita additività.

\subsubsection{Teorema 07}
$\forall A \in F$, $\forall B \in F$ e $\forall C \in F$ si ha:
$$ P(A \cup B \cup C) = P(A) + P(B) + P(C) \;\; - [P(A \cap B) + P(A \cap C) + P(B \cap C)] \;\; + P(A \cap B \cap C)$$
Dim: \\
Applicando le propietà delle operazioni insiemistiche ed il risultato precedenti si ha:
$$ P(A \cup B \cup C) = P[(A \cup B) \cup C)] = P(A \cup B) + P(C) - P[(A \cup B) \cup C] = $$
$$ P(A) + P(B) - P(A \cap B) + P(C) - P[(A \cap C) \cup (B \cap C)]  = $$
$$ P(A) + P(B) + P(C) - P(A \cap B) - P(A \cap C) - P(B \cap C) + P[(A \cap C) \cap (B \cap C))] = $$
$$ P(A) + P(B) +P(C) - P[(A \cap B) + P (A \cap C) + P(B \cap C)] + P(A \cap B \cap C)$$

\subsubsection{Teorema 08 (Caso Generale)}
Se $A_i \in F$ per $i = 1,2,...,n$ si ha:
$$ P(\bigcup_{i=i}^n A_i) = \sum_{i=i}^n P(A_i) - \sum_{i<j} P(Ai \cap A_j) + \sum_{i<j<k} P(A_i \cap A_j \cap A_k) +...+ (-1)^{n+1} P(A_1 \cap A_2 \cap ... \cap A_n)$$
Questo risultato è noto come \textbf{formula di inclusione-esclusione}.


%\subsubsection{Teorema 09}
%Sia $A_n \in F$, $\forall n \in N$.\\
%Per ogni intero $k$ si ha:
%$$ P(\bigcup_{n=1}^k A_n) <= \sum_{n=1}^k P(A_n) $$
%Dim:\\
%La relazione è vera per $k=2$. Infatti da:
%$$ A_1 \cup A_2 = A_1 \cup (A_2 \cap A_1^C) $$
%per la finita additività di $P$ si ottiene:
%$$ P(A_1 \cup A_2) = P(A_1) + P(A_2 \cap A_1^C) <= P(A_1) + P(A_2) $$
%L'ultimo passaggio deriva dall'essere:
%$$ A_2 \cap A_1^C \subseteq A_2 $$
Supponiamo ora la tesi vera per $k-1$:
%$$ P(\bigcup_{n=1}^k A_n) = P[(\bigcup_{n=1}^k A_n) \cup A_k ] <= P(\bigcup_{n=1}^{k-1} A_n)+P(A_k) <= $$
%$$ <= \sum_{n=1}^{k-1} P(A_n) + P(A_k) = \sum_{n=1}^k P(A_n)$$
%La tesi è vera per il principio di induzione matematica.

%\subsubsection{Teorema 10}
%Sia $A_n \in F$, $\forall n \in N$. Si ha:
%$$  P(\bigcup_{n=1}^{\infty} A_n) <= \sum_{n=1}^{\infty} P(A_n) $$
%Dim:\\
%Poniamo:
%$$ B_1 = A_1 $$
%$$ B_2 = A_2 \cap A_1^C $$
%$$ B_3 = A_3 \cap (A_1^c \cap A_2^c) $$
%$$ ... $$
%$$ B_n = A_n \cap (A_1^c \cap A_2^c ... \cap A_{n-1}^c) = A_n \cap $$

%\subsection{Test}
%$$ \mathbb{P} $$
%$$ \mathbb{N} $$
%$$ \mathbb{F} $$







