\section{Lezione 01 - 06/03/2023}

\subsection{Il Gioco della Zara con 2 Dadi}
Il gioco della Zara consiste nello scegliere un numero e lanciare in questo caso due dadi (il gioco originale ne prevede tre), se la somma dei due dadi corrisponde al numero scelto di vince.
2 dadi onesti danno luogo a 2 punteggi da 1 a 6: $P_1, P_2$\\
Ora rappresentiamo graficamente le coppie di tutti i possibili casi:

\begin{center}
\begin{tabular}{ cc }
%Normale & Trasformata\\

\begin{tabular}{ |c|c|c|c|c|c| } 
 \hline
 (1,1) & (1,2) & (1,3) & (1,4) & (1,5) & (1,6) \\ 
 (2,1) & (2,2) & (2,3) & (2,4) & (2,5) & (2,6) \\ 
 (3,1) & (3,2) & (3,3) & (3,4) & (3,5) & (3,6) \\ 
 (4,1) & (4,2) & (4,3) & (4,4) & (4,5) & (4,6) \\ 
 (5,1) & (5,2) & (5,3) & (5,4) & (5,5) & (5,6) \\ 
 (6,1) & (6,2) & (6,3) & (6,4) & (6,5) & (6,6) \\ 
 \hline
\end{tabular} &

$\xrightarrow[]{Z}$

\begin{tabular}{ |c|c|c|c|c|c| } 
 \hline
 2 & 3 & 4 & 5 & 6 & 7 \\ 
 3 & 4 & 5 & 6 & 7 & 8 \\ 
 4 & 5 & 6 & 7 & 8 & 9 \\ 
 5 & 6 & 7 & 8 & 9 & 10 \\ 
 6 & 7 & 8 & 9 & 10 & 11 \\ 
 7 & 8 & 9 & 10 & 11 & 12 \\ 
 \hline
\end{tabular}\\

\end{tabular}
\end{center}

Possiamo notare che coppie possibili sono 36, poiché ogni dado ha 6 faccie, quindi 6*6=36 possibili risultati.
Formalizziamo nel seguente modo: $$\Omega=\{(1,1), (1,2), ..., (6,6)\}$$
Questo insieme $\omega$ prende il nome di \textbf{SPAZIO CAMPIONE}.\\
Come si può facilmente notare i risultati possibili sono compresi tra 2 e 12 (inclusi), possiamo formalizzarlo nel seguente modo:
$$ S_z = \{2,3,4,5,6,7,8,9,10,11,12\} $$
Questo insieme $S_z$ prende il nome di \textbf{SPETTRO}.\\
La possibilità di trovare un numero non appartente a questo insieme è nulla.\\
Per calcolare la probabiltà ci basta mettere a rapporto i seguenti dati:
$$ \frac{\text{POSSIBILITÀ DI N}}{\text{SPAZIO CAMPIONE}} $$ 
Poniamo che voglia sapere la probabilità che la somma dei 2 dadi faccia 4, allora diremo che la \textbf{LA PROBABILITÀ DELL'EVENTO:}
$$ P(Z=4) = \frac{3}{36} $$
Il 3 è stato ricavato contando quante volte appare 4 nella tabella.\\
Possiano notare che il numero con la più alta probabilità è il 7, poiché figura sei volte, quindi $ \frac{6}{36} $.\\
Possiamo rappresentare la probabilità di ogni numero:
$$ P(Z=2) = \frac{1}{36} = P(Z=12) $$
$$ P(Z=3) = \frac{2}{36} = P(Z=11) $$
$$ P(Z=4) = \frac{3}{36} = P(Z=10) $$
$$ P(Z=5) = \frac{4}{36} = P(Z=9)  $$
$$ P(Z=6) = \frac{5}{36} = P(Z=8)  $$
$$ P(Z=7) = \frac{6}{36} $$
Espriamo $\xi$ come \textbf{ESPERIMENTO AELEATORIO (casuale)}\\
Possiamo riassumere questo esprimento con la seguente terna:
$$ (\Omega, P(\Omega), P) \xrightarrow[]{Z} (R, , P_z) $$

 




